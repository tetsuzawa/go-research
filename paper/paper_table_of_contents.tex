% Options for packages loaded elsewhere
\PassOptionsToPackage{unicode}{hyperref}
\PassOptionsToPackage{hyphens}{url}
%
\documentclass[
]{jsarticle}
\usepackage{lmodern}
\usepackage{amssymb,amsmath}
\usepackage{ifxetex,ifluatex}
\ifnum 0\ifxetex 1\fi\ifluatex 1\fi=0 % if pdftex
  \usepackage[T1]{fontenc}
  \usepackage[utf8]{inputenc}
  \usepackage{textcomp} % provide euro and other symbols
\else % if luatex or xetex
  \usepackage{unicode-math}
  \defaultfontfeatures{Scale=MatchLowercase}
  \defaultfontfeatures[\rmfamily]{Ligatures=TeX,Scale=1}
\fi
% Use upquote if available, for straight quotes in verbatim environments
\IfFileExists{upquote.sty}{\usepackage{upquote}}{}
\IfFileExists{microtype.sty}{% use microtype if available
  \usepackage[]{microtype}
  \UseMicrotypeSet[protrusion]{basicmath} % disable protrusion for tt fonts
}{}
\makeatletter
\@ifundefined{KOMAClassName}{% if non-KOMA class
  \IfFileExists{parskip.sty}{%
    \usepackage{parskip}
  }{% else
    \setlength{\parindent}{0pt}
    \setlength{\parskip}{6pt plus 2pt minus 1pt}}
}{% if KOMA class
  \KOMAoptions{parskip=half}}
\makeatother
\usepackage{xcolor}
\IfFileExists{xurl.sty}{\usepackage{xurl}}{} % add URL line breaks if available
\IfFileExists{bookmark.sty}{\usepackage{bookmark}}{\usepackage{hyperref}}
\hypersetup{
  hidelinks,
  pdfcreator={LaTeX via pandoc}}
\urlstyle{same} % disable monospaced font for URLs
\setlength{\emergencystretch}{3em} % prevent overfull lines
\providecommand{\tightlist}{%
  \setlength{\itemsep}{0pt}\setlength{\parskip}{0pt}}
\setcounter{secnumdepth}{-\maxdimen} % remove section numbering

\author{}
\date{}

\begin{document}

\begin{enumerate}
\def\labelenumi{\arabic{enumi}.}
\tightlist
\item
  序論

  \begin{enumerate}
  \def\labelenumii{\arabic{enumii}.}
  \tightlist
  \item
    ドローンの市場動向について
  \item
    ドローンの特徴と問題点
  \item
    適応フィルタ
  \item
    Raspberry Piについて
  \item
    Raspberry PiをDSPとして使用する
  \item
    先行研究
  \item
    研究目的
  \end{enumerate}
\end{enumerate}

\begin{enumerate}
\def\labelenumi{\arabic{enumi}.}
\setcounter{enumi}{1}
\tightlist
\item
  理論

  \begin{enumerate}
  \def\labelenumii{\arabic{enumii}.}
  \tightlist
  \item
    パラメータ推定法の概要

    \begin{enumerate}
    \def\labelenumiii{\arabic{enumiii}.}
    \tightlist
    \item
      非ベイズ推定法
    \item
      ベイズ推定法
    \end{enumerate}
  \item
    観測系と推定器のモデル

    \begin{enumerate}
    \def\labelenumiii{\arabic{enumiii}.}
    \tightlist
    \item
      線形観測モデル
    \item
      線形推定器
    \end{enumerate}
  \item
    最小二乗平均誤差法 (MMSE法)

    \begin{enumerate}
    \def\labelenumiii{\arabic{enumiii}.}
    \tightlist
    \item
      (xおよびzが結合ガウス分布の場合)
    \end{enumerate}
  \item
    線形MMSE法

    \begin{enumerate}
    \def\labelenumiii{\arabic{enumiii}.}
    \tightlist
    \item
      線形MMSEの導出
    \item
      (直行性と普遍性)
    \end{enumerate}
  \item
    最小二乗法 (LS法)

    \begin{enumerate}
    \def\labelenumiii{\arabic{enumiii}.}
    \tightlist
    \item
      LS法の導出
    \end{enumerate}
  \item
    ウィナーフィルタ

    \begin{enumerate}
    \def\labelenumiii{\arabic{enumiii}.}
    \tightlist
    \item
      ウィナーフィルタの構造
    \item
      ウィナーフィルタの導出
    \item
      (周波数領域でのウィナーフィルタ)
    \end{enumerate}
  \item
    適応フィルタにおけるLS法
  \item
    最急降下法
  \item
    (ニュートン法)
  \item
    最小二乗平均法 (LMS法)
  \item
    学習同定法 (NLMS法)
  \item
    アフィン射影法 (APA法)
  \item
    再帰最小二乗法 (RLS法)
  \item
    適応アルゴリズムの関係
  \item
    アクティブノイズコントロール (ANC)
  \item
    自動等化器
  \end{enumerate}
\end{enumerate}

\begin{enumerate}
\def\labelenumi{\arabic{enumi}.}
\setcounter{enumi}{2}
\tightlist
\item
  ハードウェアの制作

  \begin{enumerate}
  \def\labelenumii{\arabic{enumii}.}
  \tightlist
  \item
    ドローンの初期設定

    \begin{enumerate}
    \def\labelenumiii{\arabic{enumiii}.}
    \tightlist
    \item
      使用するドローンについて
    \item
      組み立て
    \item
      試運転
    \end{enumerate}
  \item
    バイノーラルマイクの制作

    \begin{enumerate}
    \def\labelenumiii{\arabic{enumiii}.}
    \tightlist
    \item
      マイクについて
    \item
      (アンプについて)
    \end{enumerate}
  \item
    Raspberry Piの初期設定

    \begin{enumerate}
    \def\labelenumiii{\arabic{enumiii}.}
    \tightlist
    \item
      OSの選定
    \item
      AD変換用の拡張ボードについて
    \item
      (回路の制作)
    \item
      (オーバークロック)
    \end{enumerate}
  \end{enumerate}
\end{enumerate}

\begin{enumerate}
\def\labelenumi{\arabic{enumi}.}
\setcounter{enumi}{3}
\tightlist
\item
  ソフトウェアの制作

  \begin{enumerate}
  \def\labelenumii{\arabic{enumii}.}
  \tightlist
  \item
    (Pythonによる予備実験)
  \item
    自作ライブラリの制作

    \begin{enumerate}
    \def\labelenumiii{\arabic{enumiii}.}
    \tightlist
    \item
      ライブラリの設計
    \item
      インストール方法
    \item
      使い方
    \end{enumerate}
  \end{enumerate}
\end{enumerate}

\begin{enumerate}
\def\labelenumi{\arabic{enumi}.}
\setcounter{enumi}{4}
\tightlist
\item
  予備実験

  \begin{enumerate}
  \def\labelenumii{\arabic{enumii}.}
  \tightlist
  \item
    ドローンの駆動音のサンプル収音

    \begin{enumerate}
    \def\labelenumiii{\arabic{enumiii}.}
    \tightlist
    \item
      収音方法
    \item
      駆動音の周波数特性について
    \end{enumerate}
  \item
    自作ライブラリのベンチマーク

    \begin{enumerate}
    \def\labelenumiii{\arabic{enumiii}.}
    \tightlist
    \item
      演算速度について
    \item
      使用メモリについて
    \item
      CPUの負荷(Raspberry Pi)
    \end{enumerate}
  \end{enumerate}
\end{enumerate}

\begin{enumerate}
\def\labelenumi{\arabic{enumi}.}
\setcounter{enumi}{5}
\tightlist
\item
  アクティブノイズコントロールの検討

  \begin{enumerate}
  \def\labelenumii{\arabic{enumii}.}
  \tightlist
  \item
    実験方法
  \item
    実験結果
  \item
    ドローンの駆動音に対する各アルゴリズムの収束速度
  \item
    考察
  \end{enumerate}
\end{enumerate}

\begin{enumerate}
\def\labelenumi{\arabic{enumi}.}
\setcounter{enumi}{6}
\tightlist
\item
  自動等化器の検討

  \begin{enumerate}
  \def\labelenumii{\arabic{enumii}.}
  \tightlist
  \item
    実験方法
  \item
    実験結果
  \item
    考察
  \end{enumerate}
\end{enumerate}

\begin{enumerate}
\def\labelenumi{\arabic{enumi}.}
\setcounter{enumi}{7}
\tightlist
\item
  結論
\end{enumerate}

参考文献

謝辞 1. スペースアカデミア

付録 1. 制作したプログラム

\end{document}
