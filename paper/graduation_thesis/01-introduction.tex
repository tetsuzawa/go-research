\chapter{序論}\label{introduction}

\section{研究背景}\label{research-background}

近年, ドローンの開発が進み, 着実に産業として根付いてきている. 
「ドローン (Drone)」はもともと「無人機」全般を指す言葉であるが, 日本では慣例的に「マルチコプター (Multicopter)」を指す. 
ドローンは従来の有人ヘリコプターや大型機と比べて小型・軽量で, 低コストで製造が可能という特徴から, 空撮, 農業, 測量, 災害救助, デリバリー等, 様々な用途を想定して開発が進められている. 

しかしながら, ドローンを使用するうえでは騒音の大きさ, 駆動時間の短さ, 悪天候に対する不適性といった技術的課題も多く残る. 

本研究ではドローンの応用分野の1つとして音声収録に着目し, 収録した音声信号からドローンの駆動音を取り除く手法について検討する. 

\section{研究目的}\label{ux7814ux7a76ux76eeux7684}

本研究で想定する手法は大きく分けて, ドローンの駆動音に対して外部から逆位相の音を発生させ, 駆動音そのものを打ち消すことで低減する手法と, 計算機に入力された音を内部で処理することで駆動音を低減する手法の2つに分かれる. 

これらの手法を実現するためには, 駆動音と目的信号の混合信号から, 駆動音のみを取り除く必要がある. この処理を行うためには, 動的にフィルタの係数を変化させる適応フィルタ(ADF, Adaptive
Filter) を使用ことが一般的である. 

適応フィルタのフィルタ係数を計算するアルゴリズムは複数知られているが, アルゴリズムの収束誤差と収束速度は相反する関係にある. したがって, 実装するハードウェアの規模や, 求められる収束速度などを考慮して, アルゴリズムを選択する必要がある. 

本研究では, 計算機のハードウェアとしてRaspberry Piを, ソフトウェアとしてGo言語で制作したプログラムを実行し, 代表的な適応アルゴリズムを使用した適応フィルタによるドローンの駆動音低減の評価を行うことを目的とする. 
