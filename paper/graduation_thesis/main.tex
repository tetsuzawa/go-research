\newif\ifjapanese

\japanesetrue  % 論文全体を日本語で書く(英語で書くならコメントアウト)

\ifjapanese
  %\documentclass[a4j,twoside,openright,11pt]{jreport} % 両面印刷の場合。余白を綴じ側に作って右起こし。
  \documentclass[a4j,12pt]{jreport}                  % 片面印刷の場合。
  \renewcommand{\bibname}{参考文献}
  \newcommand{\acknowledgmentname}{謝辞}
\else
  \documentclass[a4paper,12pt]{report}
  \newcommand{\acknowledgmentname}{Acknowledgment}
\fi
\usepackage{thesis}
% \usepackage{ascmac}
\usepackage[dvipdfmx]{graphicx}
% \usepackage{epsf}
\usepackage{multirow}
\usepackage{url}

% **************** added by takizawa ******************

\usepackage{longtable}
\usepackage{here}
\usepackage{amsmath}
\usepackage{amssymb}
\usepackage{comment}
\usepackage[labelsep=space]{caption}
\usepackage[subrefformat=parens, labelsep=space]{subcaption}
\usepackage{siunitx}
\usepackage{bm}

% **** ソースコードの表示に関する設定 **** 
\usepackage{listings,jlisting} %日本語のコメントアウトをする場合jlistingが必要

\lstset{
  basicstyle={\ttfamily},
  identifierstyle={\small},
  commentstyle={\smallitshape},
  keywordstyle={\small\bfseries},
  ndkeywordstyle={\small},
  stringstyle={\small\ttfamily},
  frame={tb},
  breaklines=true,
  columns=[l]{fullflexible},
  numbers=left,
  xrightmargin=0zw,
  xleftmargin=3zw,
  numberstyle={\scriptsize},
  stepnumber=1,
  numbersep=1zw,
  lineskip=-0.5ex
}

\makeatletter
    \AtBeginDocument{
    \renewcommand*{\thelstlisting}{\arabic{chapter}.\arabic{section}.\arabic{subsection}-\arabic{lstlisting}}
    \@addtoreset{lstlisting}{section}}
\makeatother
% **** ソースコードの表示に関する設定 **** 

\newcommand{\tabref}[1]{表\ref{#1}}
\newcommand{\equref}[1]{(\ref{#1})式}
\newcommand{\figref}[1]{図\ref{#1}}

\makeatletter
\renewcommand{\theequation}{% 式番号の付け方
\thesection.\arabic{equation}}
\@addtoreset{equation}{section}

\renewcommand{\thefigure}{% 図番号の付け方
\thesection.\arabic{figure}}
\@addtoreset{figure}{section}

\renewcommand{\thetable}{% 表番号の付け方
\thesection.\arabic{table}}
\@addtoreset{table}{section}
\makeatother

\def\tightlist{\itemsep1pt\parskip0pt\parsep0pt}
% **************** added by takizawa ******************


% bibtexの参照設定
% \bibliographystyle{jplain}  % アルファベット順
\bibliographystyle{junsrt}  % 引用順


\bindermode  % バインダー用余白設定

% 日本語情報(必要なら)
\jclass     {卒業論文}                             % 論文種別
\jtitle     {Raspberry Piを使用したドローンの\\駆動音低減法の検討}    % タイトル。改行する場合は\\を入れる
\juniv      {苫小牧工業高等専門学校}                  % 大学名
\jfaculty   {電気電子工学科 第16期生 18番}               % 学部、学科
\jauthor    {瀧澤 哲}                       % 著者
\jhyear     {1}                                   % 令和○年度
\jsyear     {2020}                                 % 西暦○年度
\jkeyword   { }     % 論文のキーワード
\jproject   {指導教員  工藤 彰洋} %プロジェクト名
\jdate      { }


% メインのドキュメント
\begin{document}

\jmaketitle    % 表紙(日本語)

本研究では、ドローンの応用分野の1つとして音声収録に着目し、搭載されたマイクと小型の計算機を使用して収録した音声信号からドローンの駆動音を取り除く手法について検討した。

駆動音の低減法を検討するにあたり、ドローン・バイノーラルマイクの組み立て、Raspberry
Piのセットアップなどハードウェアの準備を行った。 Raspberry
PiにはAD変換が搭載されていないため、拡張ボードとしてPumpkin
Piを使用し、初期設定を行った。

ソフトウェアに関しては、まずPythonを使用して、静的なFIRフィルタ・適応アルゴリズムの評価を行った。次にADFのライブラリをGo言語で自作し、公開した。また、Python・Go言語を使用して波形表示や音声編集用のソフトウェアを制作した。

次にドローンの駆動音に対する各適応アルゴリズムの有効性を検証するために、駆動音のサンプル収音を行い、ADFの収束特性を試験した。結果的にNLMS・APアルゴリズムに比べてRLSアルゴリズムの収束誤差は小さいが、収束速度が遅くリアルタイム処理には向かないことが判明した。

制作したADFライブラリのベンチマークを取ると、Raspberry
Piを計算媒体とした場合、一番高速なNLMSアルゴリズムでも計算速度が遅く、アクティブノイズコントロールの実装は難しいことが判明した。

最後に、実際の使用の際に入力される信号を模擬し、ADFによる雑音低減の効果を検討した。音声の信号が得られたのはSN比0dBのRLSアルゴリズムと一部のフィルタ長のAPアルゴリズムのみであった。

以上より本研究で検討した構成での雑音低減の実現可能性は低いことを確認した。
  % アブストラクト。要独自コマンド、include先参照のこと

\tableofcontents  % 目次
\listoffigures    % 表目次
\listoftables    % 図目次

\pagenumbering{arabic}  % ページ番号

\
\section{序論}\label{ux5e8fux8ad6}}

\
\subsection{研究背景}\label{ux7814ux7a76ux80ccux666f}}

近年、ドローンの開発がすすみ、着実に産業として根付いてきている。
「ドローン」はもともと「無人機」全般を指す言葉であるが、日本では慣例的に「マルチコプター」を指す。
ドローンは従来の有人ヘリコプターや大型機と比べて小型・軽量で、低コストで製造が可能という特徴から、空撮、農業、測量、災害救助、デリバリー等、様々な用途を想定して開発が進められている。

しかしながら、ドローンを使用するうえでは騒音、駆動時間が短さ、悪天候に対する弱さといった技術的課題も多く残る。

本研究ではドローンの応用分野の1つとして音声収録に着目し、収録した音声信号からドローンの駆動音を取り除く方法について検討する。

\
\subsection{研究目的}\label{ux7814ux7a76ux76eeux7684}}

本研究で想定する手法は大きく分けて、ドローンの駆動音に対して外部から逆位相の音を発生させ、駆動音そのものを低減する手法と、計算機に入力された音を内部で処理することで駆動音を低減する手法の2つに分かれる。

これらの手法を実現するためには、駆動音と目的信号の混合信号から、雑音のみを取り除く必要がある。この処理を行うためには、動的にフィルタの係数を変化させる適応フィルタ(ADF、Adaptive
Filter) を使用ことが一般的である。

適応フィルタのフィルタ係数を計算するアルゴリズムは複数知られているが、アルゴリズムの収束誤差と収束速度は相反する関係にある。したがって、実装するハードウェアの規模や、求められる収束速度などを考慮して、アルゴリズムを選択する必要がある。

本研究では、計算機のハードウェアとしてRaspberry
Piを、ソフトウェアとしてGo言語で制作したプログラムを実行し、代表的な適応アルゴリズムを使用した適応フィルタによるドローンの駆動音低減の評価を行うことを目的とする。
  
\chapter{理論}\label{theory}

\section{適応フィルタの基礎}\label{basis-adf}

\subsection{ウィナーフィルタ}\label{wiener}

\subsubsection{ウィナーフィルタの構造}\label{wiener-structure}

ウィナーフィルタは線形MMSE法(最小平均二乗誤差法)を定常な時系列に適用した特殊型であり、後述の適応アルゴリズムの基礎となる。

ウィナーフィルタのブロック図を\figref{block_jp}に示す。

\begin{figure}[H]
\centering
\includegraphics[width=10cm]{figures/block_jp.png}
\caption{ウィナーフィルタのブロック図}
\label{block_jp}
\end{figure}


ウィナーフィルタの入出力は次式で表される。

\begin{equation}
y_k = \bm{w}^H \bm{x}_k
\label{simple_y}
\end{equation}

ここで、\( \bm{w}^H\) は次式で与えられるフィルタ係数ベクトルである。

\begin{equation}
\bm{w}^H = [w_1^*, \cdots, w_{K}^*]^T
\end{equation}

一方、フィルタ入力は次式のような時系列 \({x_k}\)
を逆順に並べたものとなる。

\begin{equation}
\bm{x}_k = [x_k, x_{k-1}, \cdots, x_{k-K+1}]^T
\end{equation}

これは、\equref{simple_y}により、次式のような時間領域の畳み込み演算を表すためである。

\begin{equation}
y_k = \sum_{i=1}^K w_i^* x_{k-i+1}
\end{equation}

これにより、式はフィルタ係数\({w_i^*}\)を持つ時間領域のFIRフィルタを表すことになる。入力時系列\({x_k}\)については、平均値
\(E[x_k]=0\) の実数あるいは複素数を仮定している。

ウィナーフィルタで推定すべき値は、実数あるいは複素数のスカラー量\(d_k\)となる。\(d_k\)は望みの応答と呼ばれ、フィルタ出力\(y_k\)により、望みの応答を推定する。すなわち、\(\hat{d}_k = y_k\)となる。フィルタの推定過程では、次式で定義される推定誤差を、次式で述べる規範のもとに最小化する。

\begin{equation}
\epsilon_k := d_k - y_k
\end{equation}

上述のように、適応フィルタの目的は、入力信号\(\bm{x}_k\)から\(d_k\)を推定することである。
これを達成するための過程として、フィルタ\(\bm{w}\)の係数を決定する学習過程と、\equref{simple_y}により\(\hat{d}_k\)を推定するフィルタリング過程とに分けられる。
フィルタの学習過程では、信号\(d_k\)を教師として与え、\(d_k\)と\(\bm{x}_k\)の関係を表す係数\(\bm{w}\)を学習する。
一方、フィルタリング過程では、信号\(d_k\)が未知の場合について、観測値\(\bm{x}_k\)と学習済みのフィルタ係数\(\bm{w}\)から、信号の推定値\(\hat{d_k} (= y_k)\)を得る。学習過程で望みの応答\(d_k\)をどうやって与えるかは、応用に依存する\cite{signal_processing_for_array}。

\subsubsection{ウィナーフィルタの導出}\label{wiener-introduction}

ウィナーフィルタでは、観測値\(\bm{x}_k\)から信号\(d_k\)を推定する。フィルタの学習過程では、次式の二乗平均誤差をコスト関数として最小化する。

\begin{equation}
\begin{split}
J &= E[|\epsilon_k|^2] \\
  &= E[ (d_k - \bm{w}^H \bm{x}_k) (d_k - \bm{w}^H \bm{x}_k)^H ] \\
  &= E[ d_k d_k^H] - \bm{w}^H E[\bm{x}_k d_k^H] - E[d_k \bm{x}_k^H ] \bm{w} 
\end{split}
\label{equ:J_epsilon}
\end{equation}

ここで次式を定義する。

\begin{equation}
\sigma_d^2 := E[d_k d_k^H], \bm{r}_{xd} := \bm{x}_k d_k^H, \bm{r}_{dx} := d_k \bm{x}_k^H, \bm{R}_x := E[\bm{x}_k \bm{x}_k^H]
\end{equation}

\(\bm{r}_{xd}\)および\(\bm{r}_{dx}\)は、入力ベクトル\(\bm{x}_k\)と望みの応答\(d_k\)との相互相関ベクトル、\(\bm{R}_u\)は、\(\bm{x}_k\)の自己相関行列である。これらを用いて\equref{equ:J_epsilon}は次式のように書き直せる。

\begin{equation}
J =\sigma_d^2 + - \bm{w}^H \bm{r}_{xd} -\bm{r}_{dx} \bm{w} + \bm{w}^H \bm{R}_x \bm{w}
\label{equ:J_sigma}
\end{equation}

コスト関数を\(\bm{w}^*\)について偏微分すると次式のようになる。

\begin{equation}
\frac{\partial J}{\partial \bm{w}^*} = - \bm{r}_{xd} + \bm{R}_x \bm{w}
\bm{w}
\label{equ:J_partial}
\end{equation}

\equref{equ:J_partial}を\(\bm{0}_{K \times 1}\)とおくと、次式を得る。

\begin{equation}
\bm{R}_x \bm{w} = \bm{r}_xd
\label{equ:Rwr}
\end{equation}

\equref{equ:Rwr}は、正規方程式あるいはウィナー・ホッフ方程式と呼ばれる。\equref{equ:Rwr}を解くことにより最適フィルタは、次式のように求まる\cite{signal_processing_for_array}。

\begin{equation}
\hat{\bm{w}}_{WF} = \bm{R}_u^{-1} \bm{r}_{xd}
\end{equation}

\
\subsection{適応フィルタにおける最小二乗法(LS法)}\label{ls}

適応フィルタにおけるLS法は、ウィナーフィルタの近似解を有限のサンプルから求めるものであり、\ref{rls}節で述べるRLS法の基礎となっている。

適応フィルタにおけるLS法では、次式に示す誤差の二乗和が最小化される。

\begin{equation}
J = \sum_{k=1}^{L_s} |\epsilon_k|^2 = \sum_{k=1}^{L_s} |d_k - \bm{w}^H \bm{x}_k|^2
\end{equation}

ここで、\(L_s\)はサンプル数である。ウィナーフィルタの導出過程における期待値をサンプル平均に置き換え、同様に正規方程式を求めると次式が導かれる。

\begin{equation}
\hat{\bm{R}}_x \bm{w} = \hat{\bm{r}}_xd
\end{equation}

これより、LS法における最適解は、次式のようになる。

\begin{equation}
\hat{\bm{w}}_{LS} = \hat{\bm{R}}_x^{-1} \hat{\bm{r}}_xd
\end{equation}

\subsection{最急降下法}\label{sd}

\ref{wiener}節で述べたウィナーフィルタでは、二乗平均誤差を最小とするフィルタ\(\bm{w}\)を、正規方程式の解として求めた。ここでは、この解を反復法を用いて逐次的に求める。

反復法では、フィルタ係数の初期値を適当に定め、コスト関数\(J(\bm{w})\)の最小点を目指して、フィルタ係数を少しずつ変化させていく。\equref{equ:J_sigma}で示したコスト関数の場合、\(J(\bm{w})\)は\(w\)についての2次関数となり、下に凸の誤差特性曲面となる。最急降下法では、\(k-1\)回目の反復におけるフィルタ係数を\(\bm{w}_{k-1}\)とした場合、\(\bm{w} = \bm{w}_{k-1}\)での誤差特性曲面の勾配を推定し、この勾配と逆の方向にフィルタ係数を変化させる。\(\bm{w} = \bm{w}_{k-1}\)における勾配ベクトルは\equref{equ:J_partial}から、次式のように求まる。

\begin{equation}
  \nabla=\nabla
\end{equation}

\begin{equation}
\nabla J(\bm{w}_{k-1}) := \frac{\partial J(\bm{w})}{\partial \bm{w}^*} |_{\bm{w} = \bm{w}_{k-1}} = - \bm{r}_{xd} + \bm{R}_x \bm{w}_{k-1}
\end{equation}

次回の反復で勾配とは逆方向に係数を変化させるため、フィルタ係数ベクトルの変化分は次式のようになる。

\begin{equation}
\Delta \bm{w} = - \mu \nabla J(\bm{w}_{k-1})
\end{equation}

\(\mu\)は1回の更新量を決定する性の定数であり、ステップサイズパラメータと呼ばれる。これを用いて、フィルタの更新式は、次式のようになる。

\begin{equation}
\begin{split}
\bm{w}_k &= \bm{w}_{k-1} + \Delta \bm{w} \\
         &= \bm{w}_{k-1} + \mu (\bm{r}_{xd} - \bm{R}_x \bm{w}_{k-1})
\end{split}
\label{equ:w_delta}
\end{equation}

また、\equref{equ:w_delta}は次式のように書き直すことができる。

\begin{equation}
\begin{split}
\bm{w}_k &= \bm{w}_{k-1} + \mu (E[\bm{x}_k d_k^*] - E[\bm{x}_k \bm{x}_k^H] \bm{w}_{k-1}) \\
         &= \bm{w}_{k-1} + \mu E[\bm{x}_k (d_k^* - \bm{x}_k^H \bm{w}_{k-1})]
\end{split}
\label{equ:w_complex}
\end{equation}

ここで、次式の事前推定誤差を定義する。

\begin{equation}
e_k := d_k - \bm{w}_{k-1}^H \bm{x}_k
\label{equ:prior_estimation_error}
\end{equation}

\equref{equ:w_complex}に\equref{equ:prior_estimation_error}を代入して、次式を得る。

\begin{equation}
\bm{w}_k = \bm{w}_{k-1} + \mu E[\bm{x}_k e_k^*]
\label{equ:w_update_sd}
\end{equation}


\section{代表的な適応アルゴリズム}\label{main-algo}

\subsection{最小二乗法平均法(LMS法)}\label{lms}

リアルタイムシステムを構築する場合などは、メモリや演算量に制約があることがある。最急降下法の更新式はシンプルだが、期待値演算をサンプル平均で実装することになり、メモリや演算量を必要とする。この点を改良したのが最小二乗平均法(LMS法)である。LMS法では、\equref{equ:w_update_sd}における期待値演算を用いた勾配の推定を、次式のように、瞬時値の推定値に置き換える。

\begin{equation}
\bm{w}_k = \bm{w}_{k-1} + \mu \bm{x}_k e_k^*
\label{equ:w_update_lms_simple}
\end{equation}

また、\equref{equ:w_update_lms_simple}を入力のパワーで正規化したものは、学習同定法(NLMS法)と呼ばれる。NLMS法は、次式の拘束月最適化問題空導くことができる。

\begin{equation}
min ||\bm{w}_k - \bm{w}_{k-1}||^2 \\
subject \ to \ \bm{w}_k^H \bm{x}_k = d_k
\end{equation}

この最適化問題は、ラグランジュの未定乗数法を用いて解くことができる。ラグランジュの未定乗数法では、次式のコスト関数を最小化する。

\begin{equation}
\begin{split}
J &= ||\bm{w}_k - \bm{w}_{k-1}||^2  + Re(\lambda (d_k - \bm{w}_k^H \bm{x}_k)) \\
  &= (\bm{w}_k - \bm{w}_{k-1})^H (\bm{w}_k - \bm{w}_{k-1}) + \lambda (d_k - \bm{w}_k^H \bm{x}_k) + (d_k - \bm{w}_k^H \bm{x}_k))^H \lambda^H
\end{split}
\end{equation}

コスト関数\(J\)を\(\bm{w}_k^*\)について偏微分すると

\begin{equation}
\frac{\partial J}{\partial \bm{w}_k^*} = \bm{w}_k - \bm{w}_{k-1} - \lambda \bm{x}_k
\end{equation}

これを\(\bm{0}_{K \times 1}\)とすることにより、次式を得る。

\begin{equation}
\bm{w}_k - \bm{w}_{k-1} = \lambda \bm{x}_k
\label{equ:w-w_k}
\end{equation}

\equref{equ:w-w_k}の左から\(\bm{x}_k^H\)を乗じ、\(\lambda\)について解くと

\begin{equation}
\lambda = \frac{1}{\bm{x}_k^H \bm{x}_k} \bm{x}_k^H (\bm{w}_k - \bm{w}_{k-1})
\end{equation}

これに拘束条件(\equref{equ:w_complex})および事前推定誤差(\equref{equ:prior_estimation_error})

\begin{equation}
\lambda = \frac{1}{ ||\bm{x}_k||^2 } (d_k^* - \bm{x}_k^H \bm{w}_{k-1}) = \frac{1}{||\bm{x}_k||^2} e_k^*
\label{equ:lambda}
\end{equation}

\equref{equ:lambda}を再び\equref{equ:w-w_k}に代入することにより、フィルタの変化量\(\Delta \bm{w}\)は次式のように求まる。

\begin{equation}
\Delta \bm{w} = \frac{1}{||\bm{x}_k||^2} \bm{x}_k e_k^*
\end{equation}

以上より、NLMS方のフィルタベクトル更新式は次式のようになる。

\begin{equation}
\bm{w}_k = \bm{w}_{k-1} + \lambda \frac{1}{||\bm{x}_k||^2} \bm{x}_k e_k^*
\end{equation}

実際の運用では、入力信号のパワー\(||\bm{x}_k||^2\)が非常に小さいときに更新式が不安定になるのを防ぐため、\(
1 / ||{}\bm{x}_k||^2 \)
の分母に小さい正の定数 \(\alpha\) を加えた、次式を用いる。

\begin{equation}
\bm{w}_k = \bm{w}_{k-1} + \lambda \frac{1}{\alpha + ||\bm{x}_k||^2} \bm{x}_k e_k^*
\end{equation}

\subsection{アフィン射影法(APA法)}\label{apa}

アフィン射影法は、NLMS法には、NLMS法における拘束付き最適化問題の拘束条件を次式のように複数に拡張することにより導くことができる。

\begin{equation}
min ||\bm{w}_k - \bm{w}_{k-1}||^2 \\ 
subject \ to \ \bm{w}_k^H \bm{X}_k = d_k
\end{equation}


ここで、\(\bm{X_k}\)(\(K \times L_s\)の行列)および\(\bm{d_k}\)(\(1 \times L_s\)の行ベクトル)は次式で定義される。


\begin{equation}
\bm{d}_k := [d_{k-L_s+1}, \cdots, d_k]
\end{equation}

\begin{equation}
\bm{X}_k := [\bm{x}_{k-L_s+1}, \cdots, \bm{x}_k]
\end{equation}

\begin{equation}
\bm{x}_k := [\bm{x}_{k}, \cdots, \bm{x}_{k-L_s+1}]^T
\end{equation}

拘束条件は、次式の\(L_s\)個の拘束条件を行列・ベクトル形式で表したものである。

NLMS法と同様にラグランジュの未定乗数法を用いた、最適解を導出により、次式のフィルタ更新式が得られる。

\begin{equation}
\bm{w}_k = \bm{w}_{k-1} + \mu \bm{X}_k ( \alpha \bm{I} + \bm{X}_k^H \bm{X} )^{-1} e_k^H
\end{equation}



\subsection{再帰最小二乗法(RLS法)}\label{rls}

再帰最小二乗法(RLS法)は。LS法の解を再帰的に求める手法である。

サンプル平均による自己相関行列および相互相関ベクトルの推定値\(\hat{\bm{R}}_x\)および\(\hat{\bm{r}}_{xd}\)を次式に示す。

\begin{equation}
\hat{\bm{R}}_k := \sum_{i=1}^k \bm{x}_i \bm{x}_i^H = \bm{X}_{1:k} \bm{X}_{1:k}^H
\end{equation}

\begin{equation}
\hat{\bm{r}}_k := \sum_{i=1}^k \bm{x}_i d_i^* = \bm{X}_{1:k} \bm{d}_{1:k}^H
\end{equation}

ここで、\(\bm{d}_{1:k}\)および\(\bm{X}_{1:k}\)

\begin{equation}
\bm{d}_{1:k} := [d_1, \cdots, d_k]
\end{equation}

\begin{equation}
\bm{X}_{1:k} := [\bm{x}_1, \cdots, \bm{x}_k]
\end{equation}

\begin{equation}
\bm{x}_{k} := [x_k, \cdots, u_{k-K+1}]^T
\end{equation}

時刻kにおける\(\hat{\bm{R}}_x\)および\(\hat{\bm{r}}_x\)は、一時刻前\((k-1)\)の値を用いて、次式のように再帰的に表すことができる。


\begin{equation}
\hat{\bm{R}}_k = \hat{\bm{R}}_{k-1} + \bm{x}_k \bm{x}_k^H
\end{equation}


\begin{equation}
\hat{\bm{r}}_k = \hat{\bm{r}}_{k-1} + \bm{x}_k d_k^*
\label{equ:hat_r}
\end{equation}

逆行列の補助定理を用いると、自己相関行列の逆行列\(P_k := \hat{\bm{R}}_k^{-1}\)も、次式のように再帰的に求めることができる。


\begin{equation}
\bm{P}_k = \bm{P}_{k-1} - \frac{\bm{P}_{k-1} \bm{x}_k \bm{x}_k^H \bm{P}_{k-1}}{1 + \bm{x}_k^H \bm{P}_{k-1} \bm{x}_k} 
\label{equ:P_k_inv_R}
\end{equation}

\equref{equ:hat_r}、\equref{equ:P_k_inv_}
を用いて、求めるべきフィルタ系ルウは、次式のようになる。

\begin{equation}
\bm{w}_k = \bm{P}_k \hat{\bm{r}}_k
\label{equ:w_simple_rls}
\end{equation}


ここで、更新式における演算を簡略化するため、次式のゲインベクトルを定義する。

\begin{equation}
\bm{g}_k := \frac{\bm{P}_{k-1} \bm{x}_k}{1 + \bm{x}_k^H \bm{P}_{k-1} \bm{x}_k}
\label{equ:g_k_complex}
\end{equation}

これを用いて\equref{equ:P_k_inv_R}を書き直すと


\begin{equation}
\bm{P}_k = \bm{P}_{k-1} - \bm{g}_k \bm{x}_k^H \bm{P}_{k-1} = (\bm{I} - \bm{g}_k \bm{x}_k^H) \bm{P}_{k-1}
\label{equ:P_k_flat}
\end{equation}

一方、\equref{equ:g_k_complex}から、次式を得る。

\begin{equation}
\bm{g}_k = (\bm{I} - \bm{g}_k \bm{x}_k^H) \bm{P}_{k-1} \bm{x}_k
\label{equ:g_k_flat}
\end{equation}


\equref{equ:P_k_flat}、\equref{equ:g_k_flat}を用いて、ゲインベクトルは次式のように書き直せる。


\begin{equation}
\bm{g}_k = \bm{P}_k \bm{x}_k
\label{equ:g_k_simple}
\end{equation}


\equref{equ:P_k_flat}および\equref{equ:g_k_simple}を\equref{equ:w_k_simple}に代入することにより、フィルタ係数ベクトル\(\bm{w}_k\)の更新式は、次式のように求まる。

\begin{equation}
\begin{split}
\bm{w}_k &= \bm{P}_k (\hat{\bm{r}}_{k-1} + \bm{x}_k d_k^*) \\
         &= \bm{w}_{k-1} + \bm{g}_k (d_k^* - \bm{x}_k^H \bm{w}_{k-1})
\end{split}
\label{equ:w_k_complex}
\end{equation}


最後に、\equref{equ:w_k_complex}に\equref{equ:g_k_flat}を代入して、次式の更新式を得る。

\begin{equation}
\bm{w}_k = \bm{w}_{k-1} + \bm{g}_k e_k^*
\end{equation}


\subsubsection{適応アルゴリズムの比較}\label{algo-compare}

\tabref{tab:formula}は代表的な適応アルゴリズムの関係をニュートン法を使用して整理したものである。

\begin{table}[H]
  \centering
  \caption{代表的な適応アルゴリズムの係数更新式}
  \label{tab:formula}
  \begin{tabular}{|c|c|}
  \hline
       & \(\Delta \bm{w}\)                                                                                             \\ \hline
  LMS  & \( \mu \bm{x}_k \bm{\epsilon}_k \)                                                                         \\ \hline
  NLMS & \( \mu \left( \alpha \bm{I} + \bm{x}_k \bm{x}_k^H \right)^{-1} \bm{x}_k \bm{\epsilon}_k \)        \\ \hline
  AP   & \(\mu \bm{X}_k \left( \alpha \bm{I} + \bm{X}_k^H \bm{X}_k \right)^{-1} \bm{\epsilon}_k\)          \\ \hline
  RLS  & \( \mu \bm{X}_k \left( \alpha \bm{I} + \bm{X}_{1:k}^H \bm{X}_{1:k} \right)^{-1} \bm{\epsilon}_k\) \\ \hline
  \end{tabular}
\end{table}

\tabref{tab:formula}のアルゴリズムは下に行くほど予測に用いるサンプル数が増加する。これに伴い、一般に収束速度も速くなる。一方、この代償として計算量が増大する。したがって、実装するハードウェアの規模や、求められる収束速度などを考慮して、アルゴリズムを選択する必要がある。

\
\subsection{適応フィルタを使用した雑音低減の手法}\label{adf-noise-reduction}

\
\subsubsection{アクティブノイズコントロール(ANC)}\label{anc}

アクティブノイズコントロール(ANC)とは、マイクやスピーカ等の機器を利用し、対象とする騒音と逆位相の音を重ね合わせることで音を低減する手法である。

ANCのブロック図を\figref{fig:anc_block}に示す。

\begin{figure}[H]
\centering
\includegraphics[width=10cm]{figures/anc.png}
\caption{ドローンの駆動音に対するANCの想定ブロック図}
\label{fig:anc_block}
\end{figure}

ドローンに駆動音に対するANCは\figref{fig:anc_block}における参照信号\(\bm{x}_k\)と誤差信号\(\epsilon_k\)からADF(\(\bm{w}\))で未知システムを推定することで次サンプルの信号を予測し、スピーカーなどの音響機器からフィルタ出力\(y_k\)を逆位相で発生させて打ち消す動作を繰り返す。

したがって、適応アルゴリズムの予測性能が駆動音抑圧の効果を決定づける要因となる。

\subsubsection{自動等化器}\label{automatic-equalizer}

自動等化器はアクティブノイズコントロールとは異なり、計算機内部で信号の処理を行う手法である。
一般に、適応フィルタに対する入力信号は長期間で平均をとると、ドローンの駆動音に対して音声のエネルギーが小さい。自動等化器では、フィルタの更新係数を小さくすることで係数がドローンの駆動音のみに適応することを利用し、フィルタ出力\(y_k\)との誤差\(e_k\)を目的信号の予測値として取り出す手法である。

自動等化器のブロック図を\figref{equ:automatic_equlizer_block}に示す。

\begin{figure}[H]
\centering
\includegraphics[width=13cm]{figures/automatic_equalizer_block.png}
\caption{ドローンの駆動音に対する自動等化器の想定ブロック図}
\label{equ:automatic_equlizer_block}
\end{figure}
  
\
\section{ハードウェアの製作}\label{ux30cfux30fcux30c9ux30a6ux30a7ux30a2ux306eux88fdux4f5c}}

この章では本研究で使用するドローン、マイク、Raspberry
Piといったハードウェアの構成と製作について述べる。

\
\section{システムの外観}\label{ux30b7ux30b9ux30c6ux30e0ux306eux5916ux89b3}}

まず、ドローンの本体のシステムについて述べる。

\
\section{ドローンについて}\label{ux30c9ux30edux30fcux30f3ux306bux3064ux3044ux3066}}

ドローンはリンクスモーション株式会社の組み立て式ドローン (Hquad500)
を使用した。このドローンは機体にカーボンファイバーを採用しており、軽量、高強度、高剛性を兼ね備えている。また、拡張性が高く、
容易に機能の追加が可能で様々な研究用途に適した製品である。

\
\subsection{使用機器}\label{ux4f7fux7528ux6a5fux5668}}

使用機器を以下に示す。

\begin{enumerate}
\def\labelenumi{\arabic{enumi}.}
\item
  ドローンの機体 HQuad500 Hardware kit Lynxmotion株式会社
  \href{http://www.lynxmotion.com/p-1058-hquad500-hardware-only-kit.aspx}{HQuad500}
  \includegraphics{figures/hquad500_hardware.jpg}
  \includegraphics{figures/hquad500_parts.jpg}
\item
  ESC (Electronic Speed Controller) 12A ESC (SimonK) Lynxmotion株式会社
  \href{http://www.lynxmotion.com/p-915-12a-esc-simonk.aspx}{ESC}
  \includegraphics{figures/esc.jpg}
\item
  ブラシレスモーター Brushless Motor 28x30 1000kv Lynxmotion株式会社
  \href{http://www.lynxmotion.com/p-913-brushless-motor-28x30-1000kv.aspx}{ブラシレスモーター}{]}
  \includegraphics{figures/brushless_motor.jpg}
\item
  フライトコントローラー Quadrino Nano Lynxmotion株式会社
  \href{http://www.lynxmotion.com/p-1020-lynxmotion-quadrino-nano-flight-controller-with-gps.aspx}{フライトコントローラー}
  \includegraphics{figures/quadrino_nano.jpg}
\item
  リポバッテリー充電器 18W LiPo Battery Charger Lynxmotion株式会社
  \href{http://www.lynxmotion.com/p-985-18w-lipo-battery-charger.aspx}{リポバッテリー充電器}
  \includegraphics{figures/lipo_charger.jpg}
\item
  リポバッテリー 11.1V (3S), 3500mAh 30C LiPo Battery Pack
  Lynxmotion株式会社
  \href{http://www.lynxmotion.com/p-985-18w-lipo-battery-charger.aspx}{リポバッテリー充電器}
  \includegraphics{figures/lipo_charger.jpg}
\item
  ラジオレシーバー R9DS 10 channels 2.4GHz DSSS FHSS Receiver
  RadioLink株式会社
  \href{http://www.radiolink.com.cn/doce/product-detail-120.html}{ラジオレシーバー}
  \includegraphics{figures/r9ds.jpg}
\item
  トランスミッタ AT9S 2.4GHz 10CH transmitter RadioLink株式会社
  \href{http://www.radiolink.com.cn/doce/product-detail-119.html}{トランスミッタ}
  \includegraphics{figures/at9s.jpg}
\end{enumerate}

\
\subsection{組み立ておよび動作確認}\label{ux7d44ux307fux7acbux3066ux304aux3088ux3073ux52d5ux4f5cux78baux8a8d}}

ドローンの組み立ては@に従い、次のように行った。

\begin{enumerate}
\def\labelenumi{\arabic{enumi}.}
\item
  内容物の確認

  \begin{itemize}
  \tightlist
  \item
    表でリスト
  \item
    写真 \includegraphics{drone_parts.png}
  \end{itemize}
\item
  機体の組み立て・源および信号線の配線
  \includegraphics{figures/drone_block.pdf}
  \includegraphics{figures/drone_block.png}
\item
  動作確認
  \href{http://www.lynxmotion.com/images/document/PDF/LynxmotionUAV-QuadrinoNano-UserGuideV1.1.pdf}{ドローンの取扱説明書}
  に従い、動作確認を行った。
\end{enumerate}

\
\section{バイノーラルマイクの製作}\label{ux30d0ux30a4ux30ceux30fcux30e9ux30ebux30deux30a4ux30afux306eux88fdux4f5c}}

\
\subsection{マイクについて}\label{ux30deux30a4ux30afux306bux3064ux3044ux3066}}

収音には、音像定位実験のために製作したバイノーラルマイクを流用した。

使用した素子は秋月電子のエレクトレットコンデンサーマイクロホン(ECM)\href{http://akizukidenshi.com/catalog/g/gP-08181/}{エレクトットコンデンサマイク}である。製作は@のようにLRそれぞれエレクトットコンデンサマイクをはんだ付けし、ロボットケーブルを経由してステレオミニプラグと接続した。

\begin{itemize}
\tightlist
\item
  使用機器

  \begin{enumerate}
  \def\labelenumi{\arabic{enumi}.}
  \item
    エレクトットコンデンサマイク XCM6035 株式会社秋月電子通商
    \href{http://akizukidenshi.com/catalog/g/gP-08181/}{url} x2
    \includegraphics{figures/microphone.jpg}
    \includegraphics{figures/microphone_size.jpg}
  \item
    シールドスリムロボットケーブル KRT-SW 株式会社秋月電子通商
    \href{http://akizukidenshi.com/catalog/g/gP-07457/}{url}
    \includegraphics{figures/sielded_robot_cable.jpg}
    \includegraphics{figures/sielded_robot_cable_size.jpg}
  \item
    3.5mmΦステレオミニプラグ MP-319 株式会社秋月電子通商
    \includegraphics{figures/mini_plug.jpg}
    \includegraphics{figures/mini_plug_size.jpg}
  \end{enumerate}
\end{itemize}

\begin{figure}
\centering
\includegraphics{@TODO}
\caption{バイノーラルマイクの製作}
\end{figure}

\
\section{Raspberry
Piについて}\label{raspberry-piux306bux3064ux3044ux3066}}

Raspberry
Piは英国のラズベリーパイ財団によって開発されている、ARMプロセッサを搭載したシングルボードコンピュータである。Raspberr
Piは教育用として制作されたが、現在ではIoT製品開発などの業務や人工衛星のOBC
(On Board Computer) にも使用されている。

Raspberry Piには

\begin{itemize}
\tightlist
\item
  Linux系のOSで動作するためソフトウェア開発に強みをもち、GPIOピンを通してSPI、I2C、I2Sなどの通信を行えるため、センサなどを用いた開発を容易に行える。また、USB端子を搭載し、Wi-Fi、Bluetooth接続も可能で
  プロタイプ開発に適したデバイスとなっている。
\end{itemize}

\
\subsection{OSの選定}\label{osux306eux9078ux5b9a}}

Raspberry Piで使用可能なOSには

\begin{itemize}
\tightlist
\item
  電子工作などに適した公式OS Raspbian
\item
  LinuxディストリビューションのUbuntuから派生した Ubuntu MATE
\item
  Microsoft Windows 10
\end{itemize}

などが存在する。

本研究では主にGPIOを使用して開発を行うため、Raspbianを使用した。なお、OSのバージョンは10.1
Buster Liteである。また、カーネルのバージョンは4.19.75-v7である。

\
\subsection{初期設定について}\label{ux521dux671fux8a2dux5b9aux306bux3064ux3044ux3066}}

\begin{enumerate}
\def\labelenumi{\arabic{enumi}.}
\item
  OSのインストール
\item
  地域、言語の設定 \texttt{sudo\ raspi-config}\\
  Localization Options
\item
  sshの設定 \texttt{sudo\ raspi-config}\\
  Interfacing Options\\
  SSH
\item
  プロキシに関する設定
\item
  アップデート

\begin{verbatim}
sudo apt update 
sudo apt upgrade -y 
sudo apt dist-upgrade
sudo rpi-update
sudo reboot
\end{verbatim}
\end{enumerate}

\
\subsection{AD変換用の拡張ボードについて}\label{adux5909ux63dbux7528ux306eux62e1ux5f35ux30dcux30fcux30c9ux306bux3064ux3044ux3066}}

Raspberry
PiはADC(ADコンバータ)を搭載していないため、マイクからの入力信号を扱うにはADコンバータを導入する必要がある。

本研究で用いたのはマルツエレック株式会社の\href{http://select.marutsu.co.jp/list/detail.php?id=258}{Pumpkin
Pi}である。Pumpkin
Piは計測用とオーディオ用のデュアルA-Dコンバータを搭載しており、Raspberry
Piにオーディオ入力、アナログ入力機能を加えることが可能となる。

Pumpkin Piの仕様を以下に示す。

\begin{itemize}
\tightlist
\item
  対応OS Raspbian
\item
  対応機種 Raspberry Pi Model B+/Raspberry Pi 2 Model B/Raspberry Pi 3
  Model B
\item
  LED出力 1点
\item
  赤外線リモコン機能 送受信
\item
  オーディオコネクタ φ3.5mmステレオミニジャック
\item
  オーディオ入力 量子化ビット数=24,サンプリング周波数=48/96KHz
\item
  計測用AD変換 2チャンネル,16ビット
\item
  本体寸法 65(W)×56(D)mm
\item
  本体重量 約25g
\end{itemize}

\begin{figure}
\centering
\includegraphics{figures/pumpkin_pi.jpg}
\caption{PumpkinPi}
\end{figure}

\href{http://select.marutsu.co.jp/list/detail.php?id=258}{PumpkinPi}
\href{https://www.marutsu.co.jp/pc/i/833515/}{PumpkinPi}

\
\subsection{セットアップ}\label{ux30bbux30c3ux30c8ux30a2ux30c3ux30d7}}

Pumpkin
Piのセットアップは\href{https://toragi.cqpub.co.jp/tabid/829/Default.aspx}{トランジスタ技術
2017年1月号 オールDIPで1日製作! 音声認識ハイレゾPiレコーダ「Pumpkin
Pi」} にしたがって行った。以下に簡易的な手順を示す。

\begin{enumerate}
\def\labelenumi{\arabic{enumi}.}
\item
  Pumpkin Piを使用するためのRaspberry Pi固有の設定

  まず適当な作業ディレクトリで以下のコマンドを実行する。

\begin{verbatim}
wget http://einstlab.web.fc2.com/RaspberryPi/PumpkinPi.tar
tar xvf PumpkinPi.tar
cd PumpkinPi
./setup.sh  # @参考文献では ./PumpkinPi.sh と表記されている
\end{verbatim}
\item
  カーネルとデバイス・ドライバのバージョンの確認

  カーネルのバージョンとデバイス・ドライバのバージョンは同じである必要がある。カーネルのバージョンは\texttt{uname\ -r}で、デバイス・ドライバのバージョンは\texttt{modinfo\ snd\_soc\_pcm1808\_adc.ko}でそれぞれ確認できる。
\item
  ADコンバータ用のデバイス・ドライバのインストール

  次の2つのデバイス・ドライバをインストールする。

  \begin{enumerate}
  \def\labelenumii{\arabic{enumii}.}
  \tightlist
  \item
    pcm1808-adc.ko\\
    PCM1808固有の動作を決定するドライバ。
  \item
    snd\_soc\_pcm1808\_adc.ko\\
    Raspberry Piのサウンドとして属性を決定するドライバ
  \end{enumerate}

  まず、ホームディレクトリにPumpkinPi.tarをダウンロードして展開する。

\begin{verbatim}
cd 
wget http://einstlab.web.fc2.com/RaspberryPi/PumpkinPi.tar
tar xvf PumpkinPi.tar
cd PumpkinPi/Driver
\end{verbatim}

  次にデバイス・ドライバをインストールする。

\begin{verbatim}
sudo cp Backup/pcm1808-adc.bak/ /lib/modules/`uname -r`/kernel/sound/soc/codecs/pcm1808-adc.ko
sudo cp Backup/snd_soc_pcm1808_adc.bak /lib/modules/`uname -r`/kernel/sound/soc/bcm/snd_soc_pcm1808_adc.ko
sudo depmod -a  # 依存関係を調整
\end{verbatim}

  OSのカーネル4.4以降ではデバイス・ツリー構造を導入してあるため、デバイス・ツリー情報ファイルをコピーする。

\begin{verbatim}
sudo cp pcm1808-adc.dtbo /boot/oberlays/
\end{verbatim}

  最後にデバイス・ドライバが電源起動時に自動的に読み込まれるように\texttt{/boot/config.txt}につぎの1行を追加する。

\begin{verbatim}
dtoverlay=pcm1808-adc
\end{verbatim}

  以上の作業を完了した後、再起動することで設定が適用される。
\end{enumerate}
  
\chapter{ソフトウェアの製作}\label{software}

\
\section{使用するソフトウェア・プログラムについて}\label{about-program}

本研究ではシミュレーションとデータ処理のためのプログラムをPythonとGo言語を使用して製作した. 

\
\subsection{Pythonについて}\label{about-python}

Pythonは, 汎用のプログラミング言語である. コードがシンプルで扱いやすく設計されており, C言語などに比べて, さまざまなプログラムを分かりやすく, 少ないコード行数で書けるといった優れた特徴がある. 

標準ライブラリやサードパーティ製のライブラリなど, さまざまな領域に特化した豊富で大規模なツール群が用意され, 自らの使用目的に応じて機能を拡張していくことができる. 

% このような特徴から, 様々な試行をしながら行う開発に適している. 本研究ではこの特性を活かし, 予備実験・データ処理・グラフ作成にPythonを使用した. 

\
\subsection{Go言語について}\label{about-go}

Googleで開発された, 汎用のプログラミング言語である. Go言語は, 静的型付け, C言語の伝統に則ったコンパイル言語, メモリ安全性, ガベージコレクション, 並行性などの特徴を持つ. また, 軽量スレッディングのための機能, Pythonのような動的型付け言語のようなプログラミングの容易性, などの特徴もある. 

また, Go言語はPythonのようにシンプルな記法を有し, 制作者によらず同じコードになりやすく学習しやすいため, 研究用途に適した言語である. 

\
\subsection{プログラミング言語の使い分けについて}\label{about-proper-use}

Pythonは\ref{about-python}節で述べた特徴から, 様々な試行をしながら行う開発に適している. 一方で, 実行速度が遅く, リアルタイム処理には向かない. したがって, 本研究ではこの特性を活かし, 予備実験・データ処理・グラフ作成にPythonを使用した. 

また, Go言語は言語仕様が現代的で, 実行速度が速く, クロスコンパイルが可能といった特徴から, ADFライブラリの作成および実装に使用した. 


% \section{Pythonを使用した適応フィルタの予備実験}\label{python-experiment}

\newpage

\
\section{音声ファイル処理およびグラフ作成プログラムの制作}\label{create-program}

本節では後述の実験結果の処理に使用したプログラムの制作について述べる. 

音声ファイルの処理には行列演算が必要である. この処理には数値計算を効率的に行うための拡張モジュールであるNumPyを使用した. また, 波形のプロットにはNumPyを基盤にしたグラフ描画ライブラリmatplotlibを使用した. 
音声ファイルの入出力には標準ライブラリであるwaveや外部ライブラリのPySoundFileを使用した. 


\subsection{Pythonで制作したプログラム}\label{python}

Pythonで制作したプログラムを\tabref{tab:program_python}に示す。

% Please add the following required packages to your document preamble:
% \usepackage{multirow}
\begin{table}[H]
  \centering
  \caption{Pythonで制作したプログラム}
  \label{tab:program_python}
  \begin{tabular}{|c|l|c|}
  \hline
  種類                     & \multicolumn{1}{c|}{用途}           & 付録番号       \\ \hline
  \multirow{3}{*}{モジュール} & 波形プロット簡易化                         & 1.1.1-1  \\ \cline{2-3} 
                         & 音声ファイル入出力                         & 1.1.1-2  \\ \cline{2-3} 
                         & 関数の実行時間計測・情報表示                    & 1.1.1-3  \\ \hline
  \multirow{10}{*}{ツール}  & 音声ファイル畳み込み                        & 1.1.2-4  \\ \cline{2-3} 
                         & ステレオ音声ファイルをLRモノラル音声に分割            & 1.1.2-5  \\ \cline{2-3} 
                         & 2つのwavファイル間の伝達関数を計算               & 1.1.2-6  \\ \cline{2-3} 
                         & csvファイルの各列のデータをそれぞれwavファイルに変換     & 1.1.2-7  \\ \cline{2-3} 
                         & \begin{tabular}{l}目的の音声ファイルからノイズの音声ファイル\\に含まれる成分を取り除く\end{tabular} & 1.1.2-8  \\ \cline{2-3} 
                         & 指定秒数分の白色雑音をサンプリング周波数48kHzで生成      & 1.1.2-9  \\ \cline{2-3} 
                         & csvファイルからmp4またはgif画像を生成           & 1.1.2-10 \\ \cline{2-3} 
                         & csvファイルから波形をプロット                  & 1.1.2-11 \\ \cline{2-3} 
                         & 複数のwavファイルから一枚の図に波形をプロット          & 1.1.2-12 \\ \cline{2-3} 
                         & スペクトログラムを表示                       & 1.1.2-13 \\ \hline
  \end{tabular}
  \end{table}

% \begin{enumerate}
% \renewcommand{\labelenumi}{(\arabic{enumi})}
% \item
%   モジュール

%   \begin{enumerate}
%   \renewcommand{\labelenumi}{(\arabic{enumi})}
%   \item
%     波形プロット簡易化のモジュール\ \ref{plot_tools.py}

%   \item
%     音声ファイル入出力モジュール\  \ref{wave_handler.py}

%   \item
%     関数の実行時間計測・情報表示用デコレータ\ \ref{decorators.py}

%   \end{enumerate}

% \item
%   ツール

%   \begin{enumerate}
%   \renewcommand{\labelenumi}{(\arabic{enumi})}
%   \item
%     音声ファイル畳み込み用プログラム\ \ref{calc_covolution_wav.py}

%   \item
%     ステレオ音声ファイルをLRモノラル音声に分割するプログラム\ \ref{calc_stereo2mono_LR.py}

%   \item
%     2つのwavファイル間の伝達関数を求めるプログラム\ \ref{calc_pseudo_ir_between_wav_files.py}

%   \item
%     csvファイルの各列のデータをそれぞれwavファイルに変換するプログラム\ \ref{csv_to_wav_each_column.py}

%   \item
%     目的の音声ファイルからノイズの音声ファイルに含まれる成分を取り除くプログラム(スペクトラムサブトラクション法)\ \ref{calc_subtracted_wav.py}

%   \item
%     指定秒数分の白色雑音をサンプリング周波数48kHzで生成するプログラム\ \ref{generate_white_noise_as_wav.py}

%   \item
%     csvファイルからmp4またはgif画像を生成するプログラム\ \ref{plot_animation_from_csv.py}

%   \item
%     csvファイルから波形をプロットするプログラム\ \ref{plot_from_csv.py}

%   \item
%     複数のwavファイルから一枚の図に波形をプロットするプログラム\ \ref{plot_multiwave.py}

%   \item
% スペクトログラムを表示するプログラム\ \ref{plot_spectrogram_librosa.py}

  % \end{enumerate}
% \end{enumerate}

\subsection{Go言語で制作したプログラム}\label{go}

Go言語で制作したプログラムを\tabref{tab:program_go}に示す。

\begin{table}[H]
  \centering
  \caption{Go言語で制作したプログラム}
  \label{tab:program_go}
  \begin{tabular}{|c|l|c|}
  \hline
  種類                     & \multicolumn{1}{c|}{用途}                                  & 付録番号       \\ \hline
  \multirow{3}{*}{ライブラリ} & 型変換処理                                                    & 1.2.1-2  \\ \cline{2-3} 
                         & 計算処理                                                     & 1.2.1-3  \\ \cline{2-3} 
                         & ファイル入出力処理                                                & 1.2.1-4  \\ \hline
  \multirow{9}{*}{ツール}   & 指定したSN比で音声を合成                                            & 1.2.1-5  \\ \cline{2-3} 
                         & 2つのwavファイルを畳み込み                                          & 1.2.1-6  \\ \cline{2-3} 
  & \begin{tabular}{l}2つのwavファイルを畳み込み\\(フーリエ変換を用いた高速版)\end{tabular}                          & 1.2.1-7  \\ \cline{2-3} 
                         & \begin{tabular}{l}2つのwavファイルもしくはcsvファイルを畳み込み\\(フーリエ変換を用いた高速版)\end{tabular}                 & 1.2.1-8  \\ \cline{2-3} 
                         & csvファイルのデータからwavファイルを生成する                                & 1.2.1-9  \\ \cline{2-3} 
                         & \begin{tabular}{l}第\ref{adf-practice}章の実験で用いるADFの設定を記述した\\JSONファイルを生成\end{tabular}           & 1.2.1-10 \\ \cline{2-3} 
                         & \begin{tabular}{l}csvファイルのデータから指定したサンプルの\\平均二乗誤差(MSE)を計算し, \\ csvファイルに新たな列として追記\end{tabular}  & 1.2.1-11 \\ \cline{2-3} 
                         & PortAudioを使用した多チャンネル収音用                                  & 1.2.1-12 \\ \cline{2-3} 
                         & DSBファイルからwavファイルに変換                                      & 1.2.1-13 \\ \hline
  \end{tabular}
  \end{table}

% \begin{enumerate}
% \renewcommand{\labelenumi}{(\arabic{enumi})}
% \tightlist

% \item
%   ライブラリ \\
%   \begin{enumerate}
%   \renewcommand{\labelenumi}{(\arabic{enumi})}
%   \tightlist
%   \item
%     型変換処理\ \ref{converter.go}

%   \item
%     計算処理\  \ref{calculator.go}

%   \item
%     ファイル入出力処理\ \ref{filehandler.go}
%   \end{enumerate}

% \item
%   プログラム

%   \begin{enumerate}
%   \renewcommand{\labelenumi}{(\arabic{enumi})}

%   \item
%     指定した音圧差で音声を合成するプログラム\ \ref{calc_noise_mix}

%   \item
%     2つのwavファイルを畳み込むプログラム\ \ref{convolve_wav}

%   \item
%     2つのwavファイルを畳み込むプログラム(フーリエ変換を用いた高速版)\ \ref{convolve_wav_fast}

%   \item
%     2つのwavファイルもしくはcsvファイルを畳み込むプログラム(フーリエ変換を用いた高速版)\ \ref{convolve_wav_coef}

%   \item
%     csvファイルのデータからwavファイルを生成するプログラム\ \ref{csv_to_wav}

%   \item
%     \ref{adf-practice}の実験で用いるADFの設定を記述したJSONファイルを生成するプログラム\ \ref{drone_json_generator}

%   \item
%     csvファイルのデータから指定したサンプルの平均二乗誤差(MSE)を計算し, csvファイルに新たな列として追記するプログラム\ \ref{calc_mse_csv}

%   \item
%     PortAudioを使用した多チャンネル収音用プログラム\ \ref{multirecord/main}

%   \item
%     DSBファイルからwavファイルに変換するプログラム\ \ref{wav_to_DSB/main}
%   \end{enumerate}
% \end{enumerate}

\newpage

\section{ADFライブラリの制作}\label{create-adflib}

\
\subsection{ライブラリの設計}\label{design-adf}

Pythonで記述された適応信号処理のライブラリとしてPadasip\cite{padasip}が公開されている. Padasipは信号のフィルタリング, 予測, 復元, 分類といった適応信号処理を簡易化するために設計されたライブラリであり, LMS・NLMS, AP, RLSをはじめとする主要なアルゴリズムが一通り実装されている. 

本研究で制作したADFライブラリはPadasipを参考に設計し, Go言語で実装した. 

ライブラリのコードを付録\ref{go-adflib_code}に示す. 

ライブラリの各関数にはユニットテストが書かれており, GitHub
Actionsを利用して自動テストが行われるため, 一定の質が担保されている. 

また, ライセンスはMITを採用しているため, 使用, 再配布, 商用利用などが許可されている. %詳しくは\cite{https://raw.githubusercontent.com/tetsuzawa/go-adflib/master/LICENSE}{MIT LICENCE}を参照されたい. 

\
\subsection{インストール方法}\label{how-to-install}

製作したライブラリはGitHub\cite{go-adflib:online}で公開したため, 次のコマンドを実行することでインストールすることができる. 

\texttt{go\ get\ github.com/tetsuzawa/go-adflib/adf}

\
\subsection{使用方法}\label{how-to-use}

% 以下に基本的なライブラリの使用方法を示す. 

% \begin{itemize}
% \tightlist
% \item
%   \texttt{NewFilt***} \\ 
%   各アルゴリズムはGo言語の構造体として実装されている. このコードにより, 構造体のインスタンスを生成することができる. 
% \item
%   フィルタ更新方法 \\
%   各アルゴリズムのインスタンスには次のメソッドが実装されている. 用途によって実行方法を変えることができる. 

%   \begin{itemize}
%   \tightlist
%   \item
%     \texttt{Adapt} \\
%     リアルタイムで信号の入出力を行い, フィルタリング処理をする場合を使用する. 
%   \item
%     \texttt{Run} \\
%     測定したデータなど, まとまったデータに対して後から処理する場合を使用する
%   \end{itemize}
% \item
%   \texttt{ExploreLearning} \\
%   各アルゴリズムにはフィルタの更新速度を調整するステップサイズパラメータが存在する. このパラメータは信号によって最適な値が異なるため, 効率よく処理を行うためには最適値を探索する必要がある. \texttt{ExploreLearning}を使用することで, 指定した信号に対する最適なステップサイズを探すことができる. 
% \item
%   \texttt{GetParams} \\
%   フィルタの係数やステップサイズパラメータなどの情報を取得したい場合に使用する. 
% \end{itemize}

ライブラリの使用方法はPadasipと同様である。まず、任意のフィルタのインスタンスを生成し、メソッドを実行することでフィルタリングを行える。

詳しい使い方はGoDoc(Goのパッケージリファレンス)\cite{godoc:online}を参照されたい. 
  
\
\chapter{駆動音低減のための予備実験}\label{noise-pre-experiment}

\
\section{ドローンの駆動音のサンプル収音}\label{record-drone}

\subsection{実験目的}\label{purpose-drone}

本研究で用いるLMS・AP・RLSアルゴリズムは参照信号が有色性を持つと収束性能が低下する特徴を持つ. したがって, リアルタイム処理が目的の場合, 参照信号の周波数特性が処理効果に影響を及ぼすことになる. したがって, ドローンの駆動音の周波数特性を調べるため, サンプル収音を行った. 

\
\subsection{実験方法}\label{instruction-drone}

\
\subsubsection{収音のためのハードウェア構成について}\label{hardware-for-record}

収音はドローンとバイノーラルマイクを使用し, \figref{fig:pre-ex_block}のブロック図の構成で行った. 

ただし, 前述のバイノーラルマイクのみでは十分な入力電圧が得られないため, 収音にはマイクロフォンアンプを使用する必要がある. 

なお, 本研究では室内でサンプル収音を行ったため, 据え置き型のマイクロフォンアンプ(オーディオテクニカ, AT-MA2)を使用したが, 実際にドローンに搭載する際は別途小型のマイクロフォンアンプが必要と推察される. 
\begin{figure}[H]
\centering
\includegraphics[width=10cm]{figures/pre-ex_block.pdf}
\caption{ドローンの駆動音サンプル収音のブロック図}
\label{fig:pre-ex_block}
\end{figure}

\
\subsubsection{収音に使用するソフトウェアについて}\label{software-for-record}

音声の収録にはGo言語で記述した自作の収音用コマンドラインツール(\ref{code:multirecord})を使用した. 
このプログラムはSoX(コマンドラインベースの音声編集ソフトウェア)\cite{sox:online}の\texttt{rec}コマンドを参考に設計した. 
オーディオ入出力APIのPortAudio\cite{portaudio:online}を利用しており, 多チャンネルでの収音に対応している他, 独自の機能として音声データを本研究で主に使用されている形式(.DSB - 符号付き整数型16bitバイナリファイル)で保存することが可能となっている. 

本実験では自作のソフトウェアを使用したが, .wavなどの一般的な音声データ形式を使用する場合, 前述のSoXを利用しても同様に収音が可能である. また, 同じくPortAudioをベースとしたGUIの音声編集ソフトウェアであるAudacity\cite{audacity:online}も有力な候補となる. いずれのソフトウェアも無料で使用可能である. 

\subsection{使用機器}\label{used-equipments-drone}

使用機器を以下に示す. 

\begin{enumerate}
\renewcommand{\labelenumi}{(\arabic{enumi})}
\item
  ドローン\\
  \ref{about-drone}節を参照. 
\item
  マイク \\
  \ref{binaural-mic}節のバイノーラルマイクを1chのみ使用した. 詳細は\ref{binaural-mic}節を参照. 
\item
  マイクロフォンアンプ AT-MA2 株式会社オーディオテクニカ
  % \begin{figure}[H]
  % \centering
  % \includegraphics[width=13cm]{figures/at-ma2.jpg}
  % \caption{マイクロフォンアンプ}
  % \label{fig:at-ma2}
  % \end{figure}

\item
  オーディオインターフェース ローランド株式会社 DUO-CAPTURE EX S/N.Z6C6056
  \cite{audio_interface:online}
  % \begin{figure}[H]
  % \centering
  % \includegraphics[width=13cm]{figures/duo-capture.jpg}
  % \caption{オーディオインターフェース}
  % \label{fig:duo-capture}
  % \end{figure}
\end{enumerate}

\
\subsection{実験結果}\label{result-drone}

収録したドローンの駆動音の波形を\ref{plot_multiwave.py}のプログラムを使用し, 描画した図を\figref{fig:drone_raw}に示す. また, \ref{plot_spectrogram_librosa}のプログラムを使用し, 周波数特性を解析した. 駆動音のスペクトログラムを\figref{fig:drone_raw_spectrogram}に示す. 

\begin{figure}[H]
\centering
\includegraphics[width=13cm]{figures/drone_raw.png}
\caption{ドローンの駆動音の波形観測結果}
\label{fig:drone_raw}
\end{figure}

\begin{figure}[H]
\centering
\includegraphics[width=13cm]{figures/drone_raw_spectrogram.png}
\caption{ドローンの駆動音のスペクトログラム観測結果}
\label{fig:drone_raw_spectrogram}
\end{figure}

\figref{fig:drone_raw}, \figref{fig:drone_raw_spectrogram}における0~10[sec]および70[sec]以降の区間はドローンを停止させた無音区間となっている. しかしながら, 無音区間に何らかの定常な雑音が存在するためドローンの駆動音のみを解析するためには, この定常雑音を取り除く必要がある. 

そこで, \ref{calc_subtracted_wav}のプログラムを使用し, スペクトラルサブトラクション法\cite{spectral_subtraction}により, 無音区間の周波数成分の平均を全体から取り除いた. 
スペクトラルサブトラクション法適用後の波形を\figref{fig:drone_subtracted}に, スペクトログラムを\figref{fig:drone_subtracted_spectrogram}に示す. 

\begin{figure}[H]
\centering
\includegraphics[width=13cm]{figures/drone_subtracted.png}
\caption{ドローンの駆動音の波形観測結果(雑音処理後)}
\label{fig:drone_subtracted}
\end{figure}

\begin{figure}[H]
\centering
\includegraphics[width=13cm]{figures/drone_subtracted_spectrogram.png}
\caption{ドローンの駆動音のスペクトログラム観測結果(雑音処理後)}
\label{fig:drone_subtracted_spectrogram}
\end{figure}


\
\subsection{考察}\label{consideration-drone}

\figref{fig:drone_subtracted_spectrogram}より, 駆動音は100Hz付近の低域成分とその倍音成分, 5kHz付近の中域成分, 18kHz付近の高域成分を合わせた周波数特性を示すことがわかる. このうち, 低域と中域の成分はドローンの羽音およびモータの回転音によるもの, また, 高域成分はブラシレスモータの回転数を制御するインバータ回路によるものと思われる. 

上記より, ドローンの駆動音を有色性の周波数特性を持つことが確認できた. 

\
\section{信号の有色性に対するADFの性能への影響}\label{adf-color-effect}

\
\subsection{実験目的}\label{purpose-color}

本節ではADFに対する参照信号を\ref{record-drone}節で収音したドローンの駆動音とした場合と白色雑音とした場合の収束に要するサンプル数(収束速度), 収束した後の精度(推定精度)を比較する. 
これにより, 各適応アルゴリズムの信号の有色性に対する収束性能への影響を評価することを目的とする. 

\
\subsection{実験方法}\label{instruction-color}

実験方法を以下に示す. 

\begin{enumerate}
\renewcommand{\labelenumi}{(\arabic{enumi})}
\item
  \ref{record-drone}節で収音した雑音処理完了後の音声(以下, 駆動音とする)に対して, ADFの各アルゴリズムにおける最適なステップサイズを求める. 
\item
  (a)で求めたADFに対して, 駆動音を参照信号, 駆動音に白色雑音を加えたものを目的信号として十分な時間フィルタリングを実行
\item
  同様に, 白色雑音を参照信号, 白色雑音にの別の白色雑音を加えたものを目的信号として十分な時間フィルタリングを実行
\item
  フィルタ誤差をMSEで平滑化したものをグラフ化し, 各アルゴリズムで比較・評価
\end{enumerate}

ただし, 本実験におけるADFのフィルタ長は4サンプル, 付与雑音は参照信号に対して-30dBの振幅, MSEのタップ数は512サンプルとした. 

\
\subsection{実験結果}\label{result-color}

信号に駆動音を使用した場合の各アルゴリズムの収束特性を\figref{fig:drone_algo_convergence}に示す. 同様に, 白色雑音を使用した場合の各アルゴリズムの収束特性を\figref{fig:white_algo_convergence}に示す. 

\begin{figure}[H]
\centering
\includegraphics[width=13cm]{figures/algo_dr_L-4_mse.png}
\caption{信号に駆動音を使用した場合の各アルゴリズムの収束特性測定結果}
\label{fig:drone_algo_convergence}
\end{figure}

\begin{figure}[H]
\centering
\includegraphics[width=13cm]{figures/algo_white_L-4_mse.png}
\caption{信号に白色雑音を使用した場合の各アルゴリズムの収束特性測定結果}
\label{fig:white_algo_convergence}
\end{figure}

\
\subsection{考察}\label{consideration-color}

\figref{fig:drone_algo_convergence}, \figref{fig:white_algo_convergence}より, 信号に駆動音を使用した場合, 理論通り収束性能が低下していることがわかる. 特にRLSアルゴリズムの場合, 収束時のMSEが約80dBから約50dBとなり, 収束するまでに必要なサンプル数も約300,000サンプル程度まで増加していることがわかる. サンプリング周波数は48kHzであるので収束までに約6秒かかることになる. 一方, NLMSアルゴリズムとAPアルゴリズムは収束時のMSEは10dBほど悪化しているが, RLSアルゴリズムほど大きい性能低下は見られなかった. 

以上より, リアルタイム処理を目的とする場合NLMS・APアルゴリズムの使用が適していると思われる. 

\
\section{ADFライブラリのベンチマーク}\label{benchmark}

\
\subsection{実験目的}\label{purpose-benchmark}

ADFの性能を評価する尺度の1つとして, 収束速度, 推定精度の他に演算時間(演算量)が存在する. 
特にリアルタイムでフィルタ処理を行う場合, プログラムの実行速度は雑音低減の性能に大きな影響を及ぼす. 

本節では, Go言語で自作したADFライブラリのベンチマークを行い, 各アルゴリズムの実行に要する演算時間を評価することを目的とする. 

\
\subsection{実験方法}\label{instruction-benchmark}

本実験ではGo言語に標準で実装されているベンチマークの仕組みを利用して行った. 

Go言語ではテストファイルに\texttt{func\ BenchmarkXxx(b\ *testing.B)}のように関数を宣言することで容易にベンチマークを実装することができる. また, \texttt{for\ i\ :=\ 0;\ i\ \textless{}\ b.N;\ i++}のようにfor文を記述することで, 自動的に1秒あたりの実行回数を測定・出力することが可能である. なお, ベンチマークはテストファイルと同じディレクトリで\texttt{go\ test\ -bench\ .\ -benchmem}を実行することで行える. 

% ベンチマークのプログラムを@に示す.  @TODO

ベンチマークはそれぞれ信号として白色雑音を使用し, フィルタ長は8タップとした. 

実行環境はRaspberry Pi 3 Model Bである. また, 比較対象としてMacBook Pro (13-inch, 2019)上においてもベンチマークを実行した. 

\
\subsection{実験結果}\label{result-benchmark}

1サンプルあたりのフィルタ係数更新・次サンプルの予測にかかる演算時間とメモリ使用量の測定結果を\tabref{tab:benchmark_raspi}, \tabref{tab:benchmark_mac}に示す. 

\begin{table}[H]
  \centering
  \caption{ADFライブラリのベンチマーク測定結果(Raspberry Pi)}
  \label{tab:benchmark_raspi}
  \begin{tabular}{|c|c|c|}
  \hline
  アルゴリズム & 実行時間{[}ms/op{]} & メモリ使用量{[}B/op{]} \\ \hline
  NLMS   & 1.9             & 84               \\ \hline
  AP     & 78.4            & 941              \\ \hline
  RLS    & 52.9            & 2549             \\ \hline
  \end{tabular}
\end{table}

\begin{table}[H]
  \centering
  \caption{ADFライブラリのベンチマーク測定結果(MacBook Pro)}
  \label{tab:benchmark_mac}
  \begin{tabular}{|c|c|c|}
  \hline
  アルゴリズム & 実行時間{[}ms/op{]} & メモリ使用量{[}B/op{]} \\ \hline
  NLMS   & 0.13            & 96               \\ \hline
  AP     & 5.40            & 1248             \\ \hline
  RLS    & 3.25            & 2920             \\ \hline
  \end{tabular}
\end{table}
\
\subsection{考察}\label{consideration-color}

\
\subsubsection{メモリ使用量について}\label{about-memory}

\tabref{tab:benchmark_raspi}, \tabref{tab:benchmark_mac}より, 各アルゴリズムのメモリ使用量はNLMSアルゴリズムが一番少なく, APアルゴリズム, RLSアルゴリズムと多くなっている. これは理論式に即した物となっており, 妥当な値である. 

Go言語にはガーベッジコレクションが実装されており, メモリセーフである. また, Raspberry
Pi 3 Model
Bのメモリは1GBと潤沢であるため, 使用メモリに関しては実用範囲内であると思われる. 

\
\subsubsection{演算時間について}\label{about-time}

\tabref{tab:benchmark_raspi}, \tabref{tab:benchmark_mac}より, NLMSアルゴリズムの演算時間はAP・RLSに比べて数十倍高速であることがわかる. 一方, Raspberry Piの演算時間はMacBook Proに比べるて十倍程度長いことがわかる. 

ここで, これらのアルゴリズムをアクティブノイズコントロールに適用することを考える. 
音速が340m/s、参照信号用のマイクと目的信号用のマイクの距離が30cmであることを仮定すると, 音波が2つのマイク間を伝播するために要する時間は
\begin{equation}
  \frac{0.3}{340} \approx 0.88 [\si{\milli \second}]
\end{equation}
である. したがって, ADFのフィルタ実行時間以外の遅延を無視した理想条件で最低でも0.88ms/op以上の演算速度が必要なことになる. 
また, これはフィルタ長が約43タップに相当する距離である. 実験で用いたフィルタ長は8タップであるため, フィルタ長が不十分で, システムの同定が不可能であることを示す. 
したがって, 本実験の条件でのアクティブノイズコントロールの実装は, 制作したライブラリを使用する場合には現実的でないことが結論づけられる. 
なお, 実現に向けての改善策としては, プログラムの実行速度を上げる, 計算機の性能を高くする, 許容範囲内でサンプリング周波数を低くするなどが挙げられる. 

% 一方, 自動等化器の実装には空間的制約が無いため, 本研究のシステムを想定すると人間が音の遅延を知覚できると言われている20msまで許容できる. 

また, APアルゴリズムの演算時間がRLSアルゴリズムに比べて長いことが表よりわかる. 理論上の演算量はRLSアルゴリズムが上回っているため, アルゴリズムの実装を見直す必要があると考えられる. これについては今後の課題としたい. 
  
\
\chapter{適応フィルタを用いた駆動音低減法の検討}\label{adf-practice}

\
\section{実験目的}\label{purpose-practice}

本実験では, 実際に入力される信号を模擬し, ADFによる駆動音低減の効果を検討する. 信号の模擬のためには音波の伝達関数を求め、駆動音・音声信号と畳み込み、複数のSN比で合成する。これらの信号に対して、複数のフィルタ長でADFを実行することでSN比・フィルタ長を変化させたときのフィルタリング性能の比較が可能となる。
これにより, 予備実験と合わせて雑音低減の実現可能性を総合的に評価することを目的とする. 

\
\section{実験方法}\label{instruction-practice}

本実験は以下の手順で行った. 

\
\subsubsection{ドローンのマイクの取り付け}\label{installment-mic}

\figref{fig:drone_experiment}のように簡易的にマイクを木材と粘着テープで固定し, 取り付けた. 

\begin{figure}[H]
\centering
\includegraphics[width=12cm]{figures/drone_experiment_picture_caption.png}
\caption{ドローンと各マイクの取り付け}
\label{fig:drone_experiment}
\end{figure}

\
\subsubsection{音声信号の音源から2つのマイクまでの伝達関数の測定}\label{observation-tf}

目的音源からマイクまでの伝達関数を測定した. これにより, 任意の音声信号を畳み込むことで, 様々な条件での雑音低減のシミュレーションを行うことができる. 

伝達関数の測定は設備の都合上室内で行った. 実際にドローンを使用するのは屋外を想定しているため, 残響を無効化するために1000サンプル以上の伝達関数を切り捨て, 半自由音場を模擬した. 

\begin{figure}[H]
\centering
\includegraphics[width=10cm]{figures/voice_TFMeasure.pdf}
\caption{音声信号の音源から2つのマイクまでの伝達関数の測定}
\end{figure}

\
\subsubsection{ドローンの駆動音に対する2つのマイクの擬似的な伝達関数の計算}\label{pseudo-tf}

2つのマイク間の疑似伝達関数を計算するためにはそれぞれのマイクで駆動音を収音する必要がある. 収音は\ref{record-drone}節の構成を2chに拡張して行った. 

疑似伝達関数は2つの駆動音をフーリエ変換し, 除算したものを逆フーリエ変換することで得られる. この処理は\ref{calc\_pseudo\_ir\_between\_wav\_files.py}を実行して行った. 

\begin{figure}[H]
\centering
\includegraphics[width=10cm]{figures/drone_TFCalc.pdf}
\caption{ドローンの駆動音に対する2つのマイクの擬似的な伝達関数の計算}
\end{figure}

\
\subsubsection{駆動音と音声信号それぞれで参照信号と目的信号用の伝達関数を畳み込み}\label{convolve-each}

\ref{observation-tf}, \ref{pseudo-tf}で各伝達関数を求めたので, 駆動音・音声信号それぞれ\ref{calc\_convolve\_fast}のプログラムを実行して伝達関数を畳込み, 参照信号と目的信号を生成した. 

\
\subsubsection{指定したSN比で駆動音と音声信号を合成}\label{mix-snr}

\ref{calc\_noise\_mix}を実行し, -40dBから0dBまで5dB刻みで駆動音と音声信号を合成した. 

\
\subsubsection{合成した参照信号と目的信号でADFを実行}\label{exec-adf}

\ref{adf-color-effect}節と同様に合成した参照信号と目的信号に対して, ADFの各アルゴリズムにおける最適なステップサイズを求め, 十分な時間フィルタリングを実行する. 

\
\section{実験結果}\label{result-practice}

SN比を-40dB, -20dB, 0dB, フィルタ長を4, 64, 256とした場合の各アルゴリズムの収束特性を\figref{fig:onepage}に示す. 
\begin{figure}[H]
\centering
\includegraphics[width=13cm]{figures/onepage_legend_label.png}
\label{fig:onepage}
\caption{種々のSN比とフィルタにおける各アルゴリズムの収束特性}
\end{figure}

\
\section{考察}\label{consideration-practice}

\figref{fig:onepage}より, 音声が波形として見られるのはAPアルゴリズムのSN比0dB, フィルタ長4タップのものとSN比0dBのRLSアルゴリズムのみであった. 実際の運用ではSN比が0dBよりも悪化することが予想されるため, 適切に駆動音を低減し, 音声を抽出することは難しいと思われる. 

\figref{fig:onepage}の波形に音声帯域以外の高周波が存在すると仮定し, 収束特性に対してカットオフ8kHz, フィルタ長512タップのFIRローパスフィルタを適用した. これにより得られた収束特性を\figref{fig:onepage_8kHz}に示す. 

\begin{figure}[H]
\centering
\includegraphics[width=13cm]{figures/onepage_8kHz_label.png}
\label{fig:onepage_8kHz}
\caption{種々のSN比とフィルタ長における各アルゴリズムの収束特性(8kHz以下)}
\end{figure}

\figref{fig:onepage_8kHz}から分かる通り, 周波数帯域を8kHz以下に限定しても収束特性がほとんど変化しないことがわかる. したがって, 波形に現れたのは8kHz以下の雑音であると推測される. 

NLMSフィルタが有効な結果を得られなかった理由について考察する. 本実験で使用したマイクは無指向性である. また, マイクの配置は参照信号と目的信号の収音を模擬した簡易的なものであった. ここで, \ref{mix-snr}で合成した参照信号と目的信号を比較する. 参照信号と目的信号を\figref{fig:compare_x_d}に示す. 

\begin{figure}[H]
\centering
\includegraphics[width=13cm]{figures/compare.png}
\caption{作成した参照信号と目的信号の比較}
\label{fig:compare_x_d}
\end{figure}

\figref{fig:compare_x_d}より, 参照信号と目的信号に大きな差が無いことがわかる. 一般にADFが十分なフィルタリング性能を発揮するためには, 参照信号に目的信号が入り込まないことが条件となる. 前述の通り, 本実験では簡易的なシステムを使用したためこのような結果が得られたと推察される. したがって, 十分な指向性を有したマイクを使用し, 適切な配置で収音することが今後の課題となる. 

\figref{fig:onepage}より雑音低減の効果が最も優れた結果となったのはRLSアルゴリズムで有ることがわかる. しかしながら, \ref{benchmark}節の予備実験で確認したように, リアルタイム処理の観点から見るとRLSは実装に適していないため, 本研究で検討した構成での雑音低減の実現可能性は低いと思われる. 
  
\
\section{結論}\label{ux7d50ux8ad6}}

本研究では、ドローンにし、
本研究では、ドローンの応用分野の1つとして音声収録に着目し、搭載されたマイクと小型の計算機を使用して収録した音声信号からドローンの駆動音を取り除く手法について検討した。

駆動音の低減法を検討するにあたり、ドローン・バイノーラルマイクの組み立て、Raspberry
Piのセットアップなどハードウェアの準備を行った。 Raspberry
PiにはAD変換が搭載されていないため、拡張ボードとしてPumpkin
Piを使用し、初期設定を行った。

ソフトウェアに関しては、まずPythonを使用して、静的なFIRフィルタ・適応アルゴリズムの評価を行った。次にADFのライブラリをGo言語で自作し、公開した。また、Python・Go言語を使用して波形表示や音声編集用のソフトウェアを制作した。

次にドローンの駆動音に対する各適応アルゴリズムの有効性を検証するために、駆動音のサンプル収音を行い、ADFの収束特性を試験した。結果的にNLMS・APアルゴリズムに比べてRLSアルゴリズムの収束誤差は小さいが、収束速度が遅くリアルタイム処理には向かないことが判明した。

制作したADFライブラリのベンチマークを取ると、Raspberry
Piを計算媒体とした場合、一番高速なNLMSアルゴリズムでも計算速度が遅く、アクティブノイズコントロールの実装は難しいことが判明した。

最後に、実際の使用の際に入力される信号を模擬し、ADFによる雑音低減の効果を検討した。音声の信号が得られたのはSN比0dBのRLSアルゴリズムと一部のフィルタ長のAPアルゴリズムのみであった。

以上より本研究で検討した構成での雑音低減の実現可能性は低いと思われる。
  

\begin{bib}[100]
    % BibTeXを使う場合
    \bibliography{main}
\end{bib}

  % 参考文献。要独自コマンド、include先参照のこと
\begin{acknowledgment}\label{acknowledgment}


本研究を進めるにあたり、御指導を頂いた苫小牧工業高等専門学校電気電子工学科准教授工藤彰洋博士に心より感謝します. 

また本研究は、文部科学省・平成29
年度宇宙航空科学技術推進委託費・宇宙航空人材育成プログラム、「超小型衛星開発を通した高専ネットワーク型宇宙人材育成」(研究代表者
徳山工業高等専門学校 北村健太郎)の支援を受けて実施しました. 

\end{acknowledgment}  % 謝辞。要独自コマンド、include先参照のこと
\appendix

\chapter{付録}

\section{Pythonで制作したプログラム}

\subsection{モジュール}

\begin{lstlisting}[caption=plot\_tools.py,label=plot_tools.py]
    # coding: utf-8

    import matplotlib.pyplot as plt
    import numpy as np
    from matplotlib.colors import Normalize
    
    # diagram display settings
    plt.rcParams['font.family'] = 'IPAPGothic'
    plt.rcParams['font.size'] = 16
    plt.rcParams['xtick.direction'] = 'in'
    plt.rcParams['ytick.direction'] = 'in'
    plt.rcParams['xtick.top'] = True
    plt.rcParams['ytick.right'] = True
    plt.rcParams['xtick.major.width'] = 1.0
    plt.rcParams['ytick.major.width'] = 1.0
    plt.rcParams['axes.linewidth'] = 1.0
    plt.rcParams['figure.figsize'] = (8, 7)
    plt.rcParams['figure.dpi'] = 100
    plt.rcParams['figure.subplot.hspace'] = 0.3
    plt.rcParams['figure.subplot.wspace'] = 0.3
    
    
    class PlotTools(object):
        """ PlotTools - plotting utility class for research
        """
    
        def __init__(self, y, fs=44100, fft_N=1024, stft_N=256, **kwargs):
    
            if "start_pos" in kwargs:
                self.start_sec = kwargs["start_sec"]
            else:
                self.start_sec = 0
    
            #
            self.y = np.array(y)  # data
            self.fs = fs  # Sampling frequency
            self.dt = 1 / fs  # Sampling interval
            # Start position to analyse
            self.start_pos = int(self.start_sec / self.dt)
    
            self.fft_N = fft_N  # FFT length
            self.stft_N = stft_N  # STFT length
            self.freq_list = np.fft.fftfreq(fft_N, d=self.dt)  # FFT frequency list
    
            if "window" in kwargs:
                window_name = kwargs["window"]
            else:
                window_name = "hamming"
    
            self.fft_window = define_window_function(name=window_name, N=self.fft_N)
            self.stft_window = define_window_function(name=window_name, N=self.stft_N)
    
            self.Y = np.fft.fft(self.fft_window * self.y)
            self.samples = np.arange(self.start_pos, fft_N + self.start_pos)
            self.t = np.arange(self.start_sec, (fft_N + self.start_pos) * self.dt, self.dt)
    
            y_abs = np.abs(y)
            # Complement 0 or less to mean to prevent from divergence
            self.y_abs = completion_0(y_abs)  # Absolute value of y
            self.y_gain = 20.0 * np.log10(y_abs)  # Gain of y
    
            # Spectrum
            self.amp_spectrum = np.abs(self.Y) / fft_N * 2
            self.amp_spectrum[0] = self.amp_spectrum[0] / 2
            self.gain_spectrum = 20 * np.log10(self.amp_spectrum / np.max(self.amp_spectrum))
            self.phase_spectrum = np.rad2deg(np.angle(self.Y))
            self.power_spectrum = self.amp_spectrum ** 2
            self.power_gain_spectrum = 10 * np.log10(self.power_spectrum / np.max(self.power_spectrum))
    
            self.acf = np.real(np.fft.ifft(self.power_spectrum / np.amax(self.power_spectrum)))
            self.acf = self.acf / np.amax(self.acf)
    
        def plot_y_time(self):
            """plot_y_time - y vs time [sec]
            """
            fig = plt.figure()
            ax1 = fig.add_subplot(111)
            ax1.plot(self.t, self.y, "-", markersize=1)
            ax1.axis([self.start_sec, (self.fft_N + self.start_pos) *
                      self.dt, np.amin(self.y) * (-1.1), np.amax(self.y) * 1.1])
            ax1.set_xlabel("Time [sec]")
            ax1.set_ylabel("Amplitude")
            ax1.set_title("Amplitude - Time")
            plt.show()
    
        def plot_y_sample(self):
            """plot_y_sample - y vs sample number
            """
            fig = plt.figure()
            ax1 = fig.add_subplot(111)
            ax1.plot(self.samples, self.y, "-", markersize=1)
            ax1.axis([self.start_pos, self.fft_N + self.start_pos,
                      np.amin(self.y) * 1.1, np.amax(self.y) * 1.1])
            ax1.set_xlabel("Sample")
            ax1.set_ylabel("Amplitude")
            ax1.set_title("Amplitude - Sample")
            plt.show()
    
        def plot_freq_analysis_log(self):
            """ plot_freq_analysis_log - 1. gain vs frequency. 2. phase vs frequency
            """
            fig = plt.figure()
            ax1 = fig.add_subplot(211)
            ax1.set_xscale('log')
            ax1.axis([10, self.fs / 2, np.amin(self.gain_spectrum),
                      np.amax(self.gain_spectrum) + 10])
            ax1.plot(self.freq_list, self.gain_spectrum, '-', markersize=1)
            ax1.set_xlabel("Frequency [Hz]")
            ax1.set_ylabel("Amplitude [dB]")
            ax1.set_title("Amplitude spectrum")
    
            ax2 = fig.add_subplot(212)
            ax2.set_xscale('log')
            ax2.axis([10, self.fs / 2, -180, 180])
            ax2.set_yticks(np.linspace(-180, 180, 9))
            ax2.plot(self.freq_list, self.phase_spectrum, '-', markersize=1)
            ax2.set_xlabel("Frequency [Hz]")
            ax2.set_ylabel("Phase [deg]")
            ax2.set_title("Phase spectrum")
            plt.show()
    
        def plot_freq_analysis(self):
            """ plot_freq_analysis - 1. amplitude vs time [sec]. 2. amplitude vs frequency
            """
            fig = plt.figure()
            ax1 = fig.add_subplot(211)
            ax1.plot(self.t, self.y, "-", markersize=1)
            ax1.axis([self.start_sec, (self.fft_N + self.start_pos) *
                      self.dt, np.amin(self.y) * 1.2, np.amax(self.y) * 1.2])
            ax1.set_xlabel("Time [sec]")
            ax1.set_ylabel("Amplitude")
            ax1.set_title("Amplitude - Time")
    
            ax2 = fig.add_subplot(212)
            ax2.axis([10, self.fs / 2, np.amin(self.amp_spectrum)
                      * 0.9, np.amax(self.amp_spectrum) * 1.1])
            ax2.plot(self.freq_list, self.amp_spectrum, '-', markersize=1)
            ax2.set_xlabel("Frequency [Hz]")
            ax2.set_ylabel("Amplitude")
            ax2.set_title("Amplitude - Frequency")
            plt.show()
    
        def plot_power_gain_spectrum(self):
            """ plot_power_gain_spectrum - power vs frequency
            """
            fig = plt.figure()
            ax1 = fig.add_subplot(111)
            ax1.set_xscale('log')
            ax1.axis([10, self.fs / 2, np.amin(self.power_gain_spectrum),
                      np.amax(self.power_gain_spectrum) + 10])
            ax1.plot(self.freq_list, self.gain_spectrum, '-', markersize=1)
            ax1.set_xlabel("Frequency [Hz]")
            ax1.set_ylabel("power")
            ax1.set_title("Power spectrum")
            plt.show()
    
        def plot_acf(self):
            """ plot_power_gain_spectrum - auto correlation function
            """
            fig = plt.figure()
            ax1 = fig.add_subplot(111)
            ax1.axis([-self.fft_N / 10, self.fft_N / 2, np.amin(self.acf) * 1.1,
                      np.amax(self.acf) * 1.1])
            ax1.plot(list(range(self.fft_N)), self.acf, '-', markersize=1)
            ax1.set_xlabel("sample number")
            ax1.set_ylabel("Correlation")
            ax1.set_title("Auto correlation Function")
            plt.show()
    
        def plot_spectrogram_acf(self):
            """ plot_power_gain_spectrum - 1. spectrogram. 2. short time auto correlation function
            """
            # The degree of frame overlap when the window is shifted
            overlap = int(self.stft_N / 2)
            # Length of wav
            frame_length = len(self.y)
            # Time per wav_file
            time_of_file = frame_length * self.dt
    
            # Define execute time
            start = overlap * self.dt
            stop = time_of_file
            step = (self.stft_N - overlap) * self.dt
            time_ruler = np.arange(start, stop, step)
    
            # Definition initialization in transposition state
            spec = np.zeros([len(time_ruler), 1 + int(self.stft_N / 2)])
            st_acf = np.zeros([len(time_ruler), 1 + int(self.stft_N / 2)])
            pos = 0
    
            for fft_index in range(len(time_ruler)):
                # Frame cut out
                frame = self.y[pos:pos + self.stft_N]
                # Frame cut out determination
                if len(frame) == self.stft_N:
                    # Multiply window function
                    windowed_data = self.stft_window * frame
                    # FFT for only real demention
                    fft_result = np.fft.rfft(windowed_data)
                    fft_result = np.abs(fft_result)
                    fft_result = fft_result / np.amax(fft_result)
                    power_spectrum = np.abs(fft_result) ** 2
                    acf = np.real(np.fft.ifft(
                        power_spectrum / np.amax(power_spectrum)))
                    acf = acf / np.amax(acf)
                    # Find power spectrum
                    fft_data = 10 * np.log10(np.abs(fft_result) ** 2)
                    # Assign to spec
                    for i in range(len(spec[fft_index])):
                        spec[fft_index][-i - 1] = fft_data[i]
                        st_acf[fft_index][-i - 1] = acf[i]
    
                    # Shift the window and execute the next frame.
                    pos += (self.stft_N - overlap)
    
            # ============  plot  =============
            fig = plt.figure()
            ax1 = fig.add_subplot(111)
            im = ax1.imshow(spec.T, extent=[0, time_of_file,
                                            0, self.fs / 2],
                            aspect="auto",
                            cmap="inferno",
                            interpolation="nearest",
                            norm=Normalize(vmin=-80, vmax=0))
            ax1.set_xlabel("Time[sec]")
            ax1.set_ylabel("Frequency[Hz]")
            ax1.set_title("Spectrogram")
            pp1 = fig.colorbar(im, ax=ax1, orientation="vertical")
            plt.tight_layout()
            # pp1.set_clim(-80, 0)
            pp1.set_label("power")
            plt.show()
    
            # ============  plot  =============
            fig = plt.figure()
            ax2 = fig.add_subplot(111)
            im2 = ax2.imshow(st_acf.T, extent=[0, time_of_file,
                                               0, self.fs / 2],
                             aspect="auto",
                             cmap="inferno",
                             interpolation="nearest",
                             norm=Normalize(vmin=0, vmax=1.00))
    
            ax2.set_xlabel("time[sec]")
            ax2.set_ylabel("frequency[Hz]")
            ax2.set_title("Short Time Auto correlation Function")
    
            pp2 = fig.colorbar(im2, ax=ax2, orientation="vertical")
            plt.tight_layout()
            pp2.set_label("power")
    
            plt.show()
    
        def plot_all(self):
            """ plot_all - execute all of plotting functions
            """
            self.plot_y_time()
            self.plot_y_sample()
            self.plot_freq_analysis()
            self.plot_freq_analysis_log()
            self.plot_power_gain_spectrum()
            self.plot_acf()
            self.plot_spectrogram_acf()
    
    
    def completion_0(data_array):
        """completion_0 - intermediate value completion
        """
        under_0list = np.where(data_array <= 0)
        for under_0 in under_0list:
            try:
                data_array[under_0] = 1e-8
                # 抜け値など, 平均を取りたい場合コメントアウトを外す
                # data_array[under_0] = (
                #     data_array[under_0-1] + data_array[under_0+1]) / 2
            except IndexError as identifier:
                print(identifier)
                data_array[under_0] = 1e-8
            return data_array
    
    
    def define_window_function(N, name):
        """define_window_function - defines window function
        """
        if name == "kaiser":
            # TODO kaiser_param kakikaeru
            # kaiser_param = input("Parameter of Kaiser Window : ")
            kaiser_param = 5
        else:
            kaiser_param = 5
    
        windows_dic = {
            "rectangular": np.ones(shape=N),
            "hamming": np.hamming(M=N),
            "hanning": np.hanning(M=N),
            "bartlett": np.bartlett(M=N),
            "blackman": np.blackman(M=N),
            "kaiser": np.kaiser(M=N, beta=kaiser_param),
        }
    
        if name in windows_dic:
            return windows_dic[name]
        else:
            raise WindowNameNotFoundError
    
    
    class WindowNameNotFoundError(Exception):
        """Window name not found
        """
        print("Window Not Found")
    
\end{lstlisting}

\begin{lstlisting}[caption=wave\_handler.py,label=wave_handler.py]

# coding:utf-8
import wave

import numpy as np


class WaveHandler(object):

    def __init__(self, filename=None, **kwargs):
        if filename:
            self.wave_read(filename)
        else:
            self.ch = 1
            self.width = 2
            self.fs = 48000
            self.chunk_size = 1024

    def wave_read(self, filename):
        # open wave file
        wf = wave.open(filename, 'r')

        # waveファイルが持つ性質を取得
        self.filename = filename
        self.ch = wf.getnchannels()
        self.width = wf.getsampwidth()
        self.fs = wf.getframerate()
        self.params = wf.getparams()
        chunk_size = wf.getnframes()
        # load wave data
        amp = (2 ** 8) ** self.width / 2
        data = wf.readframes(chunk_size)  # バイナリ読み込み
        # data = np.frombuffer(data, 'int16')  # intに変換
        data = np.frombuffer(data, dtype="int16")  # intに変換
        data = data / amp  # 振幅正規化
        self.chunk_size = chunk_size
        self.data = data
        # self.data = data[::self.ch]  # dateを1chに限定
        wf.close()  # 結果表示

        print("分析対象ファイル:", self.filename)
        print("チャンクサイズ:", self.chunk_size)
        print("サンプルサイズのバイト数:", self.width)
        print("チャンネル数:", self.ch)
        print("wavファイルのサンプリング周波数:", self.fs)
        print("パラメータ : ", self.params)
        print("wavファイルのデータ個数:", len(self.data))

    def wave_write(self, filename, data_array):
        ww = wave.open(filename, 'w')
        ww.setnchannels(self.ch)
        ww.setsampwidth(self.width)
        ww.setframerate(self.fs)
        amp = (2 ** 8) ** self.width / 2
        data_array = data_array / np.max(data_array)
        write_array = np.array(data_array * amp, dtype=np.int16)
        ww.writeframes(write_array)
        ww.close()
\end{lstlisting}

\begin{lstlisting}[caption=decorators.py,label=decorators.py]
import time
from functools import wraps


def stop_watch(func):
    @wraps(func)
    def wrapper(*args, **kwargs):
        print("###############  START  ###############")
        start = time.time()
        result = func(*args, **kwargs)
        elapsed_time = time.time() - start
        print("###############  END  ###############")
        elapsed_time = round(elapsed_time, 5)
        print(
            f"{elapsed_time}[sec] elapsed to execute the function:{func.__name__}"
        )
        return result

    return wrapper


def print_info(func):
    @wraps(func)
    def wrapper(*args, **kwargs):
        print("func:", func.__name__)
        print("args:", args)
        print("kwargs:", kwargs)
        result = func(*args, **kwargs)
        print("result:", result)
        return result

    return wrapper
\end{lstlisting}

\subsection{ツール}

\begin{lstlisting}[caption=calc\_convolution\_wav.py,label=calc_convolution_wav.py]
#! /usr/bin/env python
# coding: utf-8

import argparse
import pathlib

import numpy as np
import soundfile as sf


def main():
    description = "This script calculates the convolution of two wav files."
    usage = f"Usage: python {__file__} [-m full|valid|same] name1.wav name2.wav /path/to/output_name.wav"

    # initial setting for command line arguments
    parser = argparse.ArgumentParser(usage=usage, description=description)

    parser.add_argument('input_paths',
                        nargs="*",
                        type=str,
                        help='paths where the wav file is located.')

    parser.add_argument('-o', '--output_path',
                        nargs='?',
                        default="conv.wav",
                        type=str,
                        help='Output file path where you want to locate wav file. (default: current directory)',
                        )

    parser.add_argument('-m', '--mode',
                        action='store',
                        nargs='?',
                        default="full",
                        type=str,
                        choices=['full', 'valid', 'same'],
                        help='Convolution type',
                        )

    # parse command line arguments
    args = parser.parse_args()

    input_paths = args.input_paths
    output_path = pathlib.Path(args.output_path)
    mode = args.mode

    # validation of command line arguments
    if len(input_paths) != 2:
        print(f"Error: number of arguments is invalid. got:{len(input_paths)}")
        print(parser.usage)
        exit(1)

    if mode != "full" and mode != "valid" and mode != "same":
        print(f"Error: mode of convolution is invalid. got:{mode}")
        print(parser.usage)

        exit(1)

    input_path_1 = input_paths[0]
    input_path_2 = input_paths[1]
    output_path = output_path

    output_path = pathlib.Path(output_path)

    # read audio file
    data_1, fs_1 = sf.read(input_path_1)
    data_2, fs_2 = sf.read(input_path_2)

    # convolve
    print("working...")
    wav_out = np.convolve(data_1, data_2, mode=mode)

    # write audio file
    sf.write(file=str(output_path.stem + ".wav"), data=wav_out, samplerate=48000, endian="LITTLE",
             format="WAV", subtype="PCM_16")

    print(f"output files are saved at: {output_path}")


if __name__ == '__main__':
    main()
\end{lstlisting}

\begin{lstlisting}[caption=calc\_stereo2mono\_LR.py,label=calc_stereo2mono_LR.py]
# coding:utf-8
import sys

import soundfile as sf


def main():
    print("start")

    filename = sys.argv[1]
    filename_L = sys.argv[2]
    filename_R = sys.argv[3]
    print("input: ", filename)
    print("output L: ", filename_L)
    print("output R: ", filename_R)

    data, fs = sf.read(filename)
    mono_L = data[:, 0]
    mono_R = data[:, 1]

    sf.write(file=filename_L, data=mono_L, samplerate=48000, endian="LITTLE", format="WAV", subtype="PCM_16")
    sf.write(file=filename_R, data=mono_R, samplerate=48000, endian="LITTLE", format="WAV", subtype="PCM_16")

    print("done")


if __name__ == '__main__':
    main()
\end{lstlisting}

\begin{lstlisting}[caption=calc\_pseudo\_ir\_between\_wav\_files.py,label=calc_pseudo_ir_between_wav_files.py]
#! /usr/bin/env python
# coding: utf-8

import sys

import numpy as np
import soundfile as sf


def main():
    l_path = sys.argv[1]
    r_path = sys.argv[2]
    out_path = sys.argv[3]

    data_l, fs_l = sf.read(l_path)
    data_r, fs_r = sf.read(r_path)

    F_l = np.fft.fft(data_l)
    F_r = np.fft.fft(data_r)

    F_pseudo_ir = F_l / F_r
    f_pseude_ir = np.fft.ifft(F_pseudo_ir)
    f_pseude_ir_real = np.real(f_pseude_ir)

    sig = f_pseude_ir_real
    sf.write(file=out_path, data=sig, samplerate=48000, endian="LITTLE", format="WAV", subtype="PCM_16")


if __name__ == '__main__':
    main()
\end{lstlisting}

\begin{lstlisting}[caption=csv\_to\_wav\_each\_column.py,label=csv_to_wav_each_column.py]
#! /usr/bin/env python
# coding: utf-8

import pathlib
import sys

import numpy as np
import pandas as pd
import soundfile as sf


def main():
    input_path = sys.argv[1]
    output_path = sys.argv[2]

    df = pd.read_csv(input_path)
    output_path = pathlib.Path(output_path)

    arr = np.array(df)

    for i, data in enumerate(arr.T):
        sf.write(file=str(output_path.stem + f"_col{i}.wav"), data=data, samplerate=48000, endian="LITTLE",
                    format="WAV", subtype="PCM_16")

    print(f"output files are saved at: {output_path}")


if __name__ == '__main__':
    main()
\end{lstlisting}

\begin{lstlisting}[caption=calc\_subtracted\_wav.py,label=calc_subtracted_wav.py]
# coding: utf-8


import sys

import librosa
import numpy as np
import soundfile as sf


def main():
    input_file_path = sys.argv[1]
    noise_file_path = sys.argv[2]
    output_file_path = sys.argv[3]

    print('input file:', input_file_path)
    data, data_fs = librosa.load(input_file_path, sr=None, mono=True)
    data_st = librosa.stft(data)
    data_st_abs = np.abs(data_st)
    angle = np.angle(data_st)
    b = np.exp(1.0j * angle)

    print('noise file:', noise_file_path)
    noise_data, noise_fs = librosa.load(noise_file_path, sr=None, mono=True)
    noise_data_st = librosa.stft(noise_data)
    noise_data_st_abs = np.abs(noise_data_st)
    mean_noise_abs = np.mean(noise_data_st_abs, axis=1)

    subtracted_data = data_st_abs - mean_noise_abs.reshape(
        (mean_noise_abs.shape[0], 1))  # reshape for broadcast to subtract
    subtracted_data_phase = subtracted_data * b  # apply phase information
    y = librosa.istft(subtracted_data_phase)  # back to time domain signal

    # save as a wav file
    sf.write(file=str(output_file_path), data=y, samplerate=data_fs, endian="LITTLE", format="WAV", subtype="PCM_16")
    print('output file:', output_file_path)


if __name__ == '__main__':
    main()

\end{lstlisting}

\begin{lstlisting}[caption=generate\_white\_noise\_as\_wav.py,label=generate_white_noise_as_wav.py]
#!/usr/bin/env python3
# coding: utf-8

import argparse
import pathlib

import numpy as np
import soundfile as sf


def main():
    parser = argparse.ArgumentParser(description="This script makes white noise to designated path.")

    parser.add_argument('duration',
                        action='store',
                        type=float,
                        help='The length of noise.')

    parser.add_argument('-d', '--dst_path',
                        action='store',
                        nargs='?',
                        const="/tmp",
                        default=".",
                        type=str,
                        help='Directory path where you want to locate output files. (default: current directory)')

    args = parser.parse_args()
    duration = args.duration
    output_dir = pathlib.Path(args.dst_path)
    output_name = pathlib.Path.joinpath(output_dir, f"white_noise_{duration}s.wav")

    sig = np.random.rand(int(duration * 48000))

    sf.write(file=str(output_name), data=sig, samplerate=48000, endian="LITTLE", format="WAV", subtype="PCM_16")


if __name__ == '__main__':
    main()
\end{lstlisting}

\begin{lstlisting}[caption=plot\_animation\_from\_csv.py,label=plot_animation_from_csv.py]
# encoding: utf-8

import argparse
import pathlib

import matplotlib.pyplot as plt
import pandas as pd
from matplotlib import animation

plt.rcParams['font.family'] = 'IPAPGothic'
plt.rcParams['font.size'] = 11
plt.rcParams['xtick.direction'] = 'in'
plt.rcParams['ytick.direction'] = 'in'
plt.rcParams['xtick.top'] = True
plt.rcParams['ytick.right'] = True
plt.rcParams['xtick.major.width'] = 1.0
plt.rcParams['ytick.major.width'] = 1.0
plt.rcParams['axes.linewidth'] = 1.0
plt.rcParams['figure.figsize'] = (8, 7)
plt.rcParams['figure.dpi'] = 100
plt.rcParams['figure.subplot.hspace'] = 0.3
plt.rcParams['figure.subplot.wspace'] = 0.3


def main():
    description = "This script plots graph from a csv file with 3 columns."
    parser = argparse.ArgumentParser(description=description)

    parser.add_argument('csv_path',
                        action='store',
                        const=None,
                        default=None,
                        type=str,
                        help='Directory path where the csv file is located.',
                        metavar=None)

    parser.add_argument('-d', '--dst_path',
                        action='store',
                        nargs='?',
                        const="/tmp",
                        default=".",
                        type=str,
                        help='Directory path where you want to locate png files. (default: current directory)',
                        metavar=None)

    parser.add_argument('-s', '--samples',
                        action='store',
                        nargs='?',
                        # const="/tmp",
                        default=50,
                        type=int,
                        help='Samples to draw.',
                        metavar=None)

    parser.add_argument('-i', '--interval',
                        action='store',
                        nargs='?',
                        # const="/tmp",
                        default=100,
                        type=int,
                        help='Interval to draw.',
                        metavar=None)

    args = parser.parse_args()

    input_path = args.csv_path
    input_path = pathlib.Path(input_path)

    df = pd.read_csv(input_path, header=None)
    print("analize file name: ", input_path)

    d, y, e = df[0], df[1], df[2]

    output_dir = pathlib.Path(args.dst_path)
    output_name = pathlib.Path(input_path.name).with_suffix(".gif")
    # output_name = pathlib.Path(input_path.name).with_suffix(".mp4")
    output_path = pathlib.Path.joinpath(output_dir, output_name)

    samples = args.samples
    interval = args.interval

    fig, (ax1, ax2) = plt.subplots(2, 1)
    ax1.set_xlabel("iteration n")
    ax1.set_ylim((-1.6, 1.6))
    ax2.set_xlabel("iteration n")
    ax2.set_ylim((-0.5, 0.5))

    ani = animation.FuncAnimation(fig, update, fargs=(d[:samples], y[:samples], e[:samples], ax1, ax2),
                                    interval=interval, frames=int(samples / 2))
    ani.save(output_path, writer='imagemagick')
    # ani.save(output_path, writer='ffmpeg')

    print("\nfilterd data plot is saved at: ", output_path, "\n")


def update(i, d, y, e, ax1, ax2):
    if i == 0:
        ax1.legend(loc='upper right')
        ax1.set_title("Desired value, Filter output and Filter error")
    else:
        plt.cla()
    i = i * 2
    ax1.plot(d[0:i], color="b", alpha=0.7, label="desired value d(n)")
    ax1.plot(y[0:i], color="r", alpha=0.7, label="filter output y(n)")

    ax2.plot(e[0:i], color="y", alpha=1.0, label="filter error e(n)")


if __name__ == '__main__':
    main()
\end{lstlisting}

\begin{lstlisting}[caption=plot\_from\_csv.py,label=plot_from_csv.py]
# encoding: utf-8

import argparse
import pathlib

import matplotlib
import numpy as np
import pandas as pd

matplotlib.use('Agg')
import matplotlib.pyplot as plt

"""
Note that the modules (numpy, maplotlib, wave, scipy) are properly installed on your environment.

Plot wave, spectrum, save them as pdf and png at same directory.

Example:
    python calc_wave_analysis.py IR_test.wav
"""

plt.rcParams['font.family'] = 'IPAPGothic'
plt.rcParams['xtick.direction'] = 'in'
plt.rcParams['ytick.direction'] = 'in'
plt.rcParams['xtick.top'] = True
plt.rcParams['ytick.right'] = True
plt.rcParams['xtick.major.width'] = 1.0
plt.rcParams['ytick.major.width'] = 1.0
plt.rcParams['font.size'] = 11
plt.rcParams['axes.linewidth'] = 1.0
plt.rcParams['figure.figsize'] = (8, 7)
plt.rcParams['figure.dpi'] = 300
plt.rcParams['figure.subplot.hspace'] = 0.3
plt.rcParams['figure.subplot.wspace'] = 0.3


def main():
    parser = argparse.ArgumentParser(description="This script plots graph from a csv file with 3 columns.")

    parser.add_argument('csv_path',
                        action='store',
                        type=str,
                        help='Directory path where the csv file is located.',
                        metavar=None)

    parser.add_argument('-d', '--dst_path',
                        action='store',
                        nargs='?',
                        default=".",
                        type=str,
                        help='Directory path where you want to locate png files. (default: current directory)',
                        metavar=None)

    parser.add_argument('-l', '--log',
                        action='store_true',
                        help='Use y-axis logarithmic display.')

    args = parser.parse_args()

    input_name = args.csv_path
    input_name = pathlib.Path(input_name)

    is_logarithm = args.log

    df = pd.read_csv(input_name, header=None)
    print("analize file name: ", input_name)

    d, y, e = df[0], df[1], df[2]

    fig, (ax1, ax2) = plt.subplots(2, 1)
    if is_logarithm:
        ax1.set_yscale("log")
        ax2.set_yscale("log")
        d /= np.max(d)
        y /= np.max(d)
        e /= np.max(e)

    ax1.plot(d, "b--", alpha=0.5, label="desired signal d(n)")
    ax1.plot(y, "r-", alpha=0.5, label="output y(n)")
    ax1.legend()
    ax2.plot(e, "y-", alpha=1.0, label="error e(n)")
    plt.grid()
    ax2.legend()
    plt.title('ADF Output')

    output_dir = pathlib.Path(args.dst_path)
    output_name = pathlib.Path(input_name.name).with_suffix(".png")
    output_path = pathlib.Path.joinpath(output_dir, output_name)
    plt.savefig(output_path)
    print("\nfilterd data plot is saved at: ", output_path, "\n")


if __name__ == '__main__':
    main()
\end{lstlisting}

\begin{lstlisting}[caption=plot\_multiwave.py,label=plot_multiwave.py]
#!/usr/bin/env python
# coding: utf-8

# Usage:
#   python3 plot_multiwave.py foo.wav
# then you can see the wave's abstruct

import signal
import sys

import matplotlib
matplotlib.use('Agg')
import matplotlib.pyplot as plt
import soundfile as sf

signal.signal(signal.SIGINT, signal.SIG_DFL)

plt.rcParams['font.family'] = 'IPAPGothic'
plt.rcParams['font.size'] = 11
plt.rcParams['xtick.direction'] = 'in'
plt.rcParams['ytick.direction'] = 'in'
plt.rcParams['xtick.top'] = True
plt.rcParams['ytick.right'] = True
plt.rcParams['xtick.major.width'] = 1.0
plt.rcParams['ytick.major.width'] = 1.0
plt.rcParams['axes.linewidth'] = 1.0
plt.rcParams['figure.figsize'] = (8, 7)
plt.rcParams['figure.dpi'] = 300
plt.rcParams['figure.subplot.hspace'] = 0.3
plt.rcParams['figure.subplot.wspace'] = 0.3


def main():
    args = sys.argv

    if len(args) < 2:
        print("error : please pass the wave file argument")
        print("Usage: python3 plot_muiltiwave.py foo.wav")
        sys.exit()

    for i in range(len(args) - 1):
        data, sr = sf.read(args[i + 1])

        plt.plot(data, alpha=0.5, label=args[i + 1])
        plt.legend()
        plt.xlabel("Iteration")
        plt.grid(True)

    plt.show()


if __name__ == '__main__':
    main()
\end{lstlisting}

\begin{lstlisting}[caption=plot\_spectrogram\_librosa.py,label=plot_spectrogram_librosa.py]
#!/usr/bin/env python
# encoding: utf-8

import argparse
import pathlib

import librosa
import librosa.display
import matplotlib
import numpy as np
import pandas as pd

matplotlib.use('Agg')
import matplotlib.pyplot as plt

plt.rcParams['font.family'] = 'IPAPGothic'
plt.rcParams['font.size'] = 16
plt.rcParams['xtick.direction'] = 'in'
plt.rcParams['ytick.direction'] = 'in'
plt.rcParams['xtick.top'] = True
plt.rcParams['ytick.right'] = True
plt.rcParams['xtick.major.width'] = 1.0
plt.rcParams['ytick.major.width'] = 1.0
plt.rcParams['axes.linewidth'] = 1.0
plt.rcParams['figure.figsize'] = (8, 7)
plt.rcParams['figure.dpi'] = 300
plt.rcParams['figure.subplot.hspace'] = 0.3
plt.rcParams['figure.subplot.wspace'] = 0.3


def main():
    description = "This script plots spectrogram from csv or wav file."
    parser = argparse.ArgumentParser(description=description)

    parser.add_argument('input_path',
                        action='store',
                        type=str,
                        help='Directory path where the input file is located.',
                        metavar=None)

    parser.add_argument('-d', '--dst_path',
                        action='store',
                        nargs='?',
                        const="/tmp",
                        default=".",
                        type=str,
                        help='Directory path where you want to locate img files. (default: current directory)',
                        metavar=None)

    parser.add_argument('-l', '--log',
                        action='store_true',
                        help='Use y-axis logarithmic display.')

    args = parser.parse_args()

    input_path = pathlib.Path(args.input_path)
    output_dir = pathlib.Path(args.dst_path)
    is_logarithm = args.log

    sr = 48000
    data = []
    if input_path.suffix == ".wav":
        data, sr = librosa.load(str(input_path), sr=None)
    elif input_path.suffix == ".csv":
        df = pd.read_csv(str(input_path), header=None)
        data = np.array(df[0], dtype=np.float)
    else:
        parser.usage()
        exit(1)

    d = np.abs(librosa.stft(data))
    log_D = librosa.amplitude_to_db(d, ref=np.max)

    plt.figure(figsize=(8, 7))
    if is_logarithm:
        librosa.display.specshow(log_D, sr=sr, x_axis='time', y_axis='log')
        librosa.display.specshow(log_D, sr=sr, x_axis='time', y_axis='log')
    else:
        librosa.display.specshow(log_D, sr=sr, x_axis='s', y_axis='linear')

    plt.set_cmap("inferno")
    plt.xlabel("Time [sec]")
    plt.ylabel("Frequency [Hz]")
    plt.title('Spectrogram')
    plt.colorbar(format='%2.0f dB')
    plt.tight_layout()
    plt.show()

    output_name = pathlib.Path(input_path.name).with_suffix(".png")
    output_path = pathlib.Path.joinpath(output_dir, output_name)

    plt.savefig(str(output_path))

    print(f"\nimage is saved at: {output_path}\n")


if __name__ == '__main__':
    main()
\end{lstlisting}


\section{Go言語で制作したプログラム}



\subsection{ツール}

\begin{lstlisting}[caption=ディレクトリ構成,label=ディレクトリ構成]
.
├── converter.go
├── filehandler.go
├── calculator.go
└── cmd
   ├── calc_noise_mix
   │   └── main.go
   ├── convolve_wav
   │   └── main.go
   ├── convolve_wav_coef
   │   └── main.go
   ├── convolve_wav_fast
   │   └── main.go
   ├── csv_to_wav
   │   └── main.go
   ├── drone_json_generator
   │   └── main.go
   ├── multirecord
   │   ├── main.go
   │   ├── multirecord.go
   │   ├── util.go
   │   └── wav_to_DSB.go
   └── wav_to_DSB
       ├── main.go
       ├── util.go
       └── wtod.go
\end{lstlisting}

\begin{lstlisting}[caption=converter.go,label=converter.go]
package goresearch

import (
    "math"

    "gonum.org/v1/gonum/floats"
)

func abs(x int) int {
    if x < 0 {
        return -1 * x
    }
    return x
}

func AbsFloat64s(fs []float64) []float64 {
    fsAbs := make([]float64, len(fs))
    for i, v := range fs {
        fsAbs[i] = math.Abs(v)
    }
    return fsAbs
}

func AbsInts(is []int) []int {
    isAbs := make([]int, len(is))
    for i, v := range is {
        isAbs[i] = abs(v)
    }
    return isAbs
}

func NormToMaxInt16(data []float64) []float64 {

    maxAmp := floats.Max(AbsFloat64s(data))
    if maxAmp > math.MaxInt16+1 {
        reductionRate := math.MaxInt16 / maxAmp
        for i, _ := range data {
            data[i] *= reductionRate
        }
    }
    return data
}

func Int16sToInts(i16s []int16) []int {
    var is = make([]int, len(i16s))
    for i, v := range i16s {
        is[i] = int(v)
    }
    return is
}

func Float64sToInts(fs []float64) []int {
    is := make([]int, len(fs))
    for i, s := range fs {
        is[i] = int(s)
    }
    return is
}

func IntsToFloat64s(is []int) []float64 {
    fs := make([]float64, len(is))
    for i, s := range is {
        fs[i] = float64(s)
    }
    return fs
}
func Float64sToComplex128s(fs []float64) []complex128 {
    cs := make([]complex128, len(fs))
    for i, f := range fs {
        cs[i] = complex(f, 0)
    }
    return cs
}

func Complex128sToFloat64s(cs []complex128) []float64 {
    fs := make([]float64, len(cs))
    for i, c := range cs {
        fs[i] = real(c)
    }
    return fs
}
\end{lstlisting}

\begin{lstlisting}[caption=calculator.go,label=calculator.go]
package goresearch

import (
	"errors"
	"fmt"
	"math"

	"github.com/mjibson/go-dsp/fft"
)

func CalcAdjustedRMS(cleanRMS float64, snr float64) (noiseRMS float64) {
	a := snr / 20
	noiseRMS = cleanRMS / (math.Pow(10, a))
	return noiseRMS
}

func CalcRMS(amp []float64) float64 {
	var sum float64
	for _, v := range amp {
		sum += v * v
	}
	return math.Sqrt(sum / float64(len(amp)))
}

func CalcMSE(a []float64, b []float64) (float64, error) {
	if b == nil {
		b = make([]float64, len(a))
	} else if len(a) != len(b) {
		return 0, errors.New("length of a and b must agree")
	}

	var sum float64
	for i := 0; i < len(a); i++ {
		sum += (a[i] - b[i]) * (a[i] - b[i])
	}

	return sum / float64(len(a)), nil
}

func Convolve(xs, ys []float64) []float64 {
	var convLen, sumLen = len(xs), len(ys)
	if convLen > sumLen {
		ys = append(ys, make([]float64, convLen-sumLen)...)
	} else {
		convLen, sumLen = sumLen, convLen
		xs = append(xs, make([]float64, convLen-sumLen)...)
	}
	var rs = make([]float64, convLen)
	var nodeSum float64
	var i, j int
	for i = 0; i < convLen; i++ {
		for j = 0; j < sumLen; j++ {
			if i-j < 0 {
				continue
			}
			nodeSum += xs[i-j] * ys[j]
		}
		rs[i] = nodeSum
		nodeSum = 0
	}
	return rs
}

// FastConvolve - Linear fast convolution
func FastConvolve(xs, ys []float64) []float64 {
	L := len(xs)
	N := len(ys)
	M := N + L - 1

	// zero padding
	xsz := append(xs, make([]float64, M-L)...)
	ysz := append(ys, make([]float64, M-N)...)

	var rs = make([]float64, M)
	var Rs = make([]complex128, M)

	fmt.Printf("calcurating fft...\n")

	Xs := fft.FFT(Float64sToComplex128s(xsz))
	Ys := fft.FFT(Float64sToComplex128s(ysz))

	for i := 0; i < M; i++ {
		// progress
		fmt.Printf("calucurating convolution... %d%%\r", (i+1)*100/M)
		Rs[i] = Xs[i] * Ys[i]
	}
	fmt.Printf("\ncalcurating ifft...\n")

	rs = Complex128sToFloat64s(fft.IFFT(Rs))

	return rs
}

func LinSpace(start, end float64, n int) []float64 {
	res := make([]float64, n)
	if n == 1 {
		res[0] = end
		return res
	}
	delta := (end - start) / (float64(n) - 1)
	for i := 0; i < n; i++ {
		res[i] = start + (delta * float64(i))
	}
	return res
}
\end{lstlisting}

\begin{lstlisting}[caption=filehandler.go,label=filehandler.go]
package goresearch

import (
	"bufio"
	"fmt"
	"os"
	"path/filepath"
	"strconv"
	"strings"

	"github.com/go-audio/audio"
	"github.com/go-audio/wav"
)

func ReadDataFromCSV(inputPath string) (ds []float64, ys []float64, es []float64) {
	fr, err := os.Open(inputPath)
	check(err)
	sc := bufio.NewScanner(fr)
	var ss []string
	var d float64
	var y float64
	var e float64
	for sc.Scan() {
		ss = strings.Split(sc.Text(), ",")
		d, err = strconv.ParseFloat(ss[0], 64)
		check(err)
		ds = append(ds, d)
		y, err = strconv.ParseFloat(ss[1], 64)
		check(err)
		ys = append(ys, y)
		e, err = strconv.ParseFloat(ss[2], 64)
		check(err)
		es = append(es, e)
	}
	return ds, ys, es
}

func ReadDataFromCSVOne(inputPath string, column int) (fs []float64) {
	fr, err := os.Open(inputPath)
	check(err)
	sc := bufio.NewScanner(fr)
	var ss []string
	var f float64
	for sc.Scan() {
		ss = strings.Split(sc.Text(), ",")
		f, err = strconv.ParseFloat(ss[column], 64)
		check(err)
		fs = append(fs, f)
	}
	return fs
}

func ReadCoefFromCSV(inputPath string) (ws []float64) {
	fr, err := os.Open(inputPath)
	check(err)
	sc := bufio.NewScanner(fr)
	var ss []string
	var w float64
	for sc.Scan() {
		ss = strings.Split(sc.Text(), ",")
		w, err = strconv.ParseFloat(ss[0], 64)
		check(err)
		ws = append(ws, w)
	}
	return ws
}

func ReadDataFromWav(name string) []int {
	f, err := os.Open(name)
	check(err)
	defer f.Close()
	wavFile := wav.NewDecoder(f)
	check(err)

	wavFile.ReadInfo()
	ch := int(wavFile.NumChans)
	//byteRate := int(w.BitDepth/8) * ch
	//bps := byteRate / ch
	fs := int(wavFile.SampleRate)
	fmt.Println("ch", ch, "fs", fs)

	buf, err := wavFile.FullPCMBuffer()
	check(err)
	fmt.Printf("SourceBitDepth: %v\n", buf.SourceBitDepth)

	return buf.Data
}

func SaveDataAsWav(data []float64, dataDir string, name string) {
	outputPath := filepath.Join(dataDir, name+".wav")
	fw, err := os.Create(outputPath)
	check(err)

	const (
		SampleRate    = 48000
		BitsPerSample = 16
		NumChannels   = 1
		PCM           = 1
	)

	w1 := wav.NewEncoder(fw, SampleRate, BitsPerSample, NumChannels, PCM)
	aBuf := new(audio.IntBuffer)
	aBuf.Format = &audio.Format{
		NumChannels: NumChannels,
		SampleRate:  SampleRate,
	}
	aBuf.SourceBitDepth = BitsPerSample

	aBuf.Data = Float64sToInts(data)
	err = w1.Write(aBuf)
	check(err)

	err = w1.Close()
	check(err)

	err = fw.Close()
	check(err)

	fmt.Printf("\nwav file saved at: %v\n", outputPath)
}

func SaveDataAsCSV(d, y, e, mse []float64, dataDir string, testName string) {
	n := len(d)
	fw, err := os.Create(filepath.Join(dataDir, testName+".csv"))
	check(err)
	writer := bufio.NewWriter(fw)
	for i := 0; i < n; i++ {
		if i >= len(mse) {
			_, err = fmt.Fprintf(writer, "%g,%g,%g\n", d[i], y[i], e[i])
			check(err)
			continue
		}
		_, err = fmt.Fprintf(writer, "%g,%g,%g,%g\n", d[i], y[i], e[i], mse[i])
		check(err)
	}
	err = writer.Flush()
	check(err)
	err = fw.Close()
	check(err)
}

func SaveDataAsCSVOne(fs []float64, dataDir string, testName string) {
	n := len(fs)
	fw, err := os.Create(filepath.Join(dataDir, testName+".csv"))
	check(err)
	writer := bufio.NewWriter(fw)
	for i := 0; i < n; i++ {
		_, err = fmt.Fprintf(writer, "%g\n", fs[i])
		check(err)
	}
	err = writer.Flush()
	check(err)
	err = fw.Close()
	check(err)

	fmt.Printf("output file is saved at: %v\n", testName+".csv")
}

func SplitPathAndExt(path string) (string, string) {
	return filepath.Join(filepath.Dir(filepath.Clean(path)), filepath.Base(path[:len(path)-len(filepath.Ext(path))])), filepath.Ext(path)
}

func check(err error) {
	if err != nil {
		panic(err)
	}
}
\end{lstlisting}

\begin{lstlisting}[caption=cmd/calc\_noise\_mix/main.go,label=calc_noise_mix]
package main

import (
    "flag"
    "fmt"
    "log"
    "math"
    "math/rand"
    "os"
    "path/filepath"
    "strconv"

    "github.com/go-audio/audio"
    "github.com/go-audio/wav"
    "gonum.org/v1/gonum/floats"

    "github.com/tetsuzawa/research-tools/goresearch"
)

func main() {

    log.SetFlags(log.LstdFlags | log.Lshortfile)

    var (
        cleanFilepath string
        noiseFilepath string
        outputDir     string
        snrStart      float64
        snrEnd        float64
        snrDiv        int
    )

    flag.StringVar(&cleanFilepath, "clean", "/path/to/clean_file.wav", "designate clean file path")
    flag.StringVar(&noiseFilepath, "noise", "/path/to/noise_file.wav", "designate noise file path")
    flag.StringVar(&outputDir, "output", "/path/to/dir/", "designate ouput directory")
    flag.Float64Var(&snrStart, "start", -40, "designate start value of S/N Rate")
    flag.Float64Var(&snrEnd, "end", 40, "designate end value of S/N Rate")
    flag.IntVar(&snrDiv, "div", 19, "designate number of divisions")

    flag.Parse()

    if cleanFilepath == "/path/to/clean_file.wav" ||
        noiseFilepath == "/path/to/noise_file.wav" ||
        outputDir == "/path/to/dir/" {
        flag.Usage()
        os.Exit(1)
    }
    fmt.Println("clean file path:", cleanFilepath)
    fmt.Println("noise file path:", noiseFilepath)
    fmt.Println("ouput directory:", outputDir)
    fmt.Println("start value of S/N Rate:", snrStart)
    fmt.Println("end value of S/N Rate:", snrEnd)
    fmt.Println("number of divisions:", snrDiv)

    f1, err := os.Open(cleanFilepath)
    check(err)
    w1 := wav.NewDecoder(f1)

    f2, err := os.Open(noiseFilepath)
    check(err)
    w2 := wav.NewDecoder(f2)

    w1.ReadInfo()
    w2.ReadInfo()
    ch1 := int(w1.NumChans)
    ch2 := int(w2.NumChans)
    bitDepth1 := int(w1.BitDepth)
    bitDepth2 := int(w2.BitDepth)
    bps1 := bitDepth1 / 8
    bps2 := bitDepth2 / 8
    fs1 := int(w1.SampleRate)
    fs2 := int(w2.SampleRate)

    buf1, err := w1.FullPCMBuffer()
    check(err)
    buf2, err := w2.FullPCMBuffer()
    check(err)

    err = f1.Close()
    check(err)
    err = f2.Close()
    check(err)

    if ch1 != ch2 ||
        bitDepth1 != bitDepth2 ||
        bps1 != bps2 ||
        fs1 != fs2 {
        err = fmt.Errorf("format of wav files are not agree")
        panic(err)
    }

    cleanAMP := goresearch.IntsToFloat64s(buf1.Data)
    noiseAMP := goresearch.IntsToFloat64s(buf2.Data)

    cleanRMS := goresearch.CalcRMS(cleanAMP)

    var start int
    var cutNoiseAmp []float64
    if len(cleanAMP) == len(noiseAMP) {
        start = 0
        cutNoiseAmp = noiseAMP[start : start+len(cleanAMP)]
    } else if len(cleanAMP) > len(noiseAMP) {
        start = rand.Intn(len(cleanAMP) - len(noiseAMP))
        cleanAMP = cleanAMP[start : start+len(cleanAMP)]
        cutNoiseAmp = noiseAMP
    } else {
        start = rand.Intn(len(noiseAMP) - len(cleanAMP))
        cutNoiseAmp = noiseAMP[start : start+len(cleanAMP)]
    }
    noiseRMS := goresearch.CalcRMS(cutNoiseAmp)
    snrList := goresearch.LinSpace(snrStart, snrEnd, snrDiv)

    var (
        adjustedNoiseAmp = make([]float64, len(cutNoiseAmp))
        mixedAmp         = make([]float64, len(cutNoiseAmp))
        fw               *os.File
        ww               *wav.Encoder
        wBuf             = new(audio.IntBuffer)
        outputName       string
        outputPath       string
    )
    wBuf.Format = &audio.Format{
        NumChannels: ch1,
        SampleRate:  fs1,
    }
    wBuf.SourceBitDepth = bitDepth1
    for _, snr := range snrList {

        adjustedNoiseRMS := goresearch.CalcAdjustedRMS(cleanRMS, snr)

        for i, v := range cutNoiseAmp {
            adjustedNoiseAmp[i] = v * (adjustedNoiseRMS / noiseRMS)
            mixedAmp[i] = cleanAMP[i] + adjustedNoiseAmp[i]
        }
        maxAmp := floats.Max(goresearch.AbsFloat64s(mixedAmp))
        if maxAmp > math.MaxInt16+1 {
            reductionRate := math.MaxInt16 / maxAmp
            for i, _ := range cutNoiseAmp {
                mixedAmp[i] *= reductionRate
            }
        }

        wBuf.Data = goresearch.Float64sToInts(mixedAmp)

        outputName, _ = goresearch.SplitPathAndExt(cleanFilepath)
        outputPath = filepath.Join(outputDir, filepath.Base(outputName)+"_snr"+strconv.Itoa(int(snr))+".wav")
        fw, err = os.Create(outputPath)
        check(err)
        ww = wav.NewEncoder(fw, fs1, bitDepth1, ch1, 1)
        err = ww.Write(wBuf)
        check(err)
        err = ww.Close()
        check(err)

    }
    fmt.Printf("\nSuccessfully created following SN Rate files!! SNR: %v\n", snrList)
}

func check(err error) {
    if err != nil {
        panic(err)
    }
}
\end{lstlisting}

\begin{lstlisting}[caption=cmd/convolve\_wav/main.go,label=convolve_wav]
package main

import (
	"flag"
	"fmt"
	"log"
	"math"
	"os"

	"github.com/go-audio/audio"
	"github.com/go-audio/wav"
	"gonum.org/v1/gonum/floats"

	"github.com/tetsuzawa/research-tools/goresearch"
)

func main() {

	log.SetFlags(log.LstdFlags | log.Lshortfile)

	var (
		inputFilePath1 string
		inputFilePath2 string
		outputFilePath string
	)

	flag.StringVar(&inputFilePath1, "x", "/path/to/name.wav", "designate input wav file path")
	flag.StringVar(&inputFilePath2, "y", "/path/to/name.wav", "designate input wav file path")
	flag.StringVar(&outputFilePath, "o", "/path/to/name.wav", "designate output wav file path")

	flag.Parse()

	if inputFilePath1 == "/path/to/name.wav" ||
		inputFilePath2 == "/path/to/name.wav" ||
		outputFilePath == "/path/to/name.wav" {
		flag.Usage()
		os.Exit(1)
	}

	fmt.Println("wav_1 file path:", inputFilePath1)
	fmt.Println("wav_2 file path:", inputFilePath2)
	fmt.Println("output file path:", outputFilePath)

	f1, err := os.Open(inputFilePath1)
	check(err)
	w1 := wav.NewDecoder(f1)

	f2, err := os.Open(inputFilePath2)
	check(err)
	w2 := wav.NewDecoder(f2)

	w1.ReadInfo()
	w2.ReadInfo()
	ch1 := int(w1.NumChans)
	ch2 := int(w2.NumChans)
	bitDepth1 := int(w1.BitDepth)
	bitDepth2 := int(w2.BitDepth)
	bps1 := bitDepth1 / 8
	bps2 := bitDepth2 / 8
	fs1 := int(w1.SampleRate)
	fs2 := int(w2.SampleRate)

	buf1, err := w1.FullPCMBuffer()
	check(err)
	buf2, err := w2.FullPCMBuffer()
	check(err)

	err = f1.Close()
	check(err)
	err = f2.Close()
	check(err)

	if ch1 != ch2 ||
		bitDepth1 != bitDepth2 ||
		bps1 != bps2 ||
		fs1 != fs2 {
		err = fmt.Errorf("format of wav files are not agree")
		panic(err)
	}

	amp1 := goresearch.IntsToFloat64s(buf1.Data)
	amp2 := goresearch.IntsToFloat64s(buf2.Data)

	ampOut := goresearch.Convolve(amp1, amp2)

	var (
		fw             *os.File
		ww             *wav.Encoder
		wBuf           = new(audio.IntBuffer)
		outputFileName string
		outputPath_1   string
	)

	wBuf.Format = &audio.Format{
		NumChannels: ch1,
		SampleRate:  fs1,
	}
	wBuf.SourceBitDepth = bitDepth1

	maxAmp := floats.Max(goresearch.AbsFloat64s(ampOut))
	if maxAmp > math.MaxInt16+1 {
		reductionRate := math.MaxInt16 / maxAmp
		for i, _ := range ampOut {
			ampOut[i] *= reductionRate
		}
	}

	wBuf.Data = goresearch.Float64sToInts(ampOut)

	outputFileName, _ = goresearch.SplitPathAndExt(outputFilePath)

	outputPath_1 = outputFileName + ".wav"
	fw, err = os.Create(outputPath_1)
	check(err)
	ww = wav.NewEncoder(fw, fs1, bitDepth1, ch1, 1)
	err = ww.Write(wBuf)
	check(err)
	err = ww.Close()
	check(err)

	fmt.Printf("\nSuccessfully calcurated convolution! \n")
}

func check(err error) {
	if err != nil {
		panic(err)
	}
}
\end{lstlisting}

\begin{lstlisting}[caption=cmd/convolve\_wav\_fast/main.go,label=convolve_wav_fast]
package main

import (
	"flag"
	"fmt"
	"os"
	"path/filepath"

	"github.com/tetsuzawa/research-tools/goresearch"
)

func main() {
	var (
		wavPath  string
		coefPath string
		dataDir  string
	)

	flag.StringVar(&wavPath, "wav", "", "wav path")
	flag.StringVar(&coefPath, "coef", "", "coefficients path (.wav or .csv)")
	flag.StringVar(&dataDir, "dir", "./", "save dir")

	flag.Parse()

	if wavPath == "" {
		fmt.Printf("please specify wav path\n\n")
		flag.Usage()
		os.Exit(1)
	}

	if coefPath == "" {
		fmt.Printf("please specify coef path\n\n")
		flag.Usage()
		os.Exit(1)
	}

	name, ext := goresearch.SplitPathAndExt(wavPath)
	if ext != ".wav" {
		fmt.Printf("please specify wav path\n\n")
		flag.Usage()
		os.Exit(1)
	}

	fmt.Println("wavPath:", wavPath)
	fmt.Println("coefPath:", coefPath)
	fmt.Println("dataDir:", dataDir)

	wData := goresearch.ReadDataFromWav(wavPath)

	_, cExt := goresearch.SplitPathAndExt(coefPath)
	var cData []float64
	switch cExt {
	case ".wav":
		cData = goresearch.IntsToFloat64s(goresearch.ReadDataFromWav(coefPath))
	case ".csv":
		cData = goresearch.ReadCoefFromCSV(coefPath)
	default:
		fmt.Printf("file type is not valid. coefficients file name:%v", coefPath)
		os.Exit(1)
	}

	wDataF := goresearch.IntsToFloat64s(wData)
	convData := goresearch.FastConvolve(wDataF, cData)

	outputName := filepath.Base(name) + "_convoluted"

	data := goresearch.NormToMaxInt16(convData)

	goresearch.SaveDataAsWav(data, dataDir, outputName)

}
\end{lstlisting}

\begin{lstlisting}[caption=cmd/csv\_to\_wav/main.go,label=csv_to_wav]
    package main

    import (
        "flag"
        "fmt"
        "os"
        "path/filepath"
    
        "github.com/tetsuzawa/research-tools/goresearch"
    )
    
    func main() {
    
        var (
            csvPath string
            dataDir string
        )
    
        flag.StringVar(&csvPath, "path", "", "csv path")
        flag.StringVar(&dataDir, "dir", "./", "save dir")
    
        flag.Parse()
    
        if csvPath == "" {
            fmt.Printf("please specify csv path\n\n")
            flag.Usage()
            os.Exit(1)
        }
    
        name, ext := goresearch.SplitPathAndExt(csvPath)
        if ext != ".csv" {
            fmt.Printf("please specify csv path\n\n")
            flag.Usage()
            os.Exit(1)
        }
    
        fmt.Println("csvPath:", csvPath)
        fmt.Println("dataDir:", dataDir)
    
        _, _, e := goresearch.ReadDataFromCSV(csvPath)
    
        outputName := filepath.Base(name)
    
        data := goresearch.NormToMaxInt16(e)
    
        goresearch.SaveDataAsWav(data, dataDir, outputName)
    }
\end{lstlisting}

\begin{lstlisting}[caption=cmd/drone\_json\_generator/main.go,label=drone_json_generator]
package main

import (
    "encoding/json"
    "flag"
    "fmt"
    "os"
    "path/filepath"
)

type ADFConfig struct {
    WavName string  `json:"wav_name"`
    AdfName string  `json:"adf_name"`
    Mu      float64 `json:"mu"`
    L       int     `json:"l"`
    Order   int     `json:"order"`
}

func main() {
    var (
        wavName string
        adfName string
        L       int
        Mu      float64
        order   int
        dataDir string
    )

    flag.StringVar(&wavName, "wav", "../wavfiles/dr_static_20.wav", "wav name")
    flag.StringVar(&adfName, "adf", "NLMS", "algorithm")
    flag.Float64Var(&Mu, "mu", 1.0, "mu")
    flag.IntVar(&L, "L", 256, "L")
    flag.IntVar(&order, "order", 8, "order")
    flag.StringVar(&dataDir, "dir", "./", "save dir")

    flag.Parse()

    fmt.Println("wavName:", wavName)
    fmt.Println("adfName:", adfName)
    fmt.Println("mu:", Mu)
    fmt.Println("L:", L)
    fmt.Println("order:", order)
    fmt.Println("dataDir:", dataDir)

    var testName string
    applicationName := "static"

    switch adfName {
    case "LMS":
        testName = fmt.Sprintf("%v_%v_L-%v", adfName, applicationName, L)
    case "NLMS":
        testName = fmt.Sprintf("%v_%v_L-%v", adfName, applicationName, L)
    case "AP":
        testName = fmt.Sprintf("%v_%v_L-%v_order-%v", adfName, applicationName, L, order)
    case "RLS":
        testName = fmt.Sprintf("%v_%v_L-%v", adfName, applicationName, L)
    default:
        err := fmt.Errorf("\nadfName is not valid:%v\n", adfName)
        fmt.Println(err)
        fmt.Printf("Failed!\n")
        os.Exit(1)
    }
    fmt.Printf("testName: %v\n", testName)

    var adf = &ADFConfig{
        WavName: wavName,
        AdfName: adfName,
        Mu:      Mu,
        L:       L,
        Order:   order,
    }

    outadfJSON, err := json.Marshal(adf)
    check(err)
    fw, err := os.Create(filepath.Join(dataDir, testName+".json"))
    check(err)
    defer fw.Close()
    _, err = fw.Write(outadfJSON)
    check(err)

    fmt.Printf("json file saved at :%v\n", filepath.Join(dataDir, testName+".json"))

}

func check(err error) {
    if err != nil {
        panic(err)
    }
}

\end{lstlisting}

\begin{lstlisting}[caption=cmd/calc\_mse\_csv/main.go,label=calc_mse_csv]
package main

import (
	"flag"
	"fmt"
	"gonum.org/v1/gonum/floats"
	"math"
	"path/filepath"

	"github.com/tetsuzawa/research-tools/goresearch"
)

func main() {
	var tap int
	flag.IntVar(&tap, "tap", 256, "mse taps")

	flag.Parse()

	fmt.Println("mse taps:", tap)

	inputPath := flag.Arg(0)
	fmt.Println("inputPath:", inputPath)

	dataDir := flag.Arg(1)
	fmt.Println("dataDir:", dataDir)

	ds, ys, es := goresearch.ReadDataFromCSV(inputPath)

	var mse = make([]float64, len(es)-tap)
	var v float64
	var err error
	for i := 0; i < len(es)-tap; i++ {

		fmt.Printf("working... %d%%\r", (i+1)*100/(len(es)-tap))

		v, err = goresearch.CalcMSE(es[i:i+tap], nil)
		check(err)
		mse[i] = 20 * math.Log10(v)
	}
	floats.AddConst(-1*floats.Max(mse), mse)

	name, _ := goresearch.SplitPathAndExt(inputPath)
	outputName := filepath.Base(name) + "_mse"

	goresearch.SaveDataAsCSV(ds, ys, es, mse, dataDir, outputName)

}

func check(err error) {
	if err != nil {
		panic(err)
	}
}
\end{lstlisting}

\begin{lstlisting}[caption=cmd/multirecord/main.go,label=multirecord/main]
    /*
    Contents: multi channel record.
        This program works as typical recording app with multi channels.
        Output file format is .wav.
        Please run `multirecord --help` for details.
    Usage: multirecord (-c ch -r rate -b bits -o /path/to/out.wav) sec
    Author: Tetsu Takizawa
    E-mail: tt15219@tomakomai.kosen-ac.jp
    LastUpdate: 2019/11/18
    DateCreated  : 2019/11/18
    */
    package main
    
    import (
        "os"
    
        "github.com/urfave/cli"
    )
    
    func main() {
        app := cli.NewApp()
        defer app.Run(os.Args)
        app.CustomAppHelpTemplate = HelpTemplate
    
        app.Name = "multirecord"
        app.Usage = `This app records sounds with multi channels and save as .wav file or .DSB files if --DSB is specified.`
        app.Version = "0.1.2"
    
        app.Action = multiRecord
    
        app.Flags = []cli.Flag{
            cli.IntFlag{
                Name:  "channel, c",
                Value: 1,
                Usage: "input number of channels",
            },
            cli.IntFlag{
                Name:  "bits, b",
                Value: 16,
                Usage: "number of bits per sample",
            },
            cli.IntFlag{
                Name:  "rate, r",
                Value: 48000,
                Usage: "number of sample rate",
            },
            cli.StringFlag{
                Name:  "outpath, o",
                Value: "out_multirecord.wav",
                Usage: "specify output path",
            },
            cli.BoolFlag{
                Name:  "params, p",
                Usage: "trace import statements",
            },
            cli.BoolFlag{
                Name:  "DSB, D",
                Usage: "make .DSB files",
            },
        }
    }
\end{lstlisting}

\begin{lstlisting}[caption=cmd/multirecord/multirecord.go,label=multirecord/multirecord]
/*
Contents: multi channel record.
	This program works as typical recording app with multi channels.
	Output file format is .wav.
	Please run `multirecord --help` for details.
Usage: multirecord (-c ch -r rate -b bits -o /path/to/out.wav) sec
Author: Tetsu Takizawa
E-mail: tt15219@tomakomai.kosen-ac.jp
LastUpdate: 2019/11/18
DateCreated  : 2019/11/18
*/
package main

import (
	"fmt"
	"os"
	"strconv"
	"time"

	"github.com/go-audio/audio"
	"github.com/go-audio/wav"
	"github.com/gordonklaus/portaudio"
	"github.com/pkg/errors"
	"github.com/urfave/cli"

	"github.com/tetsuzawa/research-tools/goresearch"
)

const (
	FramesPerBuffer = 1024
	PCM             = 1
)

var (
	w1                *wav.Encoder
	RecordSeconds     float64
	NumChannels       int
	SampleRate        int
	BitsPerSample     int
	NumSamplesToWrite int
	NumWritten        int
	aBuf              = new(audio.IntBuffer)
	err               error
)

func multiRecord(ctx *cli.Context) error {
	// ************* check argument *************
	if ctx.Args().Get(0) == "" {
		return cli.NewExitError(`too few arguments. need recording duration
Usage: multirecord (-c ch -r rate -b bits -o /path/to/out.wav) duration
`, 2)
	}

	// ************* validate ext *************
	name, ext := goresearch.SplitPathAndExt(ctx.String("o"))
	if ext != ".wav" && ext != "" {
		return cli.NewExitError(`incorrect file format. multirecord saves audio as .wav file.
Usage: multirecord -o /path/to/file.wav 5.0`, 2)
	}

	// ************* validate parameter *************
	RecordSeconds, err = strconv.ParseFloat(ctx.Args().Get(0), 64)
	if err != nil {
		err = errors.Wrap(err, "error occurred while converting arg of recording time from string to float64")
		return cli.NewExitError(err, 5)
	}
	NumChannels = ctx.Int("c")
	SampleRate = ctx.Int("r")
	BitsPerSample := ctx.Int("b")
	if BitsPerSample != 16 {
		return cli.NewExitError(`sorry, this app is only for 16 bits per sample for now`, 99)
	}
	NumSamplesToWrite = int(RecordSeconds * float64(SampleRate))

	// ************* create output file *************
	f1, err := os.Create(name + ".wav")
	if err != nil {
		err = errors.Wrap(err, "internal error: error occurred while creating output file")
		return cli.NewExitError(err, 5)
	}
	defer f1.Close()

	w1 = wav.NewEncoder(f1, SampleRate, BitsPerSample, NumChannels, PCM)
	aBuf.Format = &audio.Format{
		NumChannels: NumChannels,
		SampleRate:  SampleRate,
	}
	aBuf.SourceBitDepth = BitsPerSample

	// ************* initialize portaudio*************
	err = portaudio.Initialize()
	defer portaudio.Terminate()
	if err != nil {
		err = errors.Wrap(err, "internal error: error occurred while initializing portaudio")
		return cli.NewExitError(err, 5)
	}

	h, err := portaudio.DefaultHostApi()
	if err != nil {
		err = errors.Wrap(err, "internal error: error occurred while searching host API")
		return cli.NewExitError(err, 5)
	}
	paParam := portaudio.LowLatencyParameters(h.DefaultInputDevice, h.DefaultOutputDevice)
	paParam.SampleRate = float64(SampleRate)
	paParam.Input.Channels = NumChannels
	paParam.Output.Channels = 1
	paParam.FramesPerBuffer = FramesPerBuffer

	stream, err := portaudio.OpenStream(paParam, callback)
	if err != nil {
		err = errors.Wrap(err, "internal error: error occurred while opening stream on portaudio")
		return cli.NewExitError(err, 5)
	}
	defer stream.Close()

	// ************* print parameter *************
	fmt.Printf("\nOutput File: \t`%s`\n", name+".wav")
	fmt.Printf("Channels: \t%d \n", NumChannels)
	fmt.Printf("Sample Rate: \t%d\n", SampleRate)
	fmt.Printf("Precision: \t%d-bits\n", BitsPerSample)
	fmt.Printf("Duration: \t%f [sec] = %d samples\n", RecordSeconds, NumSamplesToWrite)
	fmt.Printf("Encoding: \t%d-bits Signed Integer PCM\n\n", BitsPerSample)

	if ctx.Bool("p") {
		fmt.Printf("\nFrames Per Buffer\n", paParam.FramesPerBuffer)

		fmt.Printf("\nInput Device Parameters\n")
		fmt.Printf("Input Device Name\t\t\t%v\n", paParam.Input.Device.Name)
		fmt.Printf("Input Device MaxInputChannels\t\t%v\n", paParam.Input.Device.MaxInputChannels)
		fmt.Printf("Input Device DefaultSampleRate\t\t%v\n", paParam.Input.Device.DefaultSampleRate)
		fmt.Printf("Input Device HostApi\t\t\t%v\n", paParam.Input.Device.HostApi)
		fmt.Printf("Input Device DefaultLowInputLatency\t%v\n", paParam.Input.Device.DefaultLowInputLatency)
		fmt.Printf("Input Device DefaultHighInputLatency\t%v\n", paParam.Input.Device.DefaultHighInputLatency)

		fmt.Printf("\n\nOutput params\n")
		fmt.Printf("Output Device Name\t\t\t%v\n", paParam.Output.Device.Name)
		fmt.Printf("Output Device MaxOutputChannels\t\t%v\n", paParam.Output.Device.MaxOutputChannels)
		fmt.Printf("Output Device DefaultSampleRate\t\t%v\n", paParam.Output.Device.DefaultSampleRate)
		fmt.Printf("Output Device HostApi\t\t\t%v\n", paParam.Output.Device.HostApi)
		fmt.Printf("Output Device DefaultLowOutputLatency\t%v\n", paParam.Output.Device.DefaultLowOutputLatency)
		fmt.Printf("Output Device DefaultHighOutputLatency\t%v\n", paParam.Output.Device.DefaultHighOutputLatency)
	}

	// ************* record audio *************
	err = stream.Start()
	if err != nil {
		err = errors.Wrap(err, "internal error: error occurred while starting stream")
		return cli.NewExitError(err, 5)
	}

	fmt.Printf("\nrecording...\n")

	st := time.Now()
	for time.Since(st).Seconds() < RecordSeconds {
		fmt.Printf("%.1f[sec] : %.1f[sec]\r", time.Since(st).Seconds(), RecordSeconds)
	}
	err = stream.Stop()
	if err != nil {
		err = errors.Wrap(err, "internal error: error occurred while stopping stream")
		return cli.NewExitError(err, 5)
	}
	err = w1.Close()
	check(err)

	fmt.Printf("\n\nSuccessfully recorded!!\n")

	if ctx.Bool("D") {
		err = wavToDSB(ctx, name)
		if err != nil {
			err = errors.Wrap(err, "error occurred while converting .wav file to .DSB files")
		}
	}

	return nil
}

// ************* callback function *************
func callback(inBuf, outBuf []int16) {
	if NumWritten+FramesPerBuffer > NumSamplesToWrite {
		numWrite := NumSamplesToWrite - NumWritten
		NumWritten += numWrite
		aBuf.Data = goresearch.Int16sToInts(inBuf[:numWrite])
		err = w1.Write(aBuf)
		check(err)
		return
	}
	NumWritten += len(inBuf) / NumChannels
	aBuf.Data = goresearch.Int16sToInts(inBuf)
	err = w1.Write(aBuf)
	check(err)
}
\end{lstlisting}

\begin{lstlisting}[caption=cmd/multirecord/util.go,label=multirecord/util]
package main


func check(err error) {
	if err != nil {
		panic(err)
	}
}

var HelpTemplate = `NAME:
   {{.Name}}{{if .Usage}} - {{.Usage}}{{end}}

USAGE:
   {{if .UsageText}}{{.UsageText}}{{else}}{{.HelpName}} {{if .VisibleFlags}}[options]{{end}} {{if .ArgsUsage}}{{.ArgsUsage}}{{else}}[arguments...]{{end}}{{end}}{{if .Version}}{{if not .HideVersion}}

VERSION:
   {{.Version}}{{end}}{{end}}{{if .Description}}

DESCRIPTION:
   {{.Description}}{{end}}{{if len .Authors}}

AUTHOR{{with $length := len .Authors}}{{if ne 1 $length}}S{{end}}{{end}}:
   {{range $index, $author := .Authors}}{{if $index}}
   {{end}}{{$author}}{{end}}{{end}}{{if .VisibleCommands}}

OPTIONS:
   {{range $index, $option := .VisibleFlags}}{{if $index}}
   {{end}}{{$option}}{{end}}{{end}}`
\end{lstlisting}

\begin{lstlisting}[caption=cmd/multirecord/wav\_to\_DSB.go,label=multirecord/wav_to_DSB]
/*
Contents: multi channel record.
	This program works as typical recording app with multi channels.
	Output file format is .wav.
	Please run `multirecord --help` for details.
Usage: multirecord (-c ch -r rate -b bits -o /path/to/out.wav) sec
Author: Tetsu Takizawa
E-mail: tt15219@tomakomai.kosen-ac.jp
LastUpdate: 2019/11/18
DateCreated  : 2019/11/18
*/
package main

import (
	"bytes"
	"encoding/binary"
	"fmt"
	"os"

	"github.com/go-audio/wav"
	"github.com/pkg/errors"
	"github.com/urfave/cli"
)

func wavToDSB(ctx *cli.Context, name string) error {

	// ************* create output file *************
	f, err := os.Open(name + ".wav")
	if err != nil {
		err = errors.Wrap(err, "internal error: error occurred while creating output file")
		return cli.NewExitError(err, 5)
	}
	defer f.Close()
	w := wav.NewDecoder(f)
	if err != nil {
		err = errors.Wrap(err, "error occurred while processing .wav file")
		return cli.NewExitError(err, 3)
	}

	// ************* validate parameter *************
	w.ReadInfo()
	ch := int(w.NumChans)
	byteRate := int(w.BitDepth/8) * ch
	bps := byteRate / ch
	fs := int(w.SampleRate)

	if fs != 48000 || bps != 2 || w.WavAudioFormat != 1 {
		errMsg := fmt.Sprintf(`audio format error: wav format must be as follows.
sample rate: want 48000 Hz, got %v Hz
sampling bit rate: want 16 bits per sample, got %v bits per sample
audio format: want 1 (PCM), got %v`, fs, w.BitDepth, w.WavAudioFormat)
		return cli.NewExitError(errMsg, 3)
	}

	aBuf, err := w.FullPCMBuffer()
	if err != nil {
		err = errors.Wrap(err, "error occurred while processing .wav file")
		return cli.NewExitError(err, 3)
	}

	if aBuf.SourceBitDepth != 16 {
		err = errors.New("sampling bit rate is incorrect. need 16 bits per sample")
		err = errors.Wrap(err, "error occurred while processing .wav file")
		return cli.NewExitError(err, 3)
	}

	iter := aBuf.NumFrames()
	wBuf := make([]byte, iter*bps)

	// ************* split channel and write to .DSB files *************
	var fw *os.File
	for j := 0; j < ch; j++ {
		if ch == 1 {
			fw, err = os.Create(fmt.Sprintf("%s.DSB", name))
		} else {
			fw, err = os.Create(fmt.Sprintf("%s_ch%d.DSB", name, j+1))
		}
		if err != nil {
			return cli.NewExitError(err, 3)
		}
		defer fw.Close()

		for i := 0; i < iter; i++ {
			fmt.Printf("converting... %d%%\r", (i+1)*100/iter)
			b := new(bytes.Buffer)
			//err = binary.Write(b, binary.LittleEndian, value[ch*i+j])
			err = binary.Write(b, binary.LittleEndian, int16(aBuf.Data[ch*i+j]))
			if err != nil {
				err = errors.Wrap(err, "internal error: error occurred while writing data to buffer")
				return cli.NewExitError(err, 5)
			}
			copy(wBuf[bps*i:bps*(i+1)], b.Bytes())
		}
		_, err = fw.Write(wBuf)
		if err != nil {
			err = errors.Wrap(err, "error occurred while writing data to .DSB file")
			return cli.NewExitError(err, 3)
		}

	}
	if ch == 1 {
		fmt.Printf("\n\n%d file created as `%s.DSB`\n", ch, name)
	} else {
		fmt.Printf("\n\n%d files created as `%s_chX.DSB`\n", ch, name)
	}

	fmt.Printf("\n")
	fmt.Println("end!!")

	return nil
}
\end{lstlisting}

\begin{lstlisting}[caption=cmd/wav\_to\_DSB/main.go,label=wav_to_DSB/main]
/*
Contents: wav to DSB converter.
	This program converts .wav file to .DSB files.
	Please run `wav_to_DSB --help` for details.
Usage: wav_to_DSB (-o /path/to/out.DSB) /path/to/file.wav
Author: Tetsu Takizawa
E-mail: tt15219@tomakomai.kosen-ac.jp
LastUpdate: 2019/11/16
DateCreated  : 2019/11/16
*/
package main

import (
	"os"

	"github.com/pkg/errors"
	"github.com/urfave/cli"

	"github.com/tetsuzawa/research-tools/goresearch"
)

func main() {
	app := cli.NewApp()
	defer app.Run(os.Args)
	app.CustomAppHelpTemplate = HelpTemplate

	app.Name = "wav_to_DSB"
	app.Usage = `This app converts .wav file to .DSB file.`
	app.Version = "0.1.0"

	app.Action = action

	app.Flags = []cli.Flag{
		cli.StringFlag{
			Name:  "outpath, o",
			Usage: "specify output path like as /path/to/file",
		},
	}
}

func action(ctx *cli.Context) error {
	if ctx.Args().Get(0) == "" {
		return cli.NewExitError("too few arguments. need input file path. \nUsage: wav-to-DSB-multi /path/to/file.wav", 2)
	}

	fileName := ctx.Args().Get(0)
	name, ext := goresearch.SplitPathAndExt(fileName)

	if ext != ".wav" {
		return cli.NewExitError("incorrect file format. need .wav file. \nUsage: wav-to-DSB-multi /path/to/file.wav", 2)
	}

	if ctx.String("o") != "" {
		argName := ctx.String("o")
		name, _ = goresearch.SplitPathAndExt(argName)
	}

	f, err := os.Open(fileName)
	if err != nil {
		err = errors.Wrap(err, "error occurred while opening input file")
		return cli.NewExitError("no such a file", 2)
	}
	defer f.Close()

	return wavToDSB(ctx, f, name)
}
\end{lstlisting}

\begin{lstlisting}[caption=cmd/wav\_to\_DSB/util.go,label=wav_to_DSB/util]
package main

var HelpTemplate = `NAME:
    {{.Name}}{{if .Usage}} - {{.Usage}}{{end}}

USAGE:
    {{if .UsageText}}{{.UsageText}}{{else}}{{.HelpName}} {{if .VisibleFlags}}[options]{{end}} {{if .ArgsUsage}}{{.ArgsUsage}}{{else}}[arguments...]{{end}}{{end}}{{if .Version}}{{if not .HideVersion}}

VERSION:
    {{.Version}}{{end}}{{end}}{{if .Description}}

DESCRIPTION:
    {{.Description}}{{end}}{{if len .Authors}}

AUTHOR{{with $length := len .Authors}}{{if ne 1 $length}}S{{end}}{{end}}:
    {{range $index, $author := .Authors}}{{if $index}}
    {{end}}{{$author}}{{end}}{{end}}{{if .VisibleCommands}}

OPTIONS:
    {{range $index, $option := .VisibleFlags}}{{if $index}}
    {{end}}{{$option}}{{end}}{{end}}`
\end{lstlisting}

\begin{lstlisting}[caption=cmd/wav\_to\_DSB/wtod.go,label=wav_to_DSB/wtod]
package main

import (
	"bytes"
	"encoding/binary"
	"fmt"
	"github.com/go-audio/wav"
	"os"

	"github.com/pkg/errors"
	"github.com/urfave/cli"
)

var err error

func wavToDSB(ctx *cli.Context, f *os.File, name string) error {
	//w, err := wav.NewReader(f)
	w := wav.NewDecoder(f)
	if err != nil {
		err = errors.Wrap(err, "error occurred while processing .wav file")
		return cli.NewExitError(err, 3)
	}

	w.ReadInfo()
	ch := int(w.NumChans)
	byteRate := int(w.BitDepth/8) * ch
	bps := byteRate / ch
	fs := int(w.SampleRate)

	if fs != 48000 || bps != 2 || w.WavAudioFormat != 1 {
		errMsg := fmt.Sprintf(`audio format error: wav format must be as follows.
sample rate: want 48000 Hz, got %v Hz
sampling bit rate: want 16 bits per sample, got %v bits per sample
audio format: want 1 (PCM), got %v`, fs, w.BitDepth, w.WavAudioFormat)
		return cli.NewExitError(errMsg, 3)
	}

	//data, err := w.ReadSamples(int(w.GetSubChunkSize()) / byteRate * ch)
	aBuf, err := w.FullPCMBuffer()
	if err != nil {
		err = errors.Wrap(err, "error occurred while processing .wav file")
		return cli.NewExitError(err, 3)
	}

	//value, ok := data.([]int16)
	if aBuf.SourceBitDepth != 16 {
		err = errors.New("sampling bit rate is incorrect. need 16 bits per sample")
		err = errors.Wrap(err, "error occurred while processing .wav file")
		return cli.NewExitError(err, 3)
	}

	//iter := len(value) / ch
	iter := aBuf.NumFrames() / ch
	wBuf := make([]byte, iter*bps)

	var fw *os.File
	for j := 0; j < ch; j++ {
		if ch == 1 {
			fw, err = os.Create(fmt.Sprintf("%s.DSB", name))
		} else {
			fw, err = os.Create(fmt.Sprintf("%s_ch%d.DSB", name, j+1))
		}
		if err != nil {
			return cli.NewExitError(err, 3)
		}
		defer fw.Close()

		for i := 0; i < iter; i++ {
			fmt.Printf("working... %d%%\r", (i+1)*100/iter)
			b := new(bytes.Buffer)
			//err = binary.Write(b, binary.LittleEndian, value[ch*i+j])
			err = binary.Write(b, binary.LittleEndian, int16(aBuf.Data[ch*i+j]))
			if err != nil {
				err = errors.Wrap(err, "internal error: error occurred while writing data to buffer")
				return cli.NewExitError(err, 5)
			}
			copy(wBuf[bps*i:bps*(i+1)], b.Bytes())
		}
		_, err = fw.Write(wBuf)
		if err != nil {
			err = errors.Wrap(err, "error occurred while writing data to .DSB file")
			return cli.NewExitError(err, 3)
		}

	}
	if ch == 1 {
		fmt.Printf("\n\n%d file created as %s.DSB\n", ch, name)
	} else {
		fmt.Printf("\n\n%d files created as %s_chX.DSB\n", ch, name)
	}

	fmt.Printf("\n")
	fmt.Println("end!!")

	return nil
}
\end{lstlisting}


\subsection{ADFライブラリ}\label{go-adflib_code}

\begin{lstlisting}[caption=ディレクトリ構成,label=ディレクトリ構成]
.
├── adf
│   ├── ap.go
│   ├── base.go
│   ├── lms.go
│   ├── nlms.go
│   └── rls.go
├── fdadf
│   ├── base.go
│   └── fblms.go
└── misc
    └── misc.go

# testコードやLISENCEファイルなどは省略
\end{lstlisting}

\begin{lstlisting}[caption=ap.go,label=ap.go]
package adf

import (
    "errors"
    "gonum.org/v1/gonum/mat"
)

//FiltAP is base struct for AP filter.
//Use NewFiltAP to make instance.
type FiltAP struct {
    filtBase
    order  int
    eps    float64
    wHist  [][]float64
    xMem   *mat.Dense
    dMem   *mat.Dense
    yMem   *mat.Dense
    eMem   *mat.Dense
    epsIDE *mat.Dense
    ide    *mat.Dense
}

//NewFiltAP is constructor of AP filter.
//This func initialize filter length `n`, update step size `mu`, projection order `order` and filter weight `w`.
func NewFiltAP(n int, mu float64, order int, eps float64, w []float64)(AdaptiveFilter, error) {
    var err error
    p := new(FiltAP)
    p.kind = "AP filter"
    p.n = n
    p.muMin = 0
    p.muMax = 1000
    p.mu, err = p.checkFloatParam(mu, p.muMin, p.muMax, "mu")
    if err != nil {
        return nil, err
    }
    p.order = order
    p.eps, err = p.checkFloatParam(eps, 0, 1000, "eps")
    if err != nil {
        return nil, err
    }
    err = p.initWeights(w, n)
    if err != nil {
        return nil, err
    }
    p.xMem = mat.NewDense(n, order, nil)
    p.dMem = mat.NewDense(1, order, nil)

    elmNum := order * order

    //make diagonal matrix
    diaMat := make([]float64, elmNum)
    for i := 0; i < order; i++ {
        diaMat[i*(order+1)] = eps
    }
    p.epsIDE = mat.NewDense(order, order, diaMat)

    for i := 0; i < order; i++ {
        diaMat[i*(order+1)] = 1
    }
    p.ide = mat.NewDense(order, order, diaMat)

    p.yMem = mat.NewDense(order, 1, nil)
    p.eMem = mat.NewDense(1, order, nil)

    return p, nil
}

//Adapt calculates the error `e` between desired value `d` and estimated value `y`,
//and update filter weights according to error `e`.
func (af *FiltAP) Adapt(d float64, x []float64) {
    xr, _ := af.xMem.Dims()
    xCol := make([]float64, xr)
    dr, _ := af.dMem.Dims()
    dCol := make([]float64, dr)
    // create input matrix and target vector
    // shift column
    for i := af.order - 1; i > 0; i-- {
        mat.Col(xCol, i-1, af.xMem)
        af.xMem.SetCol(i, xCol)
        mat.Col(dCol, i-1, af.dMem)
        af.dMem.SetCol(i, dCol)
    }
    af.xMem.SetCol(0, x)
    af.dMem.Set(0, 0, d)

    // estimate output and error
    af.yMem.Mul(af.xMem.T(), af.w.T())
    af.eMem.Sub(af.dMem, af.yMem.T())

    // update
    dw1 := mat.NewDense(af.order, af.order, nil)
    dw1.Mul(af.xMem.T(), af.xMem)
    dw1.Add(dw1, af.epsIDE)
    dw2 := mat.NewDense(af.order, af.order, nil)
    err := dw2.Solve(dw1, af.ide)
    if err != nil {
        panic(err)
    }
    dw3 := mat.NewDense(1, af.order, nil)
    dw3.Mul(af.eMem, dw2)
    dw := mat.NewDense(1, af.n, nil)
    dw.Mul(dw3, af.xMem.T())
    dw.Scale(af.mu, dw)
    af.w.Add(af.w, dw)
}

//Run calculates the errors `e` between desired values `d` and estimated values `y` in a row,
//while updating filter weights according to error `e`.
func (af *FiltAP) Run(d []float64, x [][]float64) (y []float64, e []float64, wHist [][]float64, err error) {
    //measure the data and check if the dimension agree
    N := len(x)
    if len(d) != N {
        return nil, nil, nil, errors.New("the length of slice d and x must agree")
    }
    af.n = len(x[0])
    af.wHist = make([][]float64, N)
    for i := 0; i < N; i++ {
        af.wHist[i] = make([]float64, af.n)
    }

    y = make([]float64, N)
    e = make([]float64, N)
    w := af.w.RawRowView(0)

    xr, _ := af.xMem.Dims()
    xCol := make([]float64, xr)
    dr, _ := af.dMem.Dims()
    dCol := make([]float64, dr)

    //adaptation loop
    for i := 0; i < N; i++ {
        copy(af.wHist[i], w)

        // create input matrix and target vector
        // shift column
        for i := af.order - 1; i > 0; i-- {
            mat.Col(xCol, i-1, af.xMem)
            af.xMem.SetCol(i, xCol)
            mat.Col(dCol, i-1, af.dMem)
            af.dMem.SetCol(i, dCol)
        }
        af.xMem.SetCol(0, x[i])
        af.dMem.Set(0, 0, d[i])

        // estimate output and error
        // same as af.yMem.Mul(af.xMem, af.w.T()).T()
        af.yMem.Mul(af.xMem.T(), af.w.T())
        af.eMem.Sub(af.dMem, af.yMem.T())
        y[i] = af.yMem.At(0, 0)
        e[i] = af.eMem.At(0, 0)

        // update
        dw1 := mat.NewDense(af.order, af.order, nil)
        dw1.Mul(af.xMem.T(), af.xMem)
        dw1.Add(dw1, af.epsIDE)
        dw2 := mat.NewDense(af.order, af.order, nil)
        err := dw2.Solve(dw1, af.ide)
        if err != nil {
            return nil, nil, nil, err
        }
        dw3 := mat.NewDense(1, af.order, nil)
        dw3.Mul(af.eMem, dw2)
        dw := mat.NewDense(af.n, 1, nil)
        dw.Mul(af.xMem, dw3.T())
        dw.Scale(af.mu, dw)
        af.w.Add(af.w, dw.T())
    }
    wHist = af.wHist
    return y, e, wHist, nil
}

func (af *FiltAP) clone() AdaptiveFilter {
    altaf := *af
    return &altaf
}

\end{lstlisting}

\begin{lstlisting}[caption=base.go,label=base.go]
package adf

import (
	"fmt"

	"github.com/pkg/errors"
	"gonum.org/v1/gonum/floats"
	"gonum.org/v1/gonum/mat"

	"github.com/tetsuzawa/go-adflib/misc"
)

// AdaptiveFilter is the basic Adaptive Filter interface type.
type AdaptiveFilter interface {
	initWeights(w []float64, n int) error
	//Predict calculates the new estimated value `y` from input slice `x`.
	Predict(x []float64) (y float64)

	//Adapt calculates the error `e` between desired value `d` and estimated value `y`,
	//and update filter weights according to error `e`.
	Adapt(d float64, x []float64)

	//Run calculates the errors `e` between desired values `d` and estimated values `y` in a row,
	//while updating filter weights according to error `e`.
	Run(d []float64, x [][]float64) (y []float64, e []float64, wHist [][]float64, err error)
	checkFloatParam(p, low, high float64, name string) (float64, error)
	checkIntParam(p, low, high int, name string) (int, error)

	//SetStepSize sets the step size of adaptive filter.
	SetStepSize(mu float64) error

	//GetParams returns the parameters at the time this func is called.
	//parameters contains `n`: filter length, `mu`: filter update step size and `w`: filter weights.
	GetParams() (n int, mu float64, w []float64)

	//GetParams returns the name of ADF.
	GetKindName() (kind string)

	clone() AdaptiveFilter
}

//Must checks whether err is nil or not. If err in not nil, this func causes panic.
func Must(af AdaptiveFilter, err error) AdaptiveFilter {
	if err != nil {
		panic(err)
	}
	return af
}

//PreTrainedRun use part of the data for few epochs of learning.
//The arg `d` is desired values.
//`x` is input matrix. columns are bunch of samples and rows are set of samples.
//`nTrain` is train to test ratio, typical value is 0.5. (that means 50% of data is used for training).
//`epochs` is number of training epochs, typical value is 1. This number describes how many times the training will be repeated.
func PreTrainedRun(af AdaptiveFilter, d []float64, x [][]float64, nTrain float64, epochs int) (y, e []float64, w [][]float64, err error) {
	var nTrainI = int(float64(len(d)) * nTrain)
	//train
	for i := 0; i < epochs; i++ {
		_, _, _, err = af.Run(d[:nTrainI], x[:nTrainI])
		if err != nil {
			return nil, nil, nil, err
		}
	}
	//run
	y, e, w, err = af.Run(d[:nTrainI], x[:nTrainI])
	if err != nil {
		return nil, nil, nil, err
	}
	return y, e, w, nil
}

//ExploreLearning searches the `mu` with the smallest error value from the input matrix `x` and desired values `d`.
//
//The arg `d` is desired value.
//
//`x` is input matrix.
//
//`muStart` is starting learning rate.
//
//`muEnd` is final learning rate.
//
//`steps` : how many learning rates should be tested between `muStart` and `muEnd`.
//
//`nTrain` is train to test ratio, typical value is 0.5. (that means 50% of data is used for training)
//
//`epochs` is number of training epochs, typical value is 1. This number describes how many times the training will be repeated.
//
//`criteria` is how should be measured the mean error. Available values are "MSE", "MAE" and "RMSE".
//
//`target_w` is target weights. If the slice is nil, the mean error is estimated from prediction error.
//
// If an slice is provided, the error between weights and `target_w` is used.
func ExploreLearning(af AdaptiveFilter, d []float64, x [][]float64, muStart, muEnd float64, steps int,
	nTrain float64, epochs int, criteria string, targetW []float64) ([]float64, []float64, error) {
	mus := misc.LinSpace(muStart, muEnd, steps)
	es := make([]float64, len(mus))
	zeros := make([]float64, int(float64(len(x))*nTrain))
	for i, mu := range mus {
		//init
		err := af.initWeights(nil, len(x[0]))
		if err != nil {
			return nil, nil, errors.Wrap(err, "failed to init weights at InitWights()")
		}
		err = af.SetStepSize(mu)
		if err != nil {
			return nil, nil, errors.Wrap(err, "failed to set step size at StetStepSize()")
		}
		//run
		_, e, _, err := PreTrainedRun(af, d, x, nTrain, epochs)
		if err != nil {
			return nil, nil, errors.Wrap(err, "failed to pre train at PreTrainedRun()")
		}
		es[i], err = misc.GetMeanError(e, zeros, criteria)
		//fmt.Println(es[i])
		if err != nil {
			return nil, nil, errors.Wrap(err, "failed to get mean error at GetMeanError()")
		}
	}
	return es, mus, nil
}

//FiltBase is base struct for adaptive filter structs.
//It puts together some functions used by all adaptive filters.
type filtBase struct {
	kind  string
	n     int
	muMin float64
	muMax float64
	mu    float64
	w     *mat.Dense
}

//NewFiltBase is constructor of base adaptive filter only for development.
func newFiltBase(n int, mu float64, w []float64) (AdaptiveFilter, error) {
	var err error
	p := new(filtBase)
	p.kind = "Base filter"
	p.n = n
	p.muMin = 0
	p.muMax = 1000
	p.mu, err = p.checkFloatParam(mu, p.muMin, p.muMax, "mu")
	if err != nil {
		return nil, err
	}
	err = p.initWeights(w, n)
	if err != nil {
		return nil, err
	}
	return p, nil
}

//initWeights initialises the adaptive weights of the filter.
//The arg `w` is initial weights of filter.
//Typical value is zeros with length `n`.
//If `w` is nil, this func initializes `w` as zeros.
//`n` is size of filter. Note that it is often mistaken for the sample length.
func (af *filtBase) initWeights(w []float64, n int) error {
	if n <= 0 {
		n = af.n
	}
	if w == nil {
		w = make([]float64, n)
	}
	if len(w) != n {
		return fmt.Errorf("the length of slice `w` and `n` must agree. len(w): %d, n: %d", len(w), n)
	}
	af.w = mat.NewDense(1, n, w)

	return nil
}

//Predict calculates the new estimated value `y` from input slice `x`.
func (af *filtBase) Predict(x []float64) (y float64) {
	y = floats.Dot(af.w.RawRowView(0), x)
	return y
}

//Adapt is just a method to satisfy the interface.
//It is used by overriding.
func (af *filtBase) Adapt(d float64, x []float64) {
	//TODO
}

//Run is just a method to satisfy the interface.
//It is used by overriding.
func (af *filtBase) Run(d []float64, x [][]float64) ([]float64, []float64, [][]float64, error) {
	//TODO
	//measure the data and check if the dimension agree
	N := len(x)
	if len(d) != N {
		return nil, nil, nil, errors.New("the length of slice d and x must agree")
	}
	af.n = len(x[0])

	y := make([]float64, N)
	e := make([]float64, N)
	w := make([]float64, af.n)
	wHist := make([][]float64, N)
	//adaptation loop
	for i := 0; i < N; i++ {
		w = af.w.RawRowView(0)
		y[i] = floats.Dot(w, x[i])
		e[i] = d[i] - y[i]
		copy(wHist[i], w)
	}
	return y, e, wHist, nil
}

//checkFloatParam check if the value of the given parameter
//is in the given range and a float.
func (af *filtBase) checkFloatParam(p, low, high float64, name string) (float64, error) {
	if low <= p && p <= high {
		return p, nil
	} else {
		err := fmt.Errorf("parameter %v is not in range <%v, %v>", name, low, high)
		return 0, err
	}
}

//checkIntParam check if the value of the given parameter
//is in the given range and a int.
func (af *filtBase) checkIntParam(p, low, high int, name string) (int, error) {
	if low <= p && p <= high {
		return p, nil
	} else {
		err := fmt.Errorf("parameter %v is not in range <%v, %v>", name, low, high)
		return 0, err
	}
}

//SetStepSize set a update step size mu.
func (af *filtBase) SetStepSize(mu float64) error {
	var err error
	af.mu, err = af.checkFloatParam(mu, af.muMin, af.muMax, "mu")
	if err != nil {
		return err
	}
	return nil
}

//GetParams returns the parameters at the time this func is called.
//parameters contains `n`: filter length, `mu`: filter update step size and `w`: filter weights.
func (af *filtBase) GetParams() (int, float64, []float64) {
	return af.n, af.mu, af.w.RawRowView(0)
}

//GetParams returns the name of ADF.
func (af *filtBase) GetKindName() string {
	return af.kind
}

func (af *filtBase) clone() AdaptiveFilter {
	altaf := *af
	return &altaf
}
\end{lstlisting}

\begin{lstlisting}[caption=lms.go,label=lms.go]
package adf

import (
	"errors"

	"gonum.org/v1/gonum/floats"
)

//FiltLMS is base struct for LMS filter.
//Use NewFiltLMS to make instance.
type FiltLMS struct {
	filtBase
	wHistory [][]float64
}

//NewFiltLMS is constructor of LMS filter.
//This func initialize filter length `n`, update step size `mu` and filter weight `w`.
func NewFiltLMS(n int, mu float64, w []float64) (AdaptiveFilter, error) {
	var err error
	p := new(FiltLMS)
	p.kind = "LMS filter"
	p.n = n
	p.muMin = 0
	p.muMax = 2
	p.mu, err = p.checkFloatParam(mu, p.muMin, p.muMax, "mu")
	if err != nil {
		return nil, err
	}
	err = p.initWeights(w, n)
	if err != nil {
		return nil, err
	}
	return p, nil
}

//Adapt calculates the error `e` between desired value `d` and estimated value `y`,
//and update filter weights according to error `e`.
func (af *FiltLMS) Adapt(d float64, x []float64) {
	w := af.w.RawRowView(0)
	y := floats.Dot(w, x)
	e := d - y
	for i := 0; i < len(x); i++ {
		w[i] += af.mu * e * x[i]
	}
}

//Run calculates the errors `e` between desired values `d` and estimated values `y` in a row,
//while updating filter weights according to error `e`.
func (af *FiltLMS) Run(d []float64, x [][]float64) (y []float64, e []float64, wHist [][]float64, err error) {
	//measure the data and check if the dimension agree
	N := len(x)
	if len(d) != N {
		return nil, nil, nil, errors.New("the length of slice d and x must agree")
	}
	af.n = len(x[0])
	af.wHistory = make([][]float64, N)
	for i := 0; i < N; i++ {
		af.wHistory[i] = make([]float64, af.n)
	}

	y = make([]float64, N)
	e = make([]float64, N)
	//adaptation loop
	for i := 0; i < N; i++ {
		w := af.w.RawRowView(0)
		copy(af.wHistory[i], w)
		y[i] = floats.Dot(w, x[i])
		e[i] = d[i] - y[i]
		for j := 0; j < af.n; j++ {
			w[j] += af.mu * e[i] * x[i][j]
		}
	}
	wHist = af.wHistory
	return y, e, wHist, nil
}

func (af *FiltLMS) clone() AdaptiveFilter {
	altaf := *af
	return &altaf
}
\end{lstlisting}

\begin{lstlisting}[caption=nlms.go,label=nlms.go]
package adf

import (
	"errors"

	"gonum.org/v1/gonum/floats"
)

//FiltNLMS is base struct for NLMS filter.
//Use NewFiltNLMS to make instance.
type FiltNLMS struct {
	filtBase
	eps      float64
	wHistory [][]float64
}

//NewFiltLMS is constructor of LMS filter.
//This func initialize filter length `n`, update step size `mu` and filter weight `w`.
func NewFiltNLMS(n int, mu float64, eps float64, w []float64) (AdaptiveFilter, error) {
	var err error
	p := new(FiltNLMS)
	p.kind = "NLMS filter"
	p.n = n
	p.muMin = 0
	p.muMax = 2
	p.mu, err = p.checkFloatParam(mu, p.muMin, p.muMax, "mu")
	if err != nil {
		return nil, err
	}
	p.eps, err = p.checkFloatParam(eps, 0, 1, "eps")
	if err != nil {
		return nil, err
	}
	err = p.initWeights(w, n)
	if err != nil {
		return nil, err
	}
	return p, nil
}

//Adapt calculates the error `e` between desired value `d` and estimated value `y`,
//and update filter weights according to error `e`.
func (af *FiltNLMS) Adapt(d float64, x []float64) {
	w := af.w.RawRowView(0)
	y := floats.Dot(w, x)
	e := d - y
	nu := af.mu / (af.eps + floats.Dot(x, x))
	for i := 0; i < len(x); i++ {
		w[i] += nu * e * x[i]
	}
}

//Run calculates the errors `e` between desired values `d` and estimated values `y` in a row,
//while updating filter weights according to error `e`.
func (af *FiltNLMS) Run(d []float64, x [][]float64) (y []float64, e []float64, wHist [][]float64, err error) {
	//measure the data and check if the dimension agree
	N := len(x)
	if len(d) != N {
		return nil, nil, nil, errors.New("the length of slice d and x must agree")
	}
	af.n = len(x[0])
	af.wHistory = make([][]float64, N)
	for i := 0; i < N; i++ {
		af.wHistory[i] = make([]float64, af.n)
	}

	y = make([]float64, N)
	e = make([]float64, N)
	w := af.w.RawRowView(0)
	//adaptation loop
	for i := 0; i < N; i++ {
		copy(af.wHistory[i], w)
		y[i] = floats.Dot(w, x[i])
		e[i] = d[i] - y[i]
		nu := af.mu / (af.eps + floats.Dot(x[i], x[i]))
		for j := 0; j < af.n; j++ {
			w[j] += nu * e[i] * x[i][j]
		}
	}
	wHist = af.wHistory
	return y, e, af.wHistory, nil
}

func (af *FiltNLMS) clone() AdaptiveFilter {
	altaf := *af
	return &altaf
}
\end{lstlisting}

\begin{lstlisting}[caption=rls.go,label=rls.go]
package adf

import (
	"errors"

	"gonum.org/v1/gonum/floats"
	"gonum.org/v1/gonum/mat"
)

//FiltRLS is base struct for RLS filter.
//Use NewFiltRLS to make instance.
type FiltRLS struct {
	filtBase
	wHist [][]float64
	eps   float64
	rMat  *mat.Dense
}

//NewFiltRLS is constructor of RLS filter.
//This func initialize filter length `n`, update step size `mu`, small enough value `eps`, and filter weight `w`.
func NewFiltRLS(n int, mu float64, eps float64, w []float64) (AdaptiveFilter, error) {
	var err error
	p := new(FiltRLS)
	p.kind = "RLS filter"
	p.n = n
	p.muMin = 0
	p.muMax = 1
	p.mu, err = p.checkFloatParam(mu, p.muMin, p.muMax, "mu")
	if err != nil {
		return nil, err
	}
	p.eps, err = p.checkFloatParam(mu, 0, 1, "eps")
	if err != nil {
		return nil, err
	}
	err = p.initWeights(w, n)
	if err != nil {
		return nil, err
	}
	var Rs = make([]float64, n*n)
	for i := 0; i < n; i++ {
		Rs[i*(n+1)] = 1 / eps
	}
	p.rMat = mat.NewDense(n, n, Rs)
	return p, nil
}

//Adapt calculates the error `e` between desired value `d` and estimated value `y`,
//and update filter weights according to error `e`.
func (af *FiltRLS) Adapt(d float64, x []float64) {
	w := af.w.RawRowView(0)
	R1 := mat.NewDense(af.n, af.n, nil)
	var R2 float64
	xVec := mat.NewDense(1, af.n, nil)
	aux1 := mat.NewDense(af.n, 1, nil)
	aux4 := mat.NewDense(1, af.n, nil)
	var aux2 float64
	aux3 := mat.NewDense(af.n, af.n, nil)
	dwT := mat.NewDense(af.n, 1, nil)

	y := floats.Dot(w, x)
	e := d - y

	xVec.SetRow(0, x)
	aux1.Mul(af.rMat, xVec.T())
	aux2 = floats.Dot(mat.Col(nil, 0, aux1), mat.Row(nil, 0, xVec))
	R1 = mat.DenseCopyOf(af.rMat)
	R1.Scale(aux2, R1)
	aux4.Mul(xVec, af.rMat)

	R2 = af.mu + mat.Dot(aux4.RowView(0), mat.DenseCopyOf(xVec.T()).ColView(0))
	R1.Scale(1/R2, R1)
	aux3.Sub(af.rMat, R1)
	af.rMat.Scale(1/af.mu, aux3)
	dwT.Mul(af.rMat, xVec.T())
	dwT.Scale(e, dwT)

	floats.Add(w, mat.Col(nil, 0, dwT))
}

//Run calculates the errors `e` between desired values `d` and estimated values `y` in a row,
//while updating filter weights according to error `e`.
func (af *FiltRLS) Run(d []float64, x [][]float64) (y []float64, e []float64, wHist [][]float64, err error) {
	//measure the data and check if the dimension agree
	N := len(x)
	if len(d) != N {
		return nil, nil, nil, errors.New("the length of slice d and x must agree")
	}
	af.n = len(x[0])
	af.wHist = make([][]float64, N)
	for i := 0; i < N; i++ {
		af.wHist[i] = make([]float64, af.n)
	}

	y = make([]float64, N)
	e = make([]float64, N)
	w := af.w.RawRowView(0)
	R1 := mat.NewDense(af.n, af.n, nil)
	var R2 float64
	xVec := mat.NewDense(1, af.n, nil)
	aux1 := mat.NewDense(af.n, 1, nil)
	aux4 := mat.NewDense(1, af.n, nil)
	//aux2 := mat.NewDense(af.n, af.n, nil)
	var aux2 float64
	aux3 := mat.NewDense(af.n, af.n, nil)
	dwT := mat.NewDense(af.n, 1, nil)
	//adaptation loop
	for i := 0; i < N; i++ {
		copy(af.wHist[i], w)
		y[i] = floats.Dot(w, x[i])
		e[i] = d[i] - y[i]
		xVec.SetRow(0, x[i])
		aux1.Mul(af.rMat, xVec.T())
		aux2 = floats.Dot(mat.Col(nil, 0, aux1), mat.Row(nil, 0, xVec))
		R1 = mat.DenseCopyOf(af.rMat)
		R1.Scale(aux2, R1)
		//R1.Product(aux1.T(), xVec.T())
		aux4.Mul(xVec, af.rMat)
		R2 = af.mu + mat.Dot(aux4.RowView(0), mat.DenseCopyOf(xVec.T()).ColView(0))
		//for j:=0;j<af.n;j++{
		//	floats.AddConst(af.rMat.RawRowView(j))
		//}
		R1.Scale(1/R2, R1)
		aux3.Sub(af.rMat, R1)
		af.rMat.Scale(1/af.mu, aux3)
		dwT.Mul(af.rMat, xVec.T())
		dwT.Scale(e[i], dwT)
		floats.Add(w, mat.Col(nil, 0, dwT))
	}
	wHist = af.wHist
	return y, e, af.wHist, nil
}

func (af *FiltRLS) clone() AdaptiveFilter {
	altaf := *af
	return &altaf
}
\end{lstlisting}

\begin{lstlisting}[caption=fdadf/base.go,label=fdadf/base.go]
package fdadf

import (
    "fmt"

    "github.com/pkg/errors"
    "github.com/tetsuzawa/go-adflib/misc"
    "gonum.org/v1/gonum/mat"
)

// FDAdaptiveFilter is the basic Frequency Domain Adaptive Filter interface type.
type FDAdaptiveFilter interface {
    initWeights(w interface{}, n int) error

    //Predict calculates the new estimated value `y` from input slice `x`.
    Predict(x []float64) (y []float64)

    //Adapt calculates the error `e` between desired value `d` and estimated value `y`,
    //and update filter weights according to error `e`.
    Adapt(d []float64, x []float64)

    //Run calculates the errors `e` between desired values `d` and estimated values `y` in a row,
    //while updating filter weights according to error `e`.
    Run(d [][]float64, x [][]float64) ([][]float64, [][]float64, [][]float64, error)
    checkFloatParam(p, low, high float64, name string) (float64, error)
    checkIntParam(p, low, high int, name string) (int, error)
    setStepSize(mu float64)

    //GetParams returns the parameters at the time this func is called.
    //parameters contains `n`: filter length, `mu`: filter update step size and `w`: filter weights.
    GetParams() (int, float64, []float64)

    //GetParams returns the name of FDADF.
    GetKindName() (kind string)
}

//Must checks whether err is nil or not. If err in not nil, this func causes panic.
func Must(af FDAdaptiveFilter, err error) FDAdaptiveFilter {
    if err != nil {
        panic(err)
    }
    return af
}

//PreTrainedRun use part of the data for few epochs of learning.
//The arg `d` is desired values. rows are
//`x` is input matrix. rows are samples and columns are features.
//`nTrain` is train to test ratio, typical value is 0.5. (that means 50% of data is used for training).
//`epochs` is number of training epochs, typical value is 1. This number describes how many times the training will be repeated.
func PreTrainedRun(af FDAdaptiveFilter, d [][]float64, x [][]float64, nTrain float64, epochs int) (y, e [][]float64, w [][]float64, err error) {
    var nTrainI = int(float64(len(d)) * nTrain)
    //train
    for i := 0; i < epochs; i++ {
        _, _, _, err = af.Run(d[:nTrainI], x[:nTrainI])
        if err != nil {
            return nil, nil, nil, err
        }
    }
    //run
    y, e, w, err = af.Run(d[:nTrainI], x[:nTrainI])
    if err != nil {
        return nil, nil, nil, err
    }
    return y, e, w, nil
}

//ExploreLearning searches the `mu` with the smallest error value from the input matrix `x` and desired values `d`.
//The arg `d` is desired value.
//`x` is input matrix.
//`muStart` is starting learning rate.
//`muEnd` is final learning rate.
//`steps` : how many learning rates should be tested between `muStart` and `muEnd`.
//`nTrain` is train to test ratio, typical value is 0.5. (that means 50% of data is used for training)
//`epochs` is number of training epochs, typical value is 1. This number describes how many times the training will be repeated.
//`criteria` is how should be measured the mean error. Available values are "MSE", "MAE" and "RMSE".
//`target_w` is target weights. If the slice is nil, the mean error is estimated from prediction error.
// If an slice is provided, the error between weights and `target_w` is used.
func ExploreLearning(af FDAdaptiveFilter, d [][]float64, x [][]float64, muStart, muEnd float64, steps int,
    nTrain float64, epochs int, criteria string, targetW []float64) ([]float64, []float64, error) {
    mus := misc.LinSpace(muStart, muEnd, steps)
    es := make([]float64, len(mus))
    zeros := make([]float64, int(float64(len(x))*nTrain))
    ee := make([]float64, int(float64(len(x))*nTrain))
    _, _, w := af.GetParams()
    for i, mu := range mus {
        //init
        err := af.initWeights("zeros", len(w))
        if err != nil {
            return nil, nil, errors.Wrap(err, "failed to init weights at InitWights()")
        }
        af.setStepSize(mu)
        //run
        _, e, _, err := PreTrainedRun(af, d, x, nTrain, epochs)
        if err != nil {
            return nil, nil, errors.Wrap(err, "failed to pre train at PreTrainedRun()")
        }
        for i, sl := range e {
            ee[i], err = misc.MSE(sl, make([]float64, len(sl)))
        }
        if err != nil {
            return nil, nil, errors.Wrap(err, "failed to find MSE of e at misc.MSE()")
        }
        es[i], err = misc.GetMeanError(ee, zeros, criteria)
        //fmt.Println(es[i])
        if err != nil {
            return nil, nil, errors.Wrap(err, "failed to get mean error at GetMeanError()")
        }
    }
    return es, mus, nil
}

//filtBase is base struct for frequency domain adaptive filter structs
//It puts together some functions used by all adaptive filters.
type filtBase struct {
    kind string
    n    int
    mu   float64
    w    *mat.Dense
}

//NewFiltBase is constructor of base frequency domain adaptive filter only for development.
func newFiltBase(n int, mu float64, w interface{}) (FDAdaptiveFilter, error) {
    var err error
    p := new(filtBase)
    p.n = n
    p.mu, err = p.checkFloatParam(mu, 0, 1000, "mu")
    if err != nil {
        return nil, err
    }
    err = p.initWeights(w, n)
    if err != nil {
        return nil, err
    }
    return p, nil
}

//initWeights initialises the adaptive weights of the filter.
//The arg `w` is initial weights of filter.
// Possible value "random":  create random weights with stddev 0.5 and mean is 0.
// "zeros": create zero value weights.
//`n` is size of filter. Note that it is often mistaken for the sample length.
func (af *filtBase) initWeights(w interface{}, n int) error {
    if n <= 0 {
        n = af.n
    }
    switch v := w.(type) {
    case string:
        if v == "random" {
            w := make([]float64, n)
            for i := 0; i < n; i++ {
                w[i] = misc.NewRandn(0.5, 0)
            }
            af.w = mat.NewDense(1, n, w)
        } else if v == "zeros" {
            w := make([]float64, n)
            af.w = mat.NewDense(1, n, w)
        } else {
            return errors.New("impossible to understand the w")
        }
    case []float64:
        if len(v) != n {
            return errors.New("length of w is different from n")
        }
        af.w = mat.NewDense(1, n, v)
    default:
        return errors.New(`args w must be "random" or "zeros" or []float64{...}`)
    }
    return nil
}

//Predict calculates the new output value `y` from input array `x`.
func (af *filtBase) Predict(x []float64) (y []float64) {
    //TODO
    //y = floats.Dot(af.w.RawRowView(0), x)
    //return y
    copy(y, x)
    return
}

//Adapt is just a method to satisfy the interface.
//It is used by overriding.
func (af *filtBase) Adapt(d []float64, x []float64) {
    //TODO
}

//Run is just a method to satisfy the interface.
//It is used by overriding.
func (af *filtBase) Run(d [][]float64, x [][]float64) ([][]float64, [][]float64, [][]float64, error) {
    //TODO
    return nil, nil, nil, nil
}

//checkFloatParam check if the value of the given parameter
//is in the given range and a float.
func (af *filtBase) checkFloatParam(p, low, high float64, name string) (float64, error) {
    if low <= p && p <= high {
        return p, nil
    } else {
        err := fmt.Errorf("parameter %v is not in range <%v, %v>", name, low, high)
        return 0, err
    }
}

//checkIntParam check if the value of the given parameter
//is in the given range and a int.
func (af *filtBase) checkIntParam(p, low, high int, name string) (int, error) {
    if low <= p && p <= high {
        return p, nil
    } else {
        err := fmt.Errorf("parameter %v is not in range <%v, %v>", name, low, high)
        return 0, err
    }
}

//setStepSize set a update step size mu.
func (af *filtBase) setStepSize(mu float64) {
    af.mu = mu
}

//GetParams returns the parameters at the time this func is called.
//parameters contains `n`: filter length, `mu`: filter update step size and `w`: filter weights.
func (af *filtBase) GetParams() (n int, mu float64, w []float64) {
    return af.n, af.mu, af.w.RawRowView(0)
}

//GetParams returns the kind name of ADF.
func (af *filtBase) GetKindName() (kind string) {
    return af.kind
}
\end{lstlisting}

\begin{lstlisting}[caption=fblms.go,label=fblms.go]
package fdadf

import (
	"math/cmplx"

	"github.com/mjibson/go-dsp/fft"
	"github.com/pkg/errors"
	"gonum.org/v1/gonum/mat"
)

//FiltFBLMS is base struct for FBLMS filter
//(Fast Block Least Mean Square filter).
//Use NewFiltFBLMS to make instance.
type FiltFBLMS struct {
	filtBase
	wHistory [][]float64
	xMem     *mat.Dense
}

//NewFiltFBLMS is constructor of FBLMS filter.
//This func initialize filter length `n`, update step size `mu` and filter weight `w`.
func NewFiltFBLMS(n int, mu float64, w interface{}) (FDAdaptiveFilter, error) {
	var err error
	p := new(FiltFBLMS)
	p.kind = "FBLMS filter"
	p.n = n
	p.mu, err = p.checkFloatParam(mu, 0, 1000, "mu")
	if err != nil {
		return nil, errors.Wrap(err, "Parameter error at checkFloatParam()")
	}
	err = p.initWeights(w, 2*n)
	if err != nil {
		return nil, err
	}
	p.xMem = mat.NewDense(1, n, make([]float64, n))
	return p, nil
}

//Adapt calculates the error `e` between desired value `d` and estimated value `y`,
//and update filter weights according to error `e`.
func (af *FiltFBLMS) Adapt(d []float64, x []float64) {
	zeros := make([]float64, af.n)
	Y := make([]complex128, 2*af.n)
	y := make([]float64, af.n)
	e := make([]float64, af.n)
	EU := make([]complex128, 2*af.n)

	w := af.w.RawRowView(0)
	// 1 compute the output of the filter for the block kM, ..., KM + M -1
	W := fft.FFT(float64sToComplex128s(append(w[:af.n], zeros...)))
	xSet := append(af.xMem.RawRowView(0), x...)
	U := fft.FFT(float64sToComplex128s(xSet))
	af.xMem.SetRow(0, x)
	for i := 0; i < 2*af.n; i++ {
		Y[i] = W[i] * U[i]
	}
	yc := fft.IFFT(Y)[af.n:]
	for i := 0; i < af.n; i++ {
		y[i] = real(yc[i])
		e[i] = d[i] - y[i]
	}

	// 2 compute the correlation vector
	aux1 := fft.FFT(float64sToComplex128s(append(zeros, e...)))
	aux2 := fft.FFT(float64sToComplex128s(xSet))
	for i := 0; i < 2*af.n; i++ {
		EU[i] = aux1[i] * cmplx.Conj(aux2[i])
	}
	phi := fft.IFFT(EU)[:af.n]

	// 3 update the parameters of the filter
	aux1 = fft.FFT(float64sToComplex128s(append(w[:af.n], zeros...)))
	aux2 = fft.FFT(append(phi, float64sToComplex128s(zeros)...))
	for i := 0; i < 2*af.n; i++ {
		W[i] = aux1[i] + complex(af.mu, 0)*aux2[i]
	}
	aux3 := fft.IFFT(W)
	for i := 0; i < 2*af.n; i++ {
		w[i] = real(aux3[i])
	}
}

//Predict calculates the new output value `y` from input array `x`.
func (af *FiltFBLMS) Predict(x []float64) (y []float64) {
	zeros := make([]float64, af.n)
	y = make([]float64, af.n)
	Y := make([]complex128, 2*af.n)
	W := fft.FFT(float64sToComplex128s(append(af.w.RawRowView(0)[:af.n], zeros...)))
	U := fft.FFT(float64sToComplex128s(append(af.xMem.RawRowView(0), x...)))
	for i := 0; i < 2*af.n; i++ {
		Y[i] = W[i] * U[i]
	}
	yc := fft.IFFT(Y)[af.n:]
	for i := 0; i < af.n; i++ {
		y[i] = real(yc[i])
	}
	return
}

//Run calculates the errors `e` between desired values `d` and estimated values `y` in a row,
//while updating filter weights according to error `e`.
//The arg `x`: rows are samples sets, columns are input values.
func (af *FiltFBLMS) Run(d [][]float64, x [][]float64) ([][]float64, [][]float64, [][]float64, error) {
	//measure the data and check if the dimension agree
	N := len(x)
	if len(d) != N {
		return nil, nil, nil, errors.New("the length of slice d and x must agree")
	}
	af.n = len(x[0])
	af.wHistory = make([][]float64, N)
	for i := range af.wHistory {
		af.wHistory[i] = make([]float64, af.n)
	}

	zeros := make([]float64, af.n)
	Y := make([]complex128, 2*af.n)
	y := make([][]float64, N)
	for i := range y {
		y[i] = make([]float64, af.n)
	}
	e := make([][]float64, N)
	for i := range e {
		e[i] = make([]float64, af.n)
	}

	EU := make([]complex128, 2*af.n)

	for k := 0; k < N; k++ {
		w := af.w.RawRowView(0)
		copy(af.wHistory[k], w)

		// 1 compute the output of the filter for the block kM, ..., KM + M -1
		W := fft.FFT(float64sToComplex128s(append(w[:af.n], zeros...)))
		xSet := append(af.xMem.RawRowView(0), x[k]...)
		U := fft.FFT(float64sToComplex128s(xSet))
		af.xMem.SetRow(0, x[k])

		for i := 0; i < 2*af.n; i++ {
			Y[i] = W[i] * U[i]
		}
		yc := fft.IFFT(Y)[af.n:]
		for i := 0; i < af.n; i++ {
			y[k][i] = real(yc[i])
			e[k][i] = x[k][i] - y[k][i]
		}

		// 2 compute the correlation vector
		aux1 := fft.FFT(float64sToComplex128s(append(zeros, e[k]...)))
		aux2 := fft.FFT(float64sToComplex128s(xSet))
		for i := 0; i < 2*af.n; i++ {
			EU[i] = aux1[i] * cmplx.Conj(aux2[i])
		}
		phi := fft.IFFT(EU)[:af.n]

		// 3 update the parameters of the filter
		aux1 = fft.FFT(float64sToComplex128s(append(w[:af.n], zeros...)))
		aux2 = fft.FFT(append(phi, float64sToComplex128s(zeros)...))
		for i := 0; i < 2*af.n; i++ {
			W[i] = aux1[i] + complex(af.mu, 0)*aux2[i]
		}
		aux3 := fft.IFFT(W)
		for i := 0; i < 2*af.n; i++ {
			w[i] = real(aux3[i])
		}
	}

	return y, e, af.wHistory, nil
}
\end{lstlisting}

\begin{lstlisting}[caption=misc.go,label=misc.go]
package misc

import (
	"errors"
	"math"
	"math/rand"

	"gonum.org/v1/gonum/floats"
)

func ElmAbs(fs []float64) []float64 {
	fsAbs := make([]float64, len(fs))
	for i, f := range fs {
		fsAbs[i] = math.Abs(f)
	}
	return fsAbs
}

func LogSE(x1, x2 []float64) ([]float64, error) {
	e, err := GetValidError(x1, x2)
	if err != nil {
		return nil, err
	}
	for i := 0; i < len(e); i++ {
		e[i] = 10 * math.Log10(math.Pow(e[i], 2))
	}
	return e, nil

}

func MAE(x1, x2 []float64) (float64, error) {
	e, err := GetValidError(x1, x2)
	if err != nil {
		return 0, err
	}
	return floats.Sum(ElmAbs(e)) / float64((len(e))), nil
}

func MSE(x1, x2 []float64) (float64, error) {
	e, err := GetValidError(x1, x2)
	if err != nil {
		return 0, err
	}
	return floats.Dot(e, e) / float64(len(e)), nil
}

func RMSE(x1, x2 []float64) (float64, error) {
	e, err := GetValidError(x1, x2)
	if err != nil {
		return 0, err
	}
	return math.Sqrt(floats.Dot(e, e)) / float64(len(e)), nil
}

func GetValidError(x1, x2 []float64) ([]float64, error) {
	if len(x1) != len(x2) {
		err := errors.New("length of two slices is different")
		return nil, err
	}
	floats.Sub(x1, x2)
	e := x1
	return e, nil
}

func GetMeanError(x1, x2 []float64, fn string) (float64, error) {
	switch fn {
	case "MAE":
		return MAE(x1, x2)
	case "MSE":
		return MSE(x1, x2)
	case "RMSE":
		return RMSE(x1, x2)
	default:
		err := errors.New(`The provided error function (fn) is not known.
								Use "MAE", "MSE" or "RMSE"`)
		return 0, err
	}
}

// NewRandn returns random value. stddev 0.5, mean 0.
func NewRandn(stddev, mean float64) float64 {
	return rand.NormFloat64()*stddev + mean
}

func LinSpace(start, end float64, n int) []float64 {
	res := make([]float64, n)
	if n == 1 {
		res[0] = end
		return res
	}
	delta := (end - start) / (float64(n) - 1)
	for i := 0; i < n; i++ {
		res[i] = start + (delta * float64(i))
	}
	return res
}

func Floor(fs [][]float64) []float64 {
	var fs1d = make([]float64, len(fs)*len(fs[0]))
	for i, sl := range fs {
		for j, v := range sl {
			fs1d[len(fs[0])*i+j] = v
		}
	}
	return fs1d
}

func NewRandSlice(n int) []float64 {
	rs := make([]float64, n)
	for i := 0; i < n; i++ {
		rs[i] = rand.Float64()
	}
	return rs
}

func NewNormRandSlice(n int) []float64 {
	rs := make([]float64, n)
	for i := 0; i < n; i++ {
		rs[i] = rand.NormFloat64()
	}
	return rs
}

// NewRand2dSlice make 2d slice.
// the arg n is sample number and m is number of signals.
func NewRand2dSlice(n, m int) [][]float64 {
	rs2 := make([][]float64, m)
	for j := 0; j < m; j++ {
		rs2[j] = NewRandSlice(n)
	}
	return rs2
}

// NewRandNorm2dSlice make 2d slice.
// the arg n is sample number and m is number of signals.
func NewNormRand2dSlice(n, m int) [][]float64 {
	rs2 := make([][]float64, m)
	for j := 0; j < m; j++ {
		rs2[j] = NewNormRandSlice(n)
	}
	return rs2
}

func Unset(s []float64, i int) []float64 {
	if i >= len(s) {
		return s
	}
	return append(s[:i], s[i+1:]...)
}
\end{lstlisting}  

\end{document}
