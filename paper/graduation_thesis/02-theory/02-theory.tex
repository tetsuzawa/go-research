\chapter{理論}\label{theory}

\section{適応フィルタの基礎}\label{basis-adf}

\subsection{ウィナーフィルタ}\label{wiener}

\subsubsection{ウィナーフィルタの構造}\label{wiener-structure}

ウィナーフィルタは線形MMSE法(最小平均二乗誤差法)を定常な時系列に適用した特殊型であり, 後述の適応アルゴリズムの基礎となる. 

ウィナーフィルタのブロック図を\figref{block_jp}に示す. 

\begin{figure}[H]
\centering
\includegraphics[width=10cm]{figures/block_jp.png}
\caption{ウィナーフィルタのブロック図}
\label{block_jp}
\end{figure}


ウィナーフィルタの入出力は次式で表される. 

\begin{equation}
y_k = \bm{w}^H \bm{x}_k
\label{simple_y}
\end{equation}

ここで, \( \bm{w}^H\) は次式で与えられるフィルタ係数ベクトルである. 

\begin{equation}
\bm{w}^H = [w_1^*, \cdots, w_{K}^*]^T
\end{equation}

一方, フィルタ入力は次式のような時系列 \({x_k}\)
を逆順に並べたものとなる. 

\begin{equation}
\bm{x}_k = [x_k, x_{k-1}, \cdots, x_{k-K+1}]^T
\end{equation}

これは, \equref{simple_y}により, 次式のような時間領域の畳み込み演算を表すためである. 

\begin{equation}
y_k = \sum_{i=1}^K w_i^* x_{k-i+1}
\end{equation}

これにより, 式はフィルタ係数\({w_i^*}\)を持つ時間領域のFIRフィルタを表すことになる. 入力時系列\({x_k}\)については, 平均値
\(E[x_k]=0\) の実数あるいは複素数を仮定している. 

ウィナーフィルタで推定すべき値は, 実数あるいは複素数のスカラー量\(d_k\)となる. \(d_k\)は望みの応答と呼ばれ, フィルタ出力\(y_k\)により, 望みの応答を推定する. すなわち, \(\hat{d}_k = y_k\)となる. フィルタの推定過程では, 次式で定義される推定誤差を, 次式で述べる規範のもとに最小化する. 

\begin{equation}
\epsilon_k := d_k - y_k
\end{equation}

上述のように, 適応フィルタの目的は, 入力信号\(\bm{x}_k\)から\(d_k\)を推定することである. 
これを達成するための過程として, フィルタ\(\bm{w}\)の係数を決定する学習過程と, \equref{simple_y}により\(\hat{d}_k\)を推定するフィルタリング過程とに分けられる. 
フィルタの学習過程では, 信号\(d_k\)を教師として与え, \(d_k\)と\(\bm{x}_k\)の関係を表す係数\(\bm{w}\)を学習する. 
一方, フィルタリング過程では, 信号\(d_k\)が未知の場合について, 観測値\(\bm{x}_k\)と学習済みのフィルタ係数\(\bm{w}\)から, 信号の推定値\(\hat{d_k} (= y_k)\)を得る. 学習過程で望みの応答\(d_k\)をどうやって与えるかは, 応用に依存する\cite{signal_processing_for_array}. 

\subsubsection{ウィナーフィルタの導出}\label{wiener-introduction}

ウィナーフィルタでは, 観測値\(\bm{x}_k\)から信号\(d_k\)を推定する. フィルタの学習過程では, 次式の二乗平均誤差をコスト関数として最小化する. 

\begin{equation}
\begin{split}
J &= E[|\epsilon_k|^2] \\
  &= E[ (d_k - \bm{w}^H \bm{x}_k) (d_k - \bm{w}^H \bm{x}_k)^H ] \\
  &= E[ d_k d_k^H] - \bm{w}^H E[\bm{x}_k d_k^H] - E[d_k \bm{x}_k^H ] \bm{w} 
\end{split}
\label{equ:J_epsilon}
\end{equation}

ここで次式を定義する. 

\begin{equation}
\sigma_d^2 := E[d_k d_k^H], \bm{r}_{xd} := \bm{x}_k d_k^H, \bm{r}_{dx} := d_k \bm{x}_k^H, \bm{R}_x := E[\bm{x}_k \bm{x}_k^H]
\end{equation}

\(\bm{r}_{xd}\)および\(\bm{r}_{dx}\)は, 入力ベクトル\(\bm{x}_k\)と望みの応答\(d_k\)との相互相関ベクトル, \(\bm{R}_u\)は, \(\bm{x}_k\)の自己相関行列である. これらを用いて\equref{equ:J_epsilon}は次式のように書き直せる. 

\begin{equation}
J =\sigma_d^2 + - \bm{w}^H \bm{r}_{xd} -\bm{r}_{dx} \bm{w} + \bm{w}^H \bm{R}_x \bm{w}
\label{equ:J_sigma}
\end{equation}

コスト関数を\(\bm{w}^*\)について偏微分すると次式のようになる. 

\begin{equation}
\frac{\partial J}{\partial \bm{w}^*} = - \bm{r}_{xd} + \bm{R}_x \bm{w}
\bm{w}
\label{equ:J_partial}
\end{equation}

\equref{equ:J_partial}を\(\bm{0}_{K \times 1}\)とおくと, 次式を得る. 

\begin{equation}
\bm{R}_x \bm{w} = \bm{r}_xd
\label{equ:Rwr}
\end{equation}

\equref{equ:Rwr}は, 正規方程式あるいはウィナー・ホッフ方程式と呼ばれる. \equref{equ:Rwr}を解くことにより最適フィルタは, 次式のように求まる\cite{signal_processing_for_array}. 

\begin{equation}
\hat{\bm{w}}_{WF} = \bm{R}_u^{-1} \bm{r}_{xd}
\end{equation}

\
\subsection{適応フィルタにおける最小二乗法(LS法)}\label{ls}

適応フィルタにおけるLS法は, ウィナーフィルタの近似解を有限のサンプルから求めるものであり, \ref{rls}節で述べるRLS法の基礎となっている. 

適応フィルタにおけるLS法では, 次式に示す誤差の二乗和が最小化される. 

\begin{equation}
J = \sum_{k=1}^{L_s} |\epsilon_k|^2 = \sum_{k=1}^{L_s} |d_k - \bm{w}^H \bm{x}_k|^2
\end{equation}

ここで, \(L_s\)はサンプル数である. ウィナーフィルタの導出過程における期待値をサンプル平均に置き換え, 同様に正規方程式を求めると次式が導かれる. 

\begin{equation}
\hat{\bm{R}}_x \bm{w} = \hat{\bm{r}}_xd
\end{equation}

これより, LS法における最適解は, 次式のようになる\cite{signal_processing_for_array}. 

\begin{equation}
\hat{\bm{w}}_{LS} = \hat{\bm{R}}_x^{-1} \hat{\bm{r}}_xd
\end{equation}

\subsection{最急降下法}\label{sd}

\ref{wiener}節で述べたウィナーフィルタでは, 二乗平均誤差を最小とするフィルタ\(\bm{w}\)を, 正規方程式の解として求めた. ここでは, この解を反復法を用いて逐次的に求める. 

反復法では, フィルタ係数の初期値を適当に定め, コスト関数\(J(\bm{w})\)の最小点を目指して, フィルタ係数を少しずつ変化させていく. \equref{equ:J_sigma}で示したコスト関数の場合, \(J(\bm{w})\)は\(w\)についての2次関数となり, 下に凸の誤差特性曲面となる. 最急降下法では, \(k-1\)回目の反復におけるフィルタ係数を\(\bm{w}_{k-1}\)とした場合, \(\bm{w} = \bm{w}_{k-1}\)での誤差特性曲面の勾配を推定し, この勾配と逆の方向にフィルタ係数を変化させる. \(\bm{w} = \bm{w}_{k-1}\)における勾配ベクトルは\equref{equ:J_partial}から, 次式のように求まる. 

\begin{equation}
\nabla J(\bm{w}_{k-1}) := \frac{\partial J(\bm{w})}{\partial \bm{w}^*} |_{\bm{w} = \bm{w}_{k-1}} = - \bm{r}_{xd} + \bm{R}_x \bm{w}_{k-1}
\end{equation}

次回の反復で勾配とは逆方向に係数を変化させるため, フィルタ係数ベクトルの変化分は次式のようになる. 

\begin{equation}
\Delta \bm{w} = - \mu \nabla J(\bm{w}_{k-1})
\end{equation}

\(\mu\)は1回の更新量を決定する性の定数であり, ステップサイズパラメータと呼ばれる. これを用いて, フィルタの更新式は, 次式のようになる. 

\begin{equation}
\begin{split}
\bm{w}_k &= \bm{w}_{k-1} + \Delta \bm{w} \\
         &= \bm{w}_{k-1} + \mu (\bm{r}_{xd} - \bm{R}_x \bm{w}_{k-1})
\end{split}
\label{equ:w_delta}
\end{equation}

また, \equref{equ:w_delta}は次式のように書き直すことができる. 

\begin{equation}
\begin{split}
\bm{w}_k &= \bm{w}_{k-1} + \mu (E[\bm{x}_k d_k^*] - E[\bm{x}_k \bm{x}_k^H] \bm{w}_{k-1}) \\
         &= \bm{w}_{k-1} + \mu E[\bm{x}_k (d_k^* - \bm{x}_k^H \bm{w}_{k-1})]
\end{split}
\label{equ:w_complex}
\end{equation}

ここで, 次式の事前推定誤差を定義する. 

\begin{equation}
e_k := d_k - \bm{w}_{k-1}^H \bm{x}_k
\label{equ:prior_estimation_error}
\end{equation}

\equref{equ:w_complex}に\equref{equ:prior_estimation_error}を代入して, 次式を得る\cite{signal_processing_for_array}. 

\begin{equation}
\bm{w}_k = \bm{w}_{k-1} + \mu E[\bm{x}_k e_k^*]
\label{equ:w_update_sd}
\end{equation}


\section{代表的な適応アルゴリズム}\label{main-algo}

\subsection{最小二乗法平均法(LMS法)}\label{lms}

リアルタイムシステムを構築する場合などは, メモリや演算量に制約があることがある. 最急降下法の更新式はシンプルだが, 期待値演算をサンプル平均で実装することになり, メモリや演算量を必要とする. この点を改良したのが最小二乗平均法(LMS法)である. LMS法では, \equref{equ:w_update_sd}における期待値演算を用いた勾配の推定を, 次式のように, 瞬時値の推定値に置き換える. 

\begin{equation}
\bm{w}_k = \bm{w}_{k-1} + \mu \bm{x}_k e_k^*
\label{equ:w_update_lms_simple}
\end{equation}

また, \equref{equ:w_update_lms_simple}を入力のパワーで正規化したものは, 学習同定法(NLMS法)と呼ばれる. NLMS法は, 次式の拘束付き最適化問題から導くことができる. 

\begin{equation}
\begin{split}
min ||\bm{w}_k - \bm{w}_{k-1}||^2 \\
& \mbox{subject \ to} \ \bm{w}_k^H \bm{x}_k = d_k
\end{split}
\end{equation}

この最適化問題は, ラグランジュの未定乗数法を用いて解くことができる. ラグランジュの未定乗数法では, 次式のコスト関数を最小化する. 

\begin{equation}
\begin{split}
J &= ||\bm{w}_k - \bm{w}_{k-1}||^2  + Re(\lambda (d_k - \bm{w}_k^H \bm{x}_k)) \\
  &= (\bm{w}_k - \bm{w}_{k-1})^H (\bm{w}_k - \bm{w}_{k-1}) + \lambda (d_k - \bm{w}_k^H \bm{x}_k) + (d_k - \bm{w}_k^H \bm{x}_k))^H \lambda^H
\end{split}
\end{equation}

コスト関数\(J\)を\(\bm{w}_k^*\)について偏微分すると

\begin{equation}
\frac{\partial J}{\partial \bm{w}_k^*} = \bm{w}_k - \bm{w}_{k-1} - \lambda \bm{x}_k
\end{equation}

これを\(\bm{0}_{K \times 1}\)とすることにより, 次式を得る. 

\begin{equation}
\bm{w}_k - \bm{w}_{k-1} = \lambda \bm{x}_k
\label{equ:w-w_k}
\end{equation}

\equref{equ:w-w_k}の左から\(\bm{x}_k^H\)を乗じ, \(\lambda\)について解くと

\begin{equation}
\lambda = \frac{1}{\bm{x}_k^H \bm{x}_k} \bm{x}_k^H (\bm{w}_k - \bm{w}_{k-1})
\end{equation}

これに拘束条件(\equref{equ:w_complex})および事前推定誤差(\equref{equ:prior_estimation_error})

\begin{equation}
\lambda = \frac{1}{ ||\bm{x}_k||^2 } (d_k^* - \bm{x}_k^H \bm{w}_{k-1}) = \frac{1}{||\bm{x}_k||^2} e_k^*
\label{equ:lambda}
\end{equation}

\equref{equ:lambda}を再び\equref{equ:w-w_k}に代入することにより, フィルタの変化量\(\Delta \bm{w}\)は次式のように求まる. 

\begin{equation}
\Delta \bm{w} = \frac{1}{||\bm{x}_k||^2} \bm{x}_k e_k^*
\end{equation}

以上より, NLMS方のフィルタベクトル更新式は次式のようになる. 

\begin{equation}
\bm{w}_k = \bm{w}_{k-1} + \lambda \frac{1}{||\bm{x}_k||^2} \bm{x}_k e_k^*
\end{equation}

実際の運用では, 入力信号のパワー\(||\bm{x}_k||^2\)が非常に小さいときに更新式が不安定になるのを防ぐため, \(
1 / ||{}\bm{x}_k||^2 \)
の分母に小さい正の定数 \(\alpha\) を加えた, 次式を用いる\cite{signal_processing_for_array}. 

\begin{equation}
\bm{w}_k = \bm{w}_{k-1} + \lambda \frac{1}{\alpha + ||\bm{x}_k||^2} \bm{x}_k e_k^*
\end{equation}

\subsection{アフィン射影法(APA法)}\label{apa}

アフィン射影法は, NLMS法には, NLMS法における拘束付き最適化問題の拘束条件を次式のように複数に拡張することにより導くことができる. 

\begin{equation}
\begin{split}
min ||\bm{w}_k - \bm{w}_{k-1}||^2 \\ 
\mbox{subject \ to} \ \bm{w}_k^H \bm{X}_k = d_k
\end{split}
\end{equation}


ここで, \(\bm{X_k}\)(\(K \times L_s\)の行列)および\(\bm{d_k}\)(\(1 \times L_s\)の行ベクトル)は次式で定義される. 


\begin{equation}
\bm{d}_k := [d_{k-L_s+1}, \cdots, d_k]
\end{equation}

\begin{equation}
\bm{X}_k := [\bm{x}_{k-L_s+1}, \cdots, \bm{x}_k]
\end{equation}

\begin{equation}
\bm{x}_k := [\bm{x}_{k}, \cdots, \bm{x}_{k-L_s+1}]^T
\end{equation}

拘束条件は, 次式の\(L_s\)個の拘束条件を行列・ベクトル形式で表したものである. 

NLMS法と同様にラグランジュの未定乗数法を用いた, 最適解を導出により, 次式のフィルタ更新式が得られる\cite{signal_processing_for_array}. 

\begin{equation}
\bm{w}_k = \bm{w}_{k-1} + \mu \bm{X}_k ( \alpha \bm{I} + \bm{X}_k^H \bm{X} )^{-1} e_k^H
\end{equation}



\subsection{再帰最小二乗法(RLS法)}\label{rls}

再帰最小二乗法(RLS法)は. LS法の解を再帰的に求める手法である. 

サンプル平均による自己相関行列および相互相関ベクトルの推定値\(\hat{\bm{R}}_x\)および\(\hat{\bm{r}}_{xd}\)を次式に示す. 

\begin{equation}
\hat{\bm{R}}_k := \sum_{i=1}^k \bm{x}_i \bm{x}_i^H = \bm{X}_{1:k} \bm{X}_{1:k}^H
\end{equation}

\begin{equation}
\hat{\bm{r}}_k := \sum_{i=1}^k \bm{x}_i d_i^* = \bm{X}_{1:k} \bm{d}_{1:k}^H
\end{equation}

ここで, \(\bm{d}_{1:k}\)および\(\bm{X}_{1:k}\)

\begin{equation}
\bm{d}_{1:k} := [d_1, \cdots, d_k]
\end{equation}

\begin{equation}
\bm{X}_{1:k} := [\bm{x}_1, \cdots, \bm{x}_k]
\end{equation}

\begin{equation}
\bm{x}_{k} := [x_k, \cdots, u_{k-K+1}]^T
\end{equation}

時刻kにおける\(\hat{\bm{R}}_x\)および\(\hat{\bm{r}}_x\)は, 一時刻前\((k-1)\)の値を用いて, 次式のように再帰的に表すことができる. 


\begin{equation}
\hat{\bm{R}}_k = \hat{\bm{R}}_{k-1} + \bm{x}_k \bm{x}_k^H
\end{equation}


\begin{equation}
\hat{\bm{r}}_k = \hat{\bm{r}}_{k-1} + \bm{x}_k d_k^*
\label{equ:hat_r}
\end{equation}

逆行列の補助定理を用いると, 自己相関行列の逆行列\(P_k := \hat{\bm{R}}_k^{-1}\)も, 次式のように再帰的に求めることができる. 


\begin{equation}
\bm{P}_k = \bm{P}_{k-1} - \frac{\bm{P}_{k-1} \bm{x}_k \bm{x}_k^H \bm{P}_{k-1}}{1 + \bm{x}_k^H \bm{P}_{k-1} \bm{x}_k} 
\label{equ:P_k_inv_R}
\end{equation}

\equref{equ:hat_r}, \equref{equ:P_k_inv_R}
を用いて, 求めるべきフィルタ系ルウは, 次式のようになる. 

\begin{equation}
\bm{w}_k = \bm{P}_k \hat{\bm{r}}_k
\label{equ:w_simple_rls}
\end{equation}


ここで, 更新式における演算を簡略化するため, 次式のゲインベクトルを定義する. 

\begin{equation}
\bm{g}_k := \frac{\bm{P}_{k-1} \bm{x}_k}{1 + \bm{x}_k^H \bm{P}_{k-1} \bm{x}_k}
\label{equ:g_k_complex}
\end{equation}

これを用いて\equref{equ:P_k_inv_R}を書き直すと


\begin{equation}
\bm{P}_k = \bm{P}_{k-1} - \bm{g}_k \bm{x}_k^H \bm{P}_{k-1} = (\bm{I} - \bm{g}_k \bm{x}_k^H) \bm{P}_{k-1}
\label{equ:P_k_flat}
\end{equation}

一方, \equref{equ:g_k_complex}から, 次式を得る. 

\begin{equation}
\bm{g}_k = (\bm{I} - \bm{g}_k \bm{x}_k^H) \bm{P}_{k-1} \bm{x}_k
\label{equ:g_k_flat}
\end{equation}


\equref{equ:P_k_flat}, \equref{equ:g_k_flat}を用いて, ゲインベクトルは次式のように書き直せる. 


\begin{equation}
\bm{g}_k = \bm{P}_k \bm{x}_k
\label{equ:g_k_simple}
\end{equation}


\equref{equ:P_k_flat}および\equref{equ:g_k_simple}を\equref{equ:w_simple_rls}に代入することにより, フィルタ係数ベクトル\(\bm{w}_k\)の更新式は, 次式のように求まる. 

\begin{equation}
\begin{split}
\bm{w}_k &= \bm{P}_k (\hat{\bm{r}}_{k-1} + \bm{x}_k d_k^*) \\
         &= \bm{w}_{k-1} + \bm{g}_k (d_k^* - \bm{x}_k^H \bm{w}_{k-1})
\end{split}
\label{equ:w_k_complex}
\end{equation}


最後に, \equref{equ:w_k_complex}に\equref{equ:g_k_flat}を代入して, 次式の更新式を得る\cite{signal_processing_for_array}. 

\begin{equation}
\bm{w}_k = \bm{w}_{k-1} + \bm{g}_k e_k^*
\end{equation}


\subsection{適応アルゴリズムの比較}\label{algo-compare}

\tabref{tab:formula}は代表的な適応アルゴリズムの関係をニュートン法を使用して整理したものである. 

\begin{table}[H]
  \centering
  \caption{代表的な適応アルゴリズムの係数更新式}
  \label{tab:formula}
  \begin{tabular}{|c|c|}
  \hline
       & \(\Delta \bm{w}\)                                                                                             \\ \hline
  LMS  & \( \mu \bm{x}_k \bm{\epsilon}_k \)                                                                         \\ \hline
  NLMS & \( \mu \left( \alpha \bm{I} + \bm{x}_k \bm{x}_k^H \right)^{-1} \bm{x}_k \bm{\epsilon}_k \)        \\ \hline
  AP   & \(\mu \bm{X}_k \left( \alpha \bm{I} + \bm{X}_k^H \bm{X}_k \right)^{-1} \bm{\epsilon}_k\)          \\ \hline
  RLS  & \( \mu \bm{X}_k \left( \alpha \bm{I} + \bm{X}_{1:k}^H \bm{X}_{1:k} \right)^{-1} \bm{\epsilon}_k\) \\ \hline
  \end{tabular}
\end{table}

\tabref{tab:formula}のアルゴリズムは下の手法ほど予測に用いるサンプル数が増加する. これに伴い, 一般に収束速度も速くなる. 一方, この代償として計算量が増大する. したがって, 実装するハードウェアの規模や, 求められる収束速度などを考慮して, アルゴリズムを選択する必要がある\cite{adaptive_filters}. 

\newpage

\
\section{適応フィルタを使用した雑音低減の手法}\label{adf-noise-reduction}

\
\subsection{アクティブノイズコントロール(ANC)}\label{anc}

アクティブノイズコントロール(ANC)とは, マイクやスピーカ等の機器を利用し, 対象とする騒音と逆位相の音を重ね合わせることで音を低減する手法である. 

ANCのブロック図を\figref{fig:anc_block}に示す. 

\begin{figure}[H]
\centering
\includegraphics[width=9cm]{figures/anc.png}
\caption{ドローンの駆動音に対するANCの想定ブロック図}
\label{fig:anc_block}
\end{figure}

ドローンに駆動音に対するANCは\figref{fig:anc_block}における参照信号\(\bm{x}_k\)と誤差信号\(\epsilon_k\)からADF(\(\bm{w}\))で未知システムを推定することで次サンプルの信号を予測し, スピーカーなどの音響機器からフィルタ出力\(y_k\)を逆位相で発生させて打ち消す動作を繰り返す. 

したがって, 適応アルゴリズムの予測性能が駆動音抑圧の効果を決定づける要因となる. 

\subsection{自動等化器}\label{automatic-equalizer}

自動等化器はアクティブノイズコントロールとは異なり, 計算機内部で信号の処理を行う手法である. 
一般に, 適応フィルタに対する入力信号は長時間で平均をとると, ドローンの駆動音に対して音声のエネルギーが小さい. 自動等化器では, フィルタの更新係数を小さくすることでフィルタ係数がドローンの駆動音のみに適応することを利用し, フィルタ出力\(y_k\)との誤差\(e_k\)を目的信号の予測値として取り出す手法である. 

自動等化器のブロック図を\figref{equ:automatic_equlizer_block}に示す. 

\begin{figure}[H]
\centering
\includegraphics[width=13cm]{figures/automatic_equalizer_block.png}
\caption{ドローンの駆動音に対する自動等化器の想定ブロック図}
\label{equ:automatic_equlizer_block}
\end{figure}
