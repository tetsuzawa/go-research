\begin{jabstract}
本研究では, ドローンの応用分野の1つとして音声収録に着目し, 搭載されたマイクと小型の計算機を使用して収録した音声信号からドローンの駆動音を取り除く手法について検討した. 

駆動音の低減法を検討するにあたり, ドローン・バイノーラルマイクの組み立て, Raspberry
Piのセットアップなどハードウェアの準備を行った.  Raspberry
PiにはAD変換が搭載されていないため, AD変換器を内蔵した拡張ボードとしてPumpkin
Piを使用し, 初期設定を行った. 

ソフトウェアに関しては, まずPythonを使用して, 静的なFIRフィルタ・適応フィルタ(ADF)の評価を行った. 次にADFのライブラリをGo言語で作成し, インターネットにて公開した. また, Python・Go言語を使用して波形表示や音声編集用のソフトウェアを制作した. 

次にドローンの駆動音に対する各適応アルゴリズムの有効性を検証するために, 駆動音のサンプル収音を行い, ADFの収束特性を試験した. 結果的にNLMS・APアルゴリズムに比べてRLSアルゴリズムの収束誤差は20dB小さいが, 収束速度が4167[ms]と遅くリアルタイム処理には向かないことが判明した. 

制作したADFライブラリのベンチマークを取ると, Raspberry Piを計算媒体とした場合, 一番高速なNLMSアルゴリズムでも計算速度1.9[ms/op]が遅く, アクティブノイズコントロールの実装は難しいことが判明した. 

最後に, 実際の使用の際に入力される音響信号を模擬し, ADFによる雑音低減の効果を検討した. 音声の信号を抽出するのに成功したのはSN比0dBのRLSアルゴリズムと一部のフィルタ長のAPアルゴリズムのみであった. 

以上より本研究で検討したRaspberry Piを有する構成での雑音低減の実現可能性は低いことを結論付ける. 
\end{jabstract}
