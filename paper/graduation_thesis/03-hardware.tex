\chapter{ハードウェアの製作}\label{hardware}

この章では本研究で使用するドローン, マイク, Raspberry
Piといったハードウェアの構成と製作について述べる. 

\
\section{ドローンについて}\label{about-drone}

ドローンはLynxmotion株式会社の組み立て式ドローン (HQuad500) を使用した. このドローンは機体にカーボンファイバーを採用しており, 軽量, 高強度, 高剛性を兼ね備えている. また, 拡張性が高く, 容易に機能の追加が可能で様々な研究用途に適した製品である. 

\
\subsection{使用機器}\label{used-equipments-drone}

使用機器を以下に示す. 

\begin{enumerate}
\renewcommand{\labelenumi}{(\arabic{enumi})}
\item
  ドローンの機体 HQuad500 Hardware kit Lynxmotion株式会社
  \begin{figure}[H]
  \centering
  \includegraphics[width=6cm]{figures/hquad500_hardware.jpg}
  \caption{ドローンの機体}
  \label{fig:hquad500}
  \end{figure}


\item
  ESC (Electronic Speed Controller) 12A ESC (SimonK) Lynxmotion株式会社
  \begin{figure}[H]
  \centering
  \includegraphics[width=6cm]{figures/esc.jpg}
  \caption{ESC}
  \label{fig:esc}
  \end{figure}

\item
  ブラシレスモータ Brushless Motor 28x30 1000kv Lynxmotion株式会社
  \begin{figure}[H]
  \centering
  \includegraphics[width=6cm]{figures/brushless_motor.jpg}
  \caption{ブラシレスモータ}
  \label{fig:motor}
  \end{figure}

\item
  フライトコントローラ Quadrino Nano Lynxmotion株式会社
  \begin{figure}[H]
  \centering
  \includegraphics[width=6cm]{figures/quadrino_nano.jpg}
  \caption{フライトコントローラ}
  \label{fig:quadrino_nano}
  \end{figure}
  
\newpage

\item
  リポバッテリー充電器 18W LiPo Battery Charger Lynxmotion株式会社
  \begin{figure}[H]
  \centering
  \includegraphics[width=6cm]{figures/lipo_charger.jpg}
  \caption{リポバッテリー充電器}
  \label{fig:lipo_charger}
  \end{figure}

\item
  リポバッテリー 11.1V (3S), 3500mAh 30C LiPo Battery Pack Lynxmotion株式会社
  \begin{figure}[H]
  \centering
  \includegraphics[width=6cm]{figures/lipo_battery.jpg}
  \caption{リポバッテリー}
  \label{fig:lipo_battery}
  \end{figure}

\item
  ラジオレシーバ R9DS 10 channels 2.4GHz DSSS FHSS Receiver
  RadioLink株式会社
  \begin{figure}[H]
  \centering
  \includegraphics[width=6cm]{figures/r9ds.jpg}
  \caption{ラジオレシーバ}
  \label{fig:radio_reciever}
  \end{figure}

\item
  トランスミッタ AT9S 2.4GHz 10CH transmitter RadioLink株式会社
  \begin{figure}[H]
  \centering
  \includegraphics[width=6cm]{figures/at9s.jpg}
  \caption{トランスミッタ}
  \label{fig:transmitter}
  \end{figure}
\end{enumerate}

\
\subsection{組み立ておよび動作確認}\label{assembly-drone}

ドローンの組み立ておよび動作確認はフライトコントローラの説明書\cite{quadrino_nano_instruction:online}に従い, 次のように行った. 

\begin{enumerate}
\renewcommand{\labelenumi}{(\arabic{enumi})}
\item
  内容物の確認
\item
  機体の組み立て・源および信号線の配線
\item
  動作確認
\end{enumerate}

\begin{figure}[H]
\centering
\includegraphics[width=11cm]{figures/drone_parts.png}
\caption{ドローンの部品}
\label{fig:drone_parts}
\end{figure}

\begin{figure}[H]
\centering
\includegraphics[width=11cm]{figures/drone_block.pdf}
\caption{ドローンのブロック図}
\label{fig:drone_block}
\end{figure}

\newpage

\
\section{バイノーラルマイクの製作}\label{binaural-mic}

\
\subsection{マイクについて}\label{about-mic}

収音には他の研究で行った, 音像定位実験のために製作したバイノーラルマイクを流用した. 

使用した素子は秋月電子のエレクトレットコンデンサマイクロホン(SPL (Hong Kong) Limited, XCM6035)である. 左右用にそれぞれエレクトレットコンデンサマイクはロボットケーブルとはんだ付けし, ケーブルの端をステレオミニプラグと接続して製作した. 

  \begin{figure}[H]
    \centering
    \includegraphics[width=10cm]{figures/binaural_mic.png}
    \caption{バイノーラルマイク}
    \label{fig:binaural_mic}
  \end{figure}

\subsection{使用機器}\label{used-equipments-mic}
  \begin{enumerate}
  \renewcommand{\labelenumi}{(\arabic{enumi})}
  \item
    エレクトレットコンデンサマイク XCM6035 株式会社秋月電子通商
    % \begin{figure}[H]
    %   \centering
    %   \includegraphics[width=13cm]{figures/microphone.jpg}
    %   \caption{エレクトレットコンデンサマイク}
    %   \label{fig:microphone}
    % \end{figure}
  \item
    シールドスリムロボットケーブル KRT-SW 株式会社秋月電子通商
    % \begin{figure}[H]
    %   \centering
    %   \includegraphics[width=13cm]{figures/sielded_robot_cable.jpg}
    %   \caption{シールドスリムロボットケーブル}
    %   \label{fig:sield_slim_robot_cable}
    % \end{figure}
    % \begin{figure}[H]
    %   \centering
    %   \includegraphics[width=13cm]{figures/sielded_robot_cable_size.jpg}
    %   \caption{シールドスリムロボットケーブルの大きさ}
    %   \label{fig:sield_slim_robot_cable_size}
    % \end{figure}
  \item
    3.5mmステレオミニプラグ MP-319 株式会社秋月電子通商
    % \begin{figure}[H]
    %   \centering
    %   \includegraphics[width=13cm]{figures/mini_plug.jpg}
    %   \caption{3.5mmステレオミニプラグ}
    %   \label{fig:mini_plug}
    % \end{figure}
    % \begin{figure}[H]
    %   \centering
    %   \includegraphics[width=13cm]{figures/mini_plug_size.jpg}
    %   \caption{3.5mmステレオミニプラグの大きさ}
    %   \label{fig:mini_plug_size}
    % \end{figure}
  \end{enumerate}

\newpage

\
\section{Raspberry Piについて}\label{about-raspberry}

Raspberry Piは英国のラズベリーパイ財団によって開発されている, ARMプロセッサを搭載したシングルボードコンピュータである. Raspberr
Piは教育用として製作されたが, 現在ではIoT製品開発などの業務や人工衛星のOBC (On Board Computer) にも使用されている. 

Raspberry PiはLinux系のOSで動作するためソフトウェア開発に強みをもち, GPIOピンを通してSPI, I2C, I2Sなどの通信を行えるため, センサなどを用いた開発を容易に行える. また, USB端子を搭載し, Wi-Fi, Bluetooth接続も可能で, プロタイプ開発に適したデバイスである. 

\subsection{OSの選定}\label{choose-os}

Raspberry Piで使用可能なOSには

\begin{itemize}
\tightlist
\item
  電子工作などに適した公式OS Raspbian
\item
  LinuxディストリビューションのUbuntuから派生した Ubuntu MATE
\item
  Microsoft Windows 10
\end{itemize}

などが存在する. 

本研究では主にGPIOを使用して開発を行うため, Raspbianを使用した. なお, OSのバージョンは10.1
Buster Liteである. また, カーネルのバージョンは4.19.75-v7である. 

\subsection{初期設定について}\label{about-setup}

Raspberry Piを実験で使用する準備として, 以下の手順で初期設定を行った. 

\begin{enumerate}
\renewcommand{\labelenumi}{(\arabic{enumi})}
\item
  OSのインストール
\item
  地域, 言語の設定 \\
  \ref{raspi_config}を実行し, Localization Optionsを選択して, 地域を日本, 言語を英語に設定した. 
  \begin{lstlisting}[caption=Raspberry Piのコンフィグレーションツールの起動コマンド, label=raspi_config]
  sudo raspi-config
  \end{lstlisting}
\item
  sshの設定 \\
  \ref{raspi_config}を実行し, Interfacing Optionsを選択して, SSHを有効に設定した. 
\item
  ネットワーク・プロキシに関する設定 \\
  \texttt{/etc/wpa\_supplicant/wpa\_supplicant.conf, \\
  /etc/apt/apt.conf, 
  /etc/dhcpcd.conf, 
  \textasciitilde/.bashrc, \\
  \textasciitilde/.curlrc, 
  \textasciitilde/.wgetrc
  }を編集し, ネットワーク・プロキシに関する設定を行った. 

\item
  アップデート \\
  \ref{update_command}を実行し, Raspberry Piを最新の状態に更新した. 

  \begin{lstlisting}[caption=Raspberry Piのアップデートコマンド,label=update_command]
  sudo apt update 
  sudo apt upgrade -y 
  sudo apt dist-upgrade
  sudo rpi-update
  sudo reboot
  \end{lstlisting}
\end{enumerate}

\
\subsection{AD変換用の拡張ボードについて}\label{about-adc}

Raspberry PiはADC(ADコンバータ)を搭載していないため, マイクからの入力信号を扱うにはADコンバータを導入する必要がある. 

本研究で用いたのはマルツエレック株式会社のPumpkin Piである. PumpkinPiは計測用とオーディオ用のデュアルA-Dコンバータを搭載しており, Raspberry Piにオーディオ入力, アナログ入力機能を加えることが可能となる. 

Pumpkin Piの仕様を以下に示す. 

\begin{itemize}
\tightlist
\item
  対応OS Raspbian
\item
  対応機種 Raspberry Pi Model B+/Raspberry Pi 2 Model B/Raspberry Pi 3
  Model B
\item
  LED出力 1点
\item
  赤外線リモコン機能 送受信
\item
  オーディオコネクタ φ3.5mmステレオミニジャック
\item
  オーディオ入力 量子化ビット数=24,サンプリング周波数=48/96kHz
\item
  計測用AD変換 2チャンネル,16ビット
\item
  本体寸法 65(W)×56(D)mm
\item
  本体重量 約25g
\end{itemize}

\begin{figure}[H]
\centering
\includegraphics[width=8cm]{figures/pumpkin_pi.jpg}
\label{fig:pumpkin_pi}
\caption{PumpkinPi}
\end{figure}

\
\subsection{初期設定}\label{adc-setup}

PumpkinPiの初期設定はトランジスタ技術トランジスタ技術
2017年1月号\cite{transistor:online}にしたがって行った. 以下に簡易的な手順を示す. 

\begin{enumerate}
\renewcommand{\labelenumi}{(\arabic{enumi})}
\item
  Pumpkin Piを使用するためのRaspberry Pi固有の設定

  まず適当な作業ディレクトリで以下のコマンドを実行する. 

\begin{lstlisting}[caption=PumpkinPiの設定ファイルダウンロードコマンド,label=]
wget http://einstlab.web.fc2.com/RaspberryPi/PumpkinPi.tar
tar xvf PumpkinPi.tar
cd PumpkinPi
./PumpkinPi.sh 
\end{lstlisting}
\item
  カーネルとデバイス・ドライバのバージョンの確認

  カーネルのバージョンとデバイス・ドライバのバージョンは同じである必要がある. カーネルのバージョンは\texttt{uname\ -r}で, デバイス・ドライバのバージョンは\texttt{modinfo\ snd\_soc\_pcm1808\_adc.ko}でそれぞれ確認できる. 

\item
  ADコンバータ用のデバイス・ドライバのインストール

  次の2つのデバイス・ドライバをインストールする. 

  \begin{enumerate}
  \def\labelenumii{\arabic{enumii}.}
  \tightlist
  \item
    pcm1808-adc.ko\\
    PCM1808固有の動作を決定するドライバ. 
  \item
    snd\_soc\_pcm1808\_adc.ko\\
    Raspberry Piのサウンドとして属性を決定するドライバ
  \end{enumerate}

  まず, ホームディレクトリにPumpkinPi.tarをダウンロードして展開する. 

\begin{lstlisting}[caption=デバイス・ドライバダウンロードコマンド,label=pumpkin_pi_donwload]
cd 
wget http://einstlab.web.fc2.com/RaspberryPi/PumpkinPi.tar
tar xvf PumpkinPi.tar
cd PumpkinPi/Driver
\end{lstlisting}

  次にデバイス・ドライバをインストールする. 

\begin{lstlisting}[caption=デバイス・ドライバのインストールコマンド,label=install_device_driver]
sudo cp Backup/pcm1808-adc.bak/ /lib/modules/`uname -r`/kernel/sound/soc/codecs/pcm1808-adc.ko
sudo cp Backup/snd_soc_pcm1808_adc.bak /lib/modules/`uname -r`/kernel/sound/soc/bcm/snd_soc_pcm1808_adc.ko
sudo depmod -a  # 依存関係を調整
\end{lstlisting}

  OSのカーネル4.4以降ではデバイス・ツリー構造を導入してあるため, デバイス・ツリー情報ファイルをコピーする. 

\begin{lstlisting}[caption=デバイス・ツリー情報ファイルのコピーコマンド,label=device_tree]
sudo cp pcm1808-adc.dtbo /boot/oberlays/
\end{lstlisting}

  最後にデバイス・ドライバが電源起動時に自動的に読み込まれるように\texttt{/boot/config.txt}に\texttt{dtoverlay=pcm1808-ad}を追加する. 

  以上の作業を完了した後, 再起動することで設定が適用される. 
\end{enumerate}
