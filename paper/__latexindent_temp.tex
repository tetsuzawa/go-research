% latex uft-8
\documentclass[a4paper, twocolumn]{ltjsarticle}
%
\usepackage{graphicx}
\usepackage{here}
\usepackage{booktabs}
\usepackage{amsmath}
\usepackage{amssymb}
\usepackage{comment}
\usepackage[labelsep=space]{caption}
\usepackage[subrefformat=parens, labelsep=space]{subcaption}
\usepackage{siunitx}
\usepackage{unicode-math}

\captionsetup{compatibility=false}

\setlength{\textwidth}{\fullwidth}
\setlength{\textheight}{40\baselineskip}
\addtolength{\textheight}{\topskip}
\setlength{\voffset}{-0.55in}


\title{Raspberry Piを使用した\\ドローンの雑音に対するADF適用の検討}
\author{電気電子工学科 5年 19番 瀧澤哲}
\date{2019年7月4日}
%
\begin{document}
% START DOCUMENT
%

\maketitle
  
\section{研究背景}
  近年

\section*{課題}
  ある平等電界下において\(\alpha = 7\),\(\gamma = 0.0009\)であった。陰極から1個の電子がスタートした場合,\(d=1\)[cm]における電流密度を\(\alpha\)作用のみ考慮した場合と,\(\alpha\),\(\gamma\)を考慮した場合について計算せよ。



\section*{解答}
  \subsection*{\(\alpha\)作用のみ考慮}
    \(\alpha\)作用のみを考慮した場合の電流密度\(I\)[A/cm^2]は次式で求められる\footnote{ここでのeは自然対数の底であることに注意} (\(\alpha\):電子の衝突電離係数,\(d\):電極間の距離 [cm])。
    \begin{equation}
      I = I_0 e^{\alpha d} 
      \label{equation:alpha_I}
    \end{equation}
    ただし,\(I_0\)は次式で定義される (\(e\):素電荷,\(n_0\):初期の電子の個数)。
    \begin{equation}
      I_0 = e n_0
      \label{equation:I_0}
    \end{equation}

    (\ref{equation:alpha_I})式,(\ref{equation:I_0})式より
    \begin{equation}
      \begin{split}
        I &= I_0 e^{\alpha d} \\
          &= e n_0 e^{\alpha d} \\
          &= 1.602 \times 10^{-19} \times 1 \times e^{7 \times 1} \\
          &= 1.76 \times 10^{-16} [\si{\ampere} / \si{\centi \meter ^ 2}]
      \end{split}
      \label{equation:alpha_I_answer}
    \end{equation} 
    となる。


  \subsection*{\(\alpha\)作用と\(\gamma\)作用を考慮}
    \(\alpha\)作用と\(\gamma\)作用を考慮した場合の電流密度\(I\)[A/cm^2]は次式で求められる (\(\alpha\):電子の衝突電離係数,\(\gamma\):正イオンが陰極に衝突する際の二次電子放出係数,\(d\):電極間の距離[cm])。
    \begin{equation}
      I = \frac{e^{\alpha d}}{1-\gamma (e^{\alpha d} - 1)} I_0
      \label{equation:gamma_I}
    \end{equation}

    従って,
    \begin{equation}
      \begin{split}
        I &= \frac{e^{\alpha d}}{1-\gamma (e^{\alpha d} - 1)} I_0 \\
          &= \frac{e^{\alpha d}}{1-\gamma (e^{\alpha d} - 1)} e n_0 \\
          &= \frac{e^{7 \times 1}}{1 - 0.0009(e^{7 \times 1} - 1)} \times 1.602 \times 10^{-19} \times 1 \\
          &= 1.26 \times 10^{-14} [\si{\ampere} / \si{\centi \meter ^ 2}]
      \end{split}
      \label{equation:gamma_I_answer}
    \end{equation}
    となる。以上。



% end document
\end{document}
